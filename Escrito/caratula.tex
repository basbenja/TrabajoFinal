% Formato: https://www.famaf.unc.edu.ar/documents/5247/ANEXO_II_RHCD-2024-572-E-UNC-DECFAMAF.pdf
\begin{titlepage}
\begin{center}
    % Configuración del espaciado
    \setstretch{1.5}

    \noindent
    \begin{minipage}[t]{0.5\linewidth}
        \flushleft
        \includegraphics[width=6cm]{logos/unc.png}
    \end{minipage}%
    \begin{minipage}[t]{0.5\linewidth}
        \flushright
        \includegraphics[width=6cm]{logos/famaf.png}
    \end{minipage}

    \vspace{2cm}

    {\LARGE \textbf{Aprendizaje Automático para la Selección de Grupos de Control en
    Evaluación de Impacto}}
    \\[1cm]
    {\large por}
    \\[0.5cm]
    {\Large \textbf{Benjamín Bas Peralta}}

    \vspace{1cm}

    \noindent
    Presentado ante la \textbf{FACULTAD DE MATEMÁTICA, ASTRONOMÍA, FÍSICA Y COMPUTACIÓN}
    como parte de los requerimientos para la obtención del grado de Licenciado en Ciencias
    de la Computación de la

    \textbf{UNIVERSIDAD NACIONAL DE CÓRDOBA}

    \vspace{0.5cm}

    Agosto, 2025

    \vspace{0.5cm}

    Directores: Dr. Martín Domínguez \\ Mgter. David Giuliodori

    \vfill

    \begin{center}
        \includegraphics[width=3cm]{logos/creative-commons.png}\\
        \vspace{0.3cm}
        Este trabajo se distribuye bajo una Licencia Creative Commons \\
        \href{https://creativecommons.org/licenses/by-nc-sa/4.0/deed.es}
        {Atribución - No Comercial - Compartir Igual 4.0 Internacional.}
    \end{center}

\end{center}
\end{titlepage}