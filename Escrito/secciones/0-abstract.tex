\documentclass[../main.tex]{subfiles}
\graphicspath{{\subfix{../images/}}}

\begin{document}

% Que es la evaluacion de impacto
% Grupo de control -> sesgo de autoseleccion?
% Hablar concretamente del PSM
% Variables temporales
% Tecnicas utilizadas en este trabajo (redes neuronales LSTM y convolucionales)
% Resultados preliminares?
\noindent \textbf{Resumen}

La evaluación de impacto es una metodología que se utiliza para cuantificar los efectos
que ha tenido la implementación de una política o tratamiento sobre un conjunto de
variables objetivo de la población receptora. Su objetivo principal es estimar la
diferencia entre los resultados observables en los tratados y lo que hubiera ocurrido con
ellos en ausencia del tratamiento. Esta última situación contrafáctica es aproximada
utilizando un grupo de control o comparación, que idealmente debe estar libre del sesgo de
autoselección. En tratamientos sin asignación aleatoria, lograr esto es un gran desafío
para el cual existen diferentes técnicas, siendo una de ellas el Pareamiento por Puntaje
de Propensión (PSM). Tomando este método como punto de comparación, en este trabajo se
exploran alternativas basadas en redes neuronales del tipo \textit{Long-Short Term Memory}
(LSTM) y convolucionales para la identificación de individuos pertenecientes al grupo de
control en escenarios específicos donde la asignación al programa depende potencialmente
de resultados pasados.

Se trabajó con datos de panel generados sintéticamente, en los que se modelaron diferentes
comportamientos temporales. Los resultados obtenidos indican que ...

\medskip

\noindent \textbf{Palabras Clave: } Evaluación de Impacto, Grupo de Control, Aprendizaje
Automático, Redes Neuronales, Series de Tiempo.

\bigskip
\bigskip

\noindent \textbf{Abstract}

An impact assessment is a methodology used to quantify the effects that the implementation
of a policy or treatment has had on a set of target variables of the receiving population.
Its main objective is to estimate the difference between the observable results in the
treated and what would have happened to them in absence of the treatment. This last
counterfactual situation is approximated by using a comparison or control group, which
idealy should be free from self-selection bias. In non-randomized treatments, achieving
this presents a great challenge for which different techniques exists, being one of them
the Propensity Score Matching (PSM). Taking this method as a benchmark, this work explores
alternatives based on Long-Short Term Memory (LSTM) and convolutional neural networks for
identifying individuals belonging to the control group in specific scenarios where program
assignment potentially depends on past outcomes.

We worked with synthetically generated panel data, in which we modeled different temporal
behaviors. The results show that ...

\medskip

\noindent \textbf{Keywords: } Impact Assessment, Control Group, Machine Learning, Neural
Networks, Time Series.

\end{document}