\documentclass[../main.tex]{subfiles}
\graphicspath{{\subfix{../images/}}}

\begin{document}

\noindent \textbf{Resumen}

La evaluación de impacto es una metodología que se utiliza para cuantificar los efectos
que ha tenido la implementación de una política o tratamiento sobre un conjunto de
variables objetivo de la población receptora. Su objetivo principal es estimar la
diferencia entre los resultados observables en los tratados y lo que hubiera ocurrido con
ellos en ausencia del tratamiento. Esta última situación contrafáctica es aproximada
utilizando un grupo de control o comparación, que idealmente debe estar libre del sesgo de
autoselección. En tratamientos sin asignación aleatoria, lograr esto es un gran desafío
para el cual existen diferentes técnicas, siendo una de ellas el pareamiento por puntaje
de propensión (PSM). Tomando este método como punto de comparación, en este trabajo se
exploran alternativas basadas en redes neuronales del tipo \textit{long short-term memory}
(LSTM) y convolucionales para la identificación de individuos adecuados para formar parte
del grupo de control en escenarios específicos donde la asignación al programa depende
potencialmente de resultados pasados y la implementación se hace en múltiples cohortes.

\medskip

\noindent \textbf{Palabras clave: } Evaluación de impacto, grupo de control, pareamiento
por puntaje de propensión, aprendizaje automático, redes neuronales, series de tiempo.

\bigskip
\bigskip

\noindent \textbf{Abstract}

An impact evaluation is a methodology used to quantify the effects that the implementation
of a policy or treatment has had on a set of target variables of the receiving population.
Its main objective is to estimate the difference between the observable results in the
treated and what would have happened to them in absence of the treatment. This last
counterfactual situation is approximated by using a comparison or control group, which
ideally should be free from self-selection bias. In non-randomized treatments, achieving
this presents a great challenge for which different techniques exists, being one of them
the propensity score matching (PSM). Taking this method as a benchmark, this work explores
alternatives based on long short-term memory (LSTM) and convolutional neural networks to
identify suitable individuals for the control group in specific scenarios where program
assignment potentially depends on past outcomes and the implementation occurs across
multiple cohorts.

\medskip

\noindent \textbf{Keywords: } Impact evaluation, control group, propensity score matching,
machine learning, neural networks, time series.

\end{document}