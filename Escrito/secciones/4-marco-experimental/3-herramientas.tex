\documentclass[../../main.tex]{subfiles}
% \graphicspath{{\subfix{../images/}}}

\begin{document}

A continuación, nombramos las herramientas que nos ayudaron a llevar a cabo los
experimentos, desde la generación de datos sintéticos hasta el diseño, entrenamiento y
evaluación de los modelos, junto con la búsqueda de hiperparámetros óptimos.

\subsection{Hardware}
% nabu (Centro de Cómputo de Alto Desempeño)
% Mencionar las características (buscar en la página o preguntarle a Nico)

\subsection{Python}
Python es un lenguaje de programación interpretado, orientado a objetos y con tipado
dinámico. Es multipropósito, por lo que se puede usar para una diversiad de tareas. Otros
de sus aspectos más relevantes son su sintaxis simple e intuitiva con énfasis en la
legibilidad, el alto grado de abstracción que provee, la gran cantidad de librerías que
existen, y el extenso soporte de la comunidad. Además, es gratis, de código abierto y
multiplataforma, es decir puede ejecutarse en diferentes sistemas operativos.

Durante los últimos años, ha sido el lenguaje por defecto en las áreas de Ciencia
de Datos y Aprendizaje Automático, principalmente por el desarrollo de librerías
como pandas, numpy, PyTorch, TensorFlow, y Scikit-learn, entre muchas otras.

\subsection{Pandas}
Pandas es una librería de Python que provee estructuras de datos rápidas, flexibles y
expresivas que permiten trabajar de manera fácil e intuitiva con datos tabulares. Las dos
principales estructuras de datos de Python son las \texttt{Series} (unidimensionales) y
los \texttt{DataFrames} (bidimensionales).

\subsection{PyTorch}
% Diseñar modelos y entrenarlos.
% Principal estrucutra de datos: tensores.
% Permite hacer uso de la GPU!
PyTorch es una librería de Python 



\subsection{Numpy}

\subsection{Jupyter}
% Para llevar a cabo el desarrollo
% Entorno interactivo que se puede correr en el editor de código Visual Studio Code
% Permite: celdas de código Python, celdas Markdown, ir viendo el progreso, hacer chequeos
% manuales intermedios

\subsection{Optuna}
% Optimización de hiperparámetros
% Paralelización!

\subsection{MLflow}
% Registro de experimentos: hiperparámetros, modelo utilizado, datasets, resultados
% (matriz de confusión, classification report)

\end{document}