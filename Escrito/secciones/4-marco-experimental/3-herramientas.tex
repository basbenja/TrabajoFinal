\documentclass[../../main.tex]{subfiles}
% \graphicspath{{\subfix{../images/}}}

\begin{document}

A continuación, nombramos las herramientas que nos ayudaron a llevar a cabo los
experimentos, desde la generación de datos sintéticos hasta el diseño, entrenamiento y
evaluación de los modelos, junto con la búsqueda de hiperparámetros óptimos.

\subsection{Hardware}
% nabu (Centro de Cómputo de Alto Desempeño)
% Mencionar las características (buscar en la página o preguntarle a Nico)

\subsection{Python}
% Lenguaje de programación multipropósito
% Gran variedad de librerías y enorme soporte de la comunidad
% El más utilizado durante los últimos años para entrenar modelos

\subsection{PyTorch}
% Diseñar modelos y entrenarlos.
% Principal estrucutra de datos: tensores.
% Permite hacer uso de la GPU!

\subsection{Pandas}

\subsection{Numpy}

\subsection{Jupyter}
% Para llevar a cabo el desarrollo
% Entorno interactivo que se puede correr en el editor de código Visual Studio Code
% Permite: celdas de código Python, celdas Markdown, ir viendo el progreso, hacer chequeos
% manuales intermedios

\subsection{Optuna}
% Optimización de hiperparámetros
% Paralelización!

\subsection{MLflow}
% Registro de experimentos: hiperparámetros, modelo utilizado, datasets, resultados
% (matriz de confusión, classification report)

\end{document}