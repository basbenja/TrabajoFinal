\documentclass[../../main.tex]{subfiles}
% \graphicspath{{\subfix{../images/}}}

\begin{document}
A continuación, nombramos las herramientas que más nos ayudaron a llevar a cabo los
experimentos, desde la generación de datos sintéticos hasta el diseño, entrenamiento y
evaluación de los modelos, junto con la búsqueda de hiperparámetros óptimos. En la
sección de bibliografía, se incluye el enlace a su documentación.

\subsection{Hardware}
Para el entrenamiento de los modelos y la búsqueda de hiperparámetros, utilizamos recursos
computacionales de UNC Supercómputo (CCAD) de la Universidad Nacional de Córdoba
\cite{ccad}, que forman parte del Sistema Nacional de Computación de Alto Desempeño
(SNCAD) de la República Argentina. En particular, hicimos uso de la Computadora
Nabucodonosor \cite{nabu}, destinada especialmente a potenciar el uso del ML. Cuenta con 2
procesadores Xeon E5-2680v2 de 10 núcleos cada uno, 64 GiB de RAM, y 3 GPU
(\textit{Graphic Processing Unit}, placas gráficas) NVIDIA GTX 1080Ti.

\subsection{Python}
Python \cite{python-docs} es un lenguaje de programación interpretado, orientado a objetos
y con tipado dinámico. Es multipropósito, por lo que se puede usar para una diversiad de
tareas. Otros de sus aspectos más relevantes son su sintaxis simple e intuitiva con
énfasis en la legibilidad, el alto grado de abstracción que provee, la gran cantidad de
librerías que existen, y el extenso soporte de la comunidad. Además, es gratis, de código
abierto y multiplataforma, por lo que puede ejecutarse en diferentes sistemas operativos.

Durante los últimos años, ha sido el lenguaje más utilizado en las áreas de Ciencia
de Datos y Aprendizaje Automático, principalmente por el desarrollo de librerías
como pandas, numpy, PyTorch, TensorFlow, y Scikit-learn, que facilitan el manejo
de grandes volúmenes de datos y la utilización de algoritmos de ML.

En nuestro caso, utilizamos Python para toda la implementación de este trabajo.

\subsection{Jupyter}
Jupyter \cite{jupyter-docs} es una plataforma que ofrece un entorno de desarrollo
interactivo a través de documentos llamados ``\textit{notebooks}'', organizados en
unidades denominadas ``celdas''. La potencia de estas notebooks radica en que pueden
contener al mismo tiempo celdas de código, texto en lenguaje natural escrito en Markdown,
imágenes, y diversas otras representaciones, como tablas y diagramas. Estos documentos
proporcionan una forma rápida y práctica de prototipar y explicar código, explorar y
visualizar datos, y realizar chequeos intermedios observando la salida de cada celda, lo
cual contribuye a prevenir errores. Jupyter soporta diferentes lenguajes de programación y
permite la utilización de entornos virtuales.

En este proyecto, implementamos un \textit{pipeline} mediante notebooks de Jupyter que
abarca desde la carga, visualización y transformación de los datos, hasta la selección,
entrenamiento y evaluación de los modelos.



\subsection{Pandas}
Pandas \cite{pandas-docs} es una librería de Python que provee estructuras de datos
rápidas, flexibles y expresivas que permiten trabajar de manera fácil e intuitiva con
datos tabulares. Las dos principales estructuras de datos de pandas son las
\texttt{Series} (unidimensionales) y los \texttt{DataFrames} (bidimensionales). En cada
una, se pueden utilizar diferentes tipos de datos, como números enteros, decimales,
palabras, fechas, entre otros. Algunas de sus capacidades destacadas incluyen la lectura y
escritura de datos en diferentes formatos (CSV, Microsoft Excel, Stata, etcétera), el
manejo de datos faltantes, y la simple y eficiente transofrmación de datos.

Utilizamos \texttt{DataFrames} para almacenar los datos sintéticos en archivos y luego
cargarlos, y para construir los conjuntos de entrenamiento y test.

\subsection{NumPy}
NumPy \cite{numpy-docs} es una librería de Python que permite el cómputo numérico. Ofrece
arreglos N-dimensionales, funciones matemáticas variadas, generadores de números
aleatorios, rutinas de álgebra lineal, y más. El núcleo de NumPy es código C optimizado,
por lo que combina la flexiblidad de Python con la rapidez propia del código compilado.
Estas características hacen que sea la base de muchas otras librerías utilizadas en
ámbitos muy diversos.

Empleamos numpy principalmente para crear las series de tiempo de los diferentes
individuos en matrices, que luego pasamos a \texttt{DataFrames}.

\subsection{PyTorch}
PyTorch \cite{pytorch-docs} es una librería de Python que facilita la utilización de redes
neuronales, tanto en su diseño como en su entrenamiento. Provee cómputo acelerado mediante
el uso de placas gráficas, e implementa internamente las optimizaciones numéricas
necesarias para entrenar un modelo, siendo la más importante la del algoritmo de
backpropagation. La mayoría de PyTorch está escrito en C++ y CUDA.

Aquí, utilizamos PyTorch justamente para diseñar las diferentes arquitecturas de redes
neuronales, que se implementan como clases en Python, y para llevar a cabo el ciclo de
entrenamiento de los distintos modelos.

\subsection{Scikit-learn}
Scikit-learn \cite{scikitlearn-docs} es una biblioteca de Python de código abierto para
Aprendizaje Automático. Provee varias herramientas útiles que facilitan tareas como el
preprocesamiento de datos, el entrenamiento y la evaluación de modelos, el cálculo de
métricas, la búsqueda de hiperparámetros, entre otros.

En nuestro caso, usamos Scikit-learn para automatizar el proceso de validación cruzada
durante la búsqueda de hiperparámetros, y para la evaluación final de los modelos mediante

el cálculo automático de diversas métricas, incluyendo la matriz de confusión.

\subsection{Optuna}
Optuna \cite{optuna-docs} es una librería de Python que permite realizar la optimización
automática de hiperparámetros de una manera eficiente. Resulta intuitiva de utilizar y su
incorporación en un pipeline es bastante directa. Un aspecto muy útil es que incluye un
\textit{dashboard} interactivo que posibilita la visualización en tiempo real de los
valores de hiperparámetros evaluados en los diferentes ensayos, como así también de
gráficos que reflejan la relación entre estos valores y el resultado obtenido con ellos
(medido a través de una métrica específica, determinada por el usuario). Otra
característica relevante es que Optuna permite la paralelización de las evaluaciones, lo
que contribuye a acelerar el proceso de optimización.

\subsection{MLflow}
MLflow \cite{mlflow-docs} es una librería de Python que permite llevar un registro fino de
las diferentes etapas y los diferentes experimentos llevados a cabo en un proyecto de ML.
Provee una interfaz con la cual se pueden crear diferentes proyectos, y dentro de cada uno
diferentes experimentos. En cada uno de ellos, permite cargar los parámetros y valores que
el usuario crea adecuados (como pueden ser los hiperparámetros empleados para entrenar el
modelo), los datasets utilizados, y diferentes archivos que pueden incluir gráficos,
métricas, y hasta incluso los pesos del modelo entrenado.

\subsection{Stata}
Stata \cite{stata-docs} es un paquete de software estadístico que ofrece una gran variedad
de herramientas para el análisis exhaustivo de datos, incluyendo tareas de manipulación,
exploración, visualización e inferencia estadística. Es utilizado en disciplinas como
economía, políticas públicas, educación y salud pública, entre otras. Además, ofrece
compatibilidad con Python mediante la biblioteca pystata, la cual permite integrar de
forma eficiente los objetos DataFrame de pandas con las funcionalidades de Stata.

En este trabajo, utilizamos Stata para la implementación del PSM, específicamente a través
de las funciones \texttt{logit} y \texttt{psmatch2}.

\end{document}