\documentclass[../../main.tex]{subfiles}

\begin{document}
Como mencionamos en el capítulo anterior, uno de nuestros objetivos es obtener resultados
que permitan comparar el desempeño de las redes neuronales con el del PSM en los
distintos escenarios. Las diferencias metodológicas inherentes a ambas técnicas hacen que
esto sea un desafío. En esta sección, describimos la estrategia propuesta en este trabajo
para abordarlo.

Para empezar, a partir de los datos generados, construimos los conjuntos de entrenamiento
y test de la siguiente manera, con la idea que los modelos aprendan los patrones vistos en
los tratados para luego poder identificar los controles:
\begin{itemize}
    \item \textbf{Conjunto de entrenamiento}: individuos tratados de todas las cohortes
    (1000 en total) y una muestra aleatoria de 1000 NiNis.
    \item \textbf{Conjunto de test}: individuos de control de todas las cohortes (1000 en
    total) y una muestra aleatoria de 1500 NiNis, distinta a la utilizada en el conjunto
    de entrenamiento.
\end{itemize}

Un detalle a tener en cuenta es que tanto tratados como controles tienen una
característica extra: el inicio de programa. Sin embargo, esta no está presente en los
NiNis, justamente porque no fueron tratados y porque no forman parte del grupo de
comparación en ninguna de las cohortes. Para reflejar esto e intentar ``transmitírselo''
al modelo, por cada NiNi incluido, generamos versiones múltiples de cada uno, simulando
cómo se verían si pertenecieran a distintas cohortes. Puntualmente, a cada uno lo
repetimos tantas veces como inicios de programa (o cohortes) haya, utilizando en cada
repetición los períodos previos al respectivo inicio. En todas las repeticiones, la
etiqueta asignada es 0. Por lo tanto, los conjuntos de entrenamiento y test finales fueron
los siguientes:
\begin{itemize}
    \item \textbf{Conjunto de entrenamiento}:
        \begin{itemize}
            \item Individuos tratados de todas las cohortes (1000).
            \item 1000 individuos NiNi, cada uno repetido tantas veces como cohortes haya,
            con los períodos previos al inicio de programa de cada cohorte.
        \end{itemize}
    \item \textbf{Conjunto de test}:
        \begin{itemize}
            \item Individuos de control de todas las cohortes (1000).
            \item 2500 individuos NiNi, cada uno repetido tantas veces como cohortes haya,
            con los períodos previos al inicio de programa de cada cohorte.
        \end{itemize}
\end{itemize}

Ahora bien, para evaluar las redes, se toman los resultados obtenidos en el conjunto de
test, el cual permanece ``oculto'' durante el proceso de entrenamiento. Sin embargo, en el
PSM no existe esta noción: todos los individuos disponibles son utilizados para estimar
los coeficientes de la regresión logística, calcular las probabilidades y hacer el
emparejamiento.

Además, otra distinción fundamental es que las redes, incluyendo la feature de inicio de
programa, posibilitan la utilización de un solo modelo para evaluarlo sobre las distintas
cohortes, mientras que el PSM requiere realizar una estimación por cohorte.

Debido a estas diferencias identificadas, para poder hacer una comparación razonable,
adaptamos los dos métodos, las redes y el PSM. Para el PSM, lo que hicimos fue modificarlo
de tal forma que permita incorporar la noción de ``entrenamiento'' y ``test''; y para las
redes, cambiamos la forma de evaluarlas para poder hacerlo por cohorte. Los detalles del
procedimiento propuesto se encuentran en la Tabla \ref{tab:comparacion}.

\begin{table}[ht]
    \centering
    \renewcommand{\arraystretch}{1.3} % opcional: más espacio vertical entre filas
    \begin{tabularx}{\textwidth}{|c|X|X|}
        \hline
         & \multicolumn{1}{c|}{\textbf{Redes neuronales}} & \multicolumn{1}{c|}{\textbf{PSM}} \\ \hline
        \textbf{Entrenamiento}
            & Se utiliza el conjunto de entrenamiento en su totalidad para construir
            un único modelo capaz de incorporar diferentes cohortes.

            & Para cada cohorte, se estima una regresión logística utilizando los tratados
            de la cohorte y los NiNi del conjunto de entrenamiento correspondientes al
            inicio de programa de la cohorte. % Al realizar estas estimaciones para
            % todas las cohortes, se terminan utilizando todos los tratados y NiNi disponibles
            % en el conjunto de entrenamiento.
        \\ \hline
        \textbf{Evaluación}
            & Se divide al conjunto de test en tantos subconjuntos como cohortes haya.
            Cada uno de ellos está formado por los controles de cada cohorte y los NiNi
            del conjunto de test correspondientes a cada cohorte.

            Con el modelo entrenado, se evalúa en cada uno de estos subconjuntos, y el
            resultado final es el promedio de los obtenidos en cada partición.

            & Se divide al conjunto de test en tantos subconjuntos como cohortes haya.
            Cada uno de ellos está formado por los controles de cada cohorte y los NiNi
            del conjunto de test correspondientes a cada cohorte.

            Con la regresión estimada en cada cohorte, se hace la inferencia de la
            probabilidad de participación sobre los tratados y la partición de test
            correspondientes a la cohorte; y con ello se lleva a cabo el emparejamiento.
            Se evalúa el emparejamiento en cada cohorte, y el resultado final es el
            promedio de los obtenidos en cada cohorte.
        \\ \hline
    \end{tabularx}
    \caption{Comparación entre el procedimiento de evaluación de redes neuronales y PSM.}
    \label{tab:comparacion}
\end{table}

% Con estos conjuntos en mente, propusimos el siguiente procedimiento para poder
% evaluar el PSM, aplicado por cohorte:
% \begin{enumerate}[label=\textbf{\arabic*.}]
%     \item Se toman los tratados \textbf{de la cohorte} y los NiNi del conjunto de entrenamiento
%     con la repetición correspondiente al inicio de programa de la cohorte.
%     \item Con tratados, controles y NiNi (\textbf{todos}), se estiman los coeficientes de
%     la regresión logística.
%     \item Una vez ajustada la logit, se realiza inferencia sobre los tratados, controles y
%     NiNi \textbf{del conjunto de test}, obteniendo la probabilidad de participación
%     para cada uno.
%     \item Se lleva a cabo el emparejamiento entre tratados y no tratados.
%     \item Finalmente, se consideran como controles predichos por el PSM aquellos
%     individuos que hayan sido emparejados con algún tratado. El resto de los individuos no
%     tratados (excluyendo los tratados) se consideran como clasificados en la clase 0 por
%     el PSM.
% \end{enumerate}
% Así, el resultado final del PSM para una simulación con más de una cohorte es el puntaje
% \(F_1\) promedio entre las cohortes.

% \bigskip
% Resulta pertinente hacer algunas observaciones sobre la forma de comparación propuesta:
% \begin{itemize}
%     \item En primer lugar, si bien en las redes la evaluación se hace sobre un solo
%     conjunto y en el PSM, se van haciendo evaluaciones ``parciales'', es importante notar
%     que al final, el conjunto de test está formado por los mismos individuos en ambas
%     técnicas. Mientras que el de las redes considera a todas las cohortes al mismo tiempo,
%     en el PSM los individuos son ``incluidos'' de a cohortes.
%     \item En segundo lugar, como en el PSM la estimación se lleva a cabo por cohorte, no
%     es necesario incluir la variable correspondiente al inicio del programa, añadida en
%     las redes para poder trabajar con todas las cohortes de una sola vez.
%     \item Por último, cabe destacar que si bien en el conjunto de test de las redes se
%     repite cada individuo de control con una etiqueta positiva (1) - lo cual implica una
%     relajación del objetivo de identificación de controles por cohorte -, en el PSM no se
%     realiza esta repetición, ya que el propio enfoque - al trabajar por cohorte -
%     restringe naturalmente los candidatos a cada caso, otorgándole así al modelo una
%     especie de ``ventaja'' o ``pista'' adicional.
% \end{itemize}

\end{document}