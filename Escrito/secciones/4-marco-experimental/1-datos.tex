\documentclass[../../main.tex]{subfiles}

\begin{document}
Nuestro entorno de simulación está diseñado para generar datos sintéticos que imiten
escenarios reales en donde la asignación de individuos al tratamiento es:
\begin{itemize}[itemsep=0.05cm]
    \item No aleatoria.
    \item Secuencial, es decir que hay cohortes.
    \item Potencialmente dependendiente de resultados pasados.
\end{itemize}

Concretamente, el proceso de generación de datos involucra la creación de una serie de
tiempo univariada para cada individuo o unidad, que incorpora efectos fijos y componentes
autoregresivos, resultando así en un conjunto de datos de panel. Distinguimos entre tres
tipos de individuos:
\begin{itemize}[itemsep=0.1cm]
    \item \textbf{Individuos tratados}: son aquellos que han recibido el tratamiento en
    algún período, es decir que forman parte de alguna de las cohortes.
    \item \textbf{Individuos de control}: son aquellos que en los datos sintéticos,
    sabemos que forman parte del grupo de control para una determinada cohorte de
    tratados.
    \item \textbf{Individuos ``NiNi''} (``Ni tratados Ni controles''): son aquellas
    unidades que no han sido tratadas y que en los datos sintéticos, sabemos que no forman
    parte del grupo de control para ninguna de las cohortes.
\end{itemize}

Para los tratados y los controles, agregamos una característica extra además de la serie
de tiempo, a la que llamamos ``\textit{inicio de programa}''. Para los tratados, este
representa el momento en que un individuo recibió el tratamiento; y para cada unidad de
control, representa el momento en que se aplicó el programa a los individuos tratados para
los cuales efectivamente sirve como control.

Además, al tratarse de Aprendizaje Supervisado, también incluimos la etiqueta que le
corresponde a cada individuo: 1 para tratados y controles, y 0 para los NiNi. La razón por
la cual tratados y controles comparten la misma etiqueta es que el modelo aprenda durante
su entrenamiento los patrones que tienen los tratados para luego poder clasificar como
``1'' a aquellos que se asemejen a los tratados, que serán justamente los que identifique
como posibles controles.

Con estas categorías en mente, es muy importante destacar una diferencia fundamental
respecto a lo que ocurre en situaciones reales. En un entorno real, a la hora de llevar a
cabo la evaluación de impacto, solo se puede identificar quiénes han sido tratados y
quiénes no. Dentro del conjunto de no tratados, puede haber tanto individuos de control
potenciales como individuos NiNi, pero no se cuenta con información para distinguirlos
directamente.

La motivación detrás de esta construcción artificial es que tanto tratados como controles
compartan una determinada dinámica en su comportamiento temporal, mientras que los NiNis
no. Nuevamente, cabe aclarar que estas son hipótesis que en la práctica pueden resultar
difíciles de verificar. Sin embargo, nuestro interés radica en evaluar si, al modelar
estos supuestos, los algoritmos son capaces de capturar tales patrones.

Como mencionamos anteriormente, para cada individuo \(i\) generamos una serie de tiempo
univariada \(y_i\) de una longitud \(L\) que incorpora efectos fijos y componentes
autoregresivos. La fórmula que utilizamos para simular cada valor \(y_{i,t}\) de las
series, con \(t = 0, 1, ..., L\) indicando el período generado, fue la siguiente:
\begin{align}
    \mu &= \mu_{EF} + EF_i + \mu_{ET} \\
    y_{i,0} &= \mu + u_0  \ \sigma \\
    y_{i,t} &= \phi \ y_{i,t-1} + (1 - \phi) \ \mu +  u_t \ \sigma
\end{align}
donde:
\begin{itemize}[itemsep=0.1cm]
    \item \(\mu_{EF}\) representa la media del efecto fijo para todos los individuos
    de un mismo grupo.
    \item \(\mu_{ET}\) representa la media del efecto temporal en todos los pasos
    de la serie y es la misma para todos los grupos.
    \item \(EF_i\) representa el efecto fijo del individuo \(i\), constante en todos
    los pasos de tiempo.
    \item \(\phi\) es la persistencia auto-regresiva asociada a la variable para
    todos los individuos de un mismo grupo.
    \item \(u_t\) es un valor aleatorio proveniente de una distribución normal estándar
    (se genera uno en cada paso de tiempo).
    \item \(\sigma\) es la desviación estándar del término de ruido de las series,
    y es la misma para todos los grupos.
\end{itemize}

Generamos diferentes escenarios, que se diferencian entre sí por determinados aspectos:
\begin{enumerate}[itemsep=0.01cm, label=\textbf{\arabic*.}]
    \item La longitud de las series de tiempo generadas, que denotamos con \(L\). Esta
    representa la cantidad de períodos observados para cada individuo \textbf{previos al inicio
    del tratamiento}.
    \item La cantidad de períodos previos al inicio del tratamiento en los que se observa
    un comportamiento específico en tratados y controles. A estos períodos los llamamos
    ``períodos de dependencia'' y a la cantidad la denotamos con \texttt{n\_per\_dep}.
    Se va a cumplir siempre que \(0 \le \texttt{n\_per\_dep} \le L\).
    \item Parámetros específicos relacionados a la generación de las series de tiempo
    para los individuos de cada grupo.
\end{enumerate}
Y para cada escenario, generamos 100 simulaciones para poder garantizar que los resultados
obtenidos sean estadísticamente significativos.

Las configuraciones fijas a lo largo de todos los escenarios y simulaciones a la hora de
generar los datos fueron las siguientes:
\begin{itemize}[noitemsep]
    \item Cantidad total de individuos: 10000.
    \item Cantidad de cohortes: 3.
    \item Cantidad de individuos tratados: 1000, divididos en las 3 cohortes.
    \item Cantidad de individuos de control: 1000, divididos en las 3 cohortes.
    \item Cantidad de individuos NiNi: 8000.
\end{itemize}

Más adelante, veremos cómo construimos los conjuntos de entrenamiento y de test con estos
datos; y en la sección de Resultados, detallaremos los parámetros específicos de cada
escenario.

\end{document}