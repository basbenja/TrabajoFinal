\documentclass[../../main.tex]{subfiles}
% \graphicspath{{\subfix{../images/}}}

% Simulation involves multiple parameters set to create various scenarios. These parameters
% include the number of simulations (N\_simulations), the total sample size (Total\_Sample),
% the proportion of treated individuals (PorTreated), and the proportion of control
% individuals (PorControl). Additional parameters, such as autoregressive persistence
% associated with the outcome variables, are specified for the treated (phiT), controls
% (PhiC), and neither treated nor control (Nini) groups (phiNini) to model temporal
% dependencies in the data.

% A crucial aspect of our simulation is the modeling of the treatment effect, where we
% simulate a nominal impact of the treatment on the outcome variable based on a proportion
% (ImpactProportional) and with a variabilitiy in the impact on each individual that depend
% on a standard deviation (STDImpact). Furthermore, we introduce scenarios with both linear
% and nonlinear (nonlinear) dependencies in treatment assignment, reflecting the complex
% dynamics often observed in observational studies.

% To examine the network's ability to handle heterogeneous treatment effects, we simulated
% variability in impact by cohort (staggered enters). This heterogeneity allows us to test
% the robustness of the LSTM network in identifying appropriate controls in different
% subpopulations with different treatment effects.

% The data generation process involves the creation of time series for each individual,
% incorporating fixed effects, autoregressive components, and treatment impacts as
% appropriate. This process is designed to closely replicate the generation of observational
% data in studies with nonrandomized treatment assignments.


\begin{document}
Nuestro entorno de simulación está diseñado para generar \textbf{datos sintéticos} que
imiten escenarios reales en donde la asignación de individuos al tratamiento es:
\begin{itemize}[itemsep=0.05cm]
    \item No aleatoria.
    \item Secuencial (hay cohortes).
    \item Potencialmente dependendiente de resultados pasados.
\end{itemize}

Concretamente, el proceso de generación de datos involucra la creación de una serie de
tiempo univariada para cada individuo o unidad, que incorpora efectos fijos y componentes
autoregresivos. Distinguimos entre tres tipos de individuos:
\begin{itemize}[itemsep=0.1cm]
    \item \textbf{Individuos tratados}: son aquellos que han recibido el tratamiento en
    algún período, es decir que forman parte de alguna de las cohortes.
    \item \textbf{Individuos de control}: son aquellos que en los datos sintéticos,
    sabemos que forman parte del grupo de control.
    \item \textbf{Individuos ``NiNi''}: son aquellas unidades que no han sido tratadas y
    que en los datos sintéticos, sabemos que no forman parte del grupo de control.
\end{itemize}

Para los tratados y los controles, agregamos una feature extra (además de la serie de
tiempo) que es el inicio de programa. Además, al tratarse de Aprendizaje Supervisado,
también incluimos la etiqueta que le corresponde a cada individuo: 1 para tratados y
controles, y 0 para los NiNi.

Con estas categorías en mente, es muy importante destacar una diferencia fundamental
respecto a lo que ocurre en situaciones reales. En un entorno real, a la hora de llevar a
cabo la evaluación de impacto, solo se puede identificar quiénes han sido tratados y
quiénes no. Dentro del conjunto de no tratados, puede haber tanto individuos de control
potenciales como individuos NiNi, pero no se cuenta con información para distinguirlos
directamente.

La idea detrás de esta construcción artificial es que tanto tratados como controles
posean una determinada dinámica en su comportamiento temporal, mientras que los NiNis
no. Nuevamente, cabe aclarar que estas son hipótesis que en la vida real pueden resultar
difíciles de verificar, por lo que queremos ver si modelándolas de cierta forma, los
modelos son capaces de capturarlas.

Generamos diferentes escenarios, caracterizados principalemente por dos aspectos:
\begin{enumerate}[itemsep=0.01cm, label=\textbf{\arabic*.}]
    \item La longitud de las series de tiempo generadas. Esto representa la cantidad
    de períodos observados para cada individuo previos al inicio del tratamiento.
    \item La dinámica temporal específica observada en las series de tiempo.
\end{enumerate}
Y para cada escenario, generamos 100 simulaciones para poder garantizar que los resultados
obtenidos sean estadísticamente significativos.

A continuación, detallamos algunas configuraciones fijas a lo largo de todos los
escenarios y simulaciones:
\begin{itemize}[noitemsep]
    \item Cantidad total de individuos: 10000.
    \item Cantidad de cohortes: 3.
    \item Cantidad de individuos tratados: 1000, divididos en las 3 cohortes.
    \item Cantidad de individuos de control: 1000.
    \item Cantidad de individuos NiNi: 8000.
\end{itemize}

Los parámetros para la generación de los datos son los siguientes:
\begin{itemize}[itemsep=0.1cm]
    \item \texttt{n\_sample}: cantidad de individuos en la simulación.
    \item \texttt{treated\_pct}: porcentaje de individuos tratados.
    \item \texttt{control\_pct}: porcentaje de individuos de control.
    \item \texttt{T}: cantidad de períodos observados de cada individuo.
    \item \texttt{first\_tr\_period}: primer período de tratamiento (o único, dependiendo
    de la cantidad de cohortes).
    \item \texttt{n\_cohorts}: cantidad de cohortes.
    \item \texttt{phiT}: persistencia auto-regresiva para los tratados.
    \item \texttt{phiC}: persistencia auto-regresiva para los controles.
    \item \texttt{n\_dep\_periods}: cantidad de períodos de dependencia para la
    participación en el tratamiento.
\end{itemize}

Concretamente, la fórmula con la que se generaron los valores de las series de tiempo
fue la siguiente: \textbf{TODO}

El resultado de la simulación es un conjunto de datos de panel compuesto por series de
tiempo univariadas de longitud \texttt{T} para los distintos tipos de individuos:


En el dataset, cada individuo tiene un identificador y su tipo; y aquellos de tipo 1 y 2
tienen como feature extra el período (desde \texttt{first\_tr\_period} hasta
\texttt{first\_tr\_period + n\_cohorts}) en el que ingresaron al programa. A continuación,
se muestran algunos ejemplos, tomando \texttt{T = 6}, \texttt{first\_tr\_period = 3} y
\texttt{n\_cohorts = 2}:

En este punto cabe recordar cuál es nuestra meta: a partir de información sobre los
individuos que fueron tratados, queremos identificar de entre los no tratados, quiénes son
los que podrían formar parte del grupo de control.

También resulta muy importante tener en cuenta que en los datos generados sintéticamente,
sabemos quiénes son los controles, pero en la realidad esto es justamente lo que queremos
identificar.

\subsection{Conjuntos de Entrenamiento y de Test}
Ahora bien, para entrenar y evaluar a nuestros modelos, tomamos en cuenta lo que ocurriría
en un escenario real, en el que sabríamos solamente quiénes son los tratados y quiénes los
no tratados. Por lo tanto, construimos el conjunto de entrenamiento con la totalidad de
los individuos de tipo 1 con etiqueta 1 y algunos de tipo 3 con etiqueta 0, y en el
conjunto de test colocamos a todos los de tipo 2, queriendo predecir en ellos un 1, y al
resto de tipo 3, queriendo predecir en ellos un 0.

Más aún, como queremos identificar a los grupos de control de cada cohorte,

\end{document}