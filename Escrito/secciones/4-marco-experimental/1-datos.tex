\documentclass[../../main.tex]{subfiles}

\begin{document}
Como mencionamos en la Sección \ref{sec:problema}, los experimentos fueron realizados con
datos generados sintéticamente, diseñados para imitar situaciones reales. Para comprender
cómo se construyeron estos datos, recordemos que nuestro objetivo es identificar
unidades pertenecientes al grupo de control en escenarios donde la asignación de
individuos al tratamiento tiene las siguientes características:
\begin{itemize}[itemsep=0.05cm]
    \item Es no aleatoria.
    \item Es secuencial, es decir que hay cohortes.
    \item Es potencialmente dependendiente de resultados pasados.
\end{itemize}

Dicho esto, la síntesis de los datos involucra la creación de una serie de tiempo
univariada para cada individuo o unidad, que incorpora efectos fijos\footnotemark, efectos
temporales\footnotemark\ y componentes autorregresivos\footnotemark, resultando así en un
conjunto de datos de panel. Distinguimos entre tres tipos de individuos:
\footnotetext{Un \textbf{proceso autorregresivo de orden uno}, abreviado como
\textbf{AR(1)} es un modelo de series de tiempo en donde el valor actual de la serie
depende linealmente de su valor más reciente más una perturbación impredecible
\cite{intro-econometria-wooldridge}. La fórmula de un AR(1) es la siguiente:
\[
    y_t = \rho * y_{t-1} + e_t
\]
donde a \(\rho\) se lo denomina coeficiente autorregresivo y \(e_t\) para \(t=0,1,...,T\)
es una secuencia de valores independiente e idénticamente distribuidos con media cero y
varianza \(\sigma_e^2\). Para más información, se puede consultar el Capítulo 11.1 de
\cite{intro-econometria-wooldridge}.}
\footnotetext{FALTA EFECTOS TEMPORALES}
\footnotetext{En el modelado de series de tiempo, una forma de capturar los efectos no
observables que influyen en la variable dependiente y son constantes en el tiempo consiste
en incorporar un factor denominado \textbf{efecto fijo}, el cual permanece invariante a lo
largo de toda la serie de tiempo. Parauna explicación más detallada, se puede consultar el
Capítulo 13.3 de \cite{intro-econometria-wooldridge}.}
\begin{itemize}[itemsep=0.1cm]
    \item \textbf{Individuos tratados}: son aquellos que han recibido el tratamiento en
    algún período, es decir que forman parte de alguna de las cohortes.
    \item \textbf{Individuos de control}: son aquellos que en los datos sintéticos,
    sabemos que forman parte del grupo de control para una determinada cohorte de
    tratados. Los llamaremos ``controles''.
    \item \textbf{Individuos ``NiNi''} (``Ni tratados Ni controles''): son aquellas
    unidades que no han sido tratadas y que en los datos sintéticos, sabemos que no forman
    parte del grupo de control para ninguna de las cohortes.
\end{itemize}

Para los tratados y los controles, agregamos una característica extra además de la serie
de tiempo, a la que llamamos ``\textit{inicio de programa}''. Para los primeros, este
representa el momento en que un individuo recibió el tratamiento; y para cada unidad de
control, representa el momento en que se aplicó el programa a los individuos tratados para
los cuales efectivamente sirve como control.

Además, al estar en el marco del Aprendizaje Supervisado, también incluimos la etiqueta
que le corresponde a cada individuo: 1 para tratados y controles, y 0 para los NiNi. La
razón por la cual tratados y controles comparten la misma etiqueta es que buscamos que el
modelo aprenda durante su entrenamiento los patrones que tienen los tratados para luego
poder clasificar como ``1'' a aquellos que se les asemejen. Estos serán justamente los que
identifique como potenciales individuos de control.

La motivación detrás de esta construcción artificial es forzar que tanto tratados como
controles compartan una determinada dinámica en su comportamiento temporal, mientras que
los NiNis no. Nuevamente, cabe aclarar que estas son hipótesis que en la práctica pueden
resultar difíciles de verificar. Sin embargo, nuestro interés radica en evaluar si, al
modelar estos supuestos, los algoritmos son capaces de capturar tales patrones.

Como mencionamos anteriormente, para cada individuo \(i\) generamos una serie de tiempo
univariada \(y_i\) de una longitud \(L\). La fórmula que utilizamos para simular cada
valor \(y_{i,t}\) de las series, con \(t = 0, 1, ..., L\) indicando el período generado,
fue la siguiente:
\begin{align}
    y_{i,0} &= \mu + u_0  \ \sigma \\
    y_{i,t} &= \phi \ y_{i,t-1} + (1 - \phi) \ \mu +  u_t \ \sigma \qquad (t \ge 1)
\end{align}
donde:
\[
    \mu = \mu_{EF} + EF_i + \mu_{ET}
\]
es una constante que depende del individuo \(i\) y:
\begin{itemize}[itemsep=0.1cm]
    \item \(\mu_{EF}\) representa la media de los efectos fijos para todos los individuos
    de un mismo grupo. Denotaremos con \(\mu_{EF_T}\), \(\mu_{EF_C}\) y
    \(\mu_{EF_{NiNi}}\) a la media de los efectos fijos de los tratados, controles y NiNi
    respectivamente.
    \item \(\mu_{ET}\) representa la media de los efectos temporales en todos los pasos
    de la serie y es la misma para todos los grupos.
    \item \(EF_i\) representa el efecto fijo del individuo \(i\), constante en todos
    los pasos de tiempo.
    \item \(\phi\) es el componente autorregresivo asociado a la variable generada para
    todos los individuos de un mismo grupo. En nuestros experimentos, usamos \(\phi = 0.9\)
    para todos los grupos.
    \item \(u_t\) es un valor aleatorio proveniente de una distribución normal estándar
    (se genera uno en cada paso de tiempo).
    \item \(\sigma\) es la desviación estándar del término de ruido de las series,
    y es la misma para todos los grupos. En este trabajo, tomamos \(\sigma=5\).
\end{itemize}

Generamos cinco escenarios, que se diferencian entre sí por determinados aspectos:
\begin{enumerate}[itemsep=0.05cm, label=\textbf{\arabic*.}]
    \item La longitud de las series de tiempo generadas, que denotamos con \(L\). Esta
    representa la cantidad de períodos observados para cada individuo \textbf{previos al inicio
    del tratamiento}.
    \item La cantidad de períodos previos al inicio del tratamiento en los que se observa
    un comportamiento específico en tratados y controles. A estos períodos los llamamos
    ``períodos de dependencia'' y a la cantidad la denotamos con \(n_{pd}\). Se
    va a cumplir siempre que \(0 \le n_{pd} < L\). En particular, en los
    escenarios con períodos de dependencia, al comportamiento que modelamos lo llamamos
    ``decreciente con ruido'': en dichos períodos, forzamos a que cada nuevo valor
    generado sea estrictamente menor al anterior, salvo en una cantidad \(m\) de ocasiones
    en donde le permitimos que sea mayor. Este \(m\) también varía entre escenarios.
    \item La cantidad de cohortes, que denotamos con \(c\).
    \item El valor de la media de los efectos fijos de cada grupo (\(\mu_{EF_T}\),
    \(\mu_{EF_C}\) y \(\mu_{EF_{NiNi}}\)). Modificar este provoca que el promedio de los
    valores generados en cada paso de tiempo para un grupo esté por encima o por debajo de
    otro.
\end{enumerate}
Para cada escenario, generamos 100 simulaciones para poder garantizar que los resultados
obtenidos sean estadísticamente significativos.

A modo de ejemplo, la Figura \ref{fig:treated_series_example} muestra el gráfico de 15
series de tiempo generadas para individuos tratados con \(L=45\), \(n_{pd}=20\), \(m=6\),
\(c=3\) y \(\mu_{EF_T} = 10\). En este caso, los inicios de programa son 46, 47 y 48 (uno
para cada cohorte).

\begin{figure}[h!]
    \centering
    \includegraphics[width=0.8\textwidth]{figs/series_tratados_exp1.png}
    \caption{Gráfico de 15 series de tiempo generadas para individuos tratados con
    \(L=45\), \(n_{pd}=20\), \(m=6\), \(c=3\) y \(\mu_{EF_T} = 10\). En el eje \(x\) se
    encuentra el período, relativo al inicio de programa de cada individuo, y en el eje
    \(y\) el valor de la variable en cada período. La línea vertical punteada de color
    rojo indica el inicio de programa de cada individuo, y los períodos con fondo rojo son
    los períodos de dependencia.}
    \label{fig:treated_series_example}
\end{figure}

En todos los escenarios y simulaciones, la distribución de individuos fue la siguiente:
\begin{itemize}[noitemsep]
    \item Cantidad de individuos tratados: 1000, divididos en la cantidad de cohortes
    correspondiente.
    \item Cantidad de individuos de control: 1000, divididos en la cantidad de cohortes
    correspondiente.
    \item Cantidad de individuos NiNi: 2500.
\end{itemize}

A continuación, describimos cómo a partir de los datos generados, construimos los conjuntos
de entrenamiento y test.

\subsection{Conjuntos de entrenamiento y de test}
Con la idea que el modelo aprenda los patrones vistos en los tratados para luego poder
identificar los controles, construimos los conjuntos de entrenamiento y de test de la
siguiente manera:
\begin{itemize}
    \item \textbf{Conjunto de entrenamiento}: individuos tratados de todas las cohortes
    (1000 en total) y una muestra aleatoria de 1000 NiNis.
    \item \textbf{Conjunto de test}: individuos de control de todas las cohortes (1000 en
    total) y una muestra aleatoria de 1500 NiNis, distinta a la utilizada en el conjunto
    de entrenamiento.
\end{itemize}

Ahora bien, recordemos que tanto tratados como controles tienen una característica extra:
el inicio de programa. Sin embargo, esta no está presente en los NiNis, justamente porque
no fueron tratados y porque no forman parte del grupo de comparación en ninguna de las
cohortes. Para reflejar esto e intentar ``transmitírselo'' al modelo, lo que hicimos fue,
por cada NiNi incluido, repetirlo tantas veces como inicios de programa (o cohortes) haya,
utilizando en cada repetición los períodos previos al respectivo inicio. En todas las
repeticiones, la etiqueta asignada es 0.

Para los controles, incluidos en el conjunto de test, hicimos exactamente lo mismo pero
asignando en todas las repeticiones la etiqueta 1. Al hacer esto, estamos relajando uno de
nuestros objetivos, haciendo que un individuo de control lo pueda ser para cualquier
cohorte.

De esta forma, el balanceo de clases en cada conjunto es la siguiente:
\begin{itemize}[itemsep=0.1cm]
    \item \textbf{Conjunto de entrenamiento: } 75\% etiqueta 0; 25\% etiqueta 1.
    \item \textbf{Conjunto de test: } 71.43\% etiqueta 0; 28.57\% etiqueta 1.
\end{itemize}

\end{document}