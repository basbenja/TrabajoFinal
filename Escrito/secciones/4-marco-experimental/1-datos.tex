\documentclass[../../main.tex]{subfiles}

\begin{document}
Como mencionamos en el \hyperref[chap:problema]{Capítulo 3}, los experimentos fueron
realizados con datos generados sintéticamente, diseñados para imitar situaciones reales.
Para comprender cómo se construyeron estos datos, recordemos que nuestro objetivo es
identificar unidades pertenecientes al grupo de control en escenarios donde asumimos que
la asignación de individuos (o su inscripción) al tratamiento presenta las siguientes
características:
\begin{itemize}[itemsep=0.05cm]
    \item Es no aleatoria.
    \item Es secuencial, es decir que hay cohortes.
    \item Es dependiente de los comportamientos temporales observados en la variable
    objetivo del programa durante los períodos previos a su inicio.
\end{itemize}

Dicho esto, la síntesis de los datos involucró la creación de una serie de tiempo
univariada para cada individuo o unidad, que incorpora componentes
autorregresivos\footnote{Un \textbf{proceso autorregresivo de orden uno}, abreviado como
\textbf{AR(1)} es un modelo de series de tiempo en donde el valor actual de la serie
depende linealmente de su valor más reciente más una perturbación impredecible
\cite{intro-econometria-wooldridge}. La fórmula de un AR(1) es la siguiente:
\[
    y_t = \rho \cdot y_{t-1} + e_t
\]
donde a \(\rho\) se lo denomina coeficiente autorregresivo y \(e_t\) para \(t=0,1,...,T\)
es una secuencia de valores independiente e idénticamente distribuidos con media cero y
varianza \(\sigma_e^2\). Para más información, se puede consultar el Capítulo 11.1 de
\cite{intro-econometria-wooldridge}.}, efectos fijos y efectos temporales\footnote{En el
modelado de series de tiempo, una forma de capturar los efectos no observables que
influyen en la variable dependiente y son constantes en el tiempo consiste en incorporar
un factor denominado \textbf{efecto fijo}, el cual permanece invariante a lo largo de
todos los pasos de tiempo. Para una explicación más detallada, se puede consultar el
Capítulo 13.3 de \cite{intro-econometria-wooldridge}. Similarmente, los efectos no
observables que sí cambian con el tiempo pero que afectan a todos los individuos por igual
se conocen como \textbf{efectos temporales}.}, resultando así en un conjunto de datos
de panel. Distinguimos entre tres tipos de individuos:

\begin{itemize}[itemsep=0.1cm]
    \item \textbf{Individuos tratados}: son aquellos que han recibido el tratamiento en
    algún período, es decir que forman parte de alguna de las cohortes.
    \item \textbf{Individuos de control}: son aquellos que, en los datos sintéticos,
    sabemos que pertenecen al grupo de control para una determinada cohorte de tratados.
    Los llamaremos ``controles'' y son justamente aquellos que los modelos deberán
    identificar.
    \item \textbf{Individuos ``NiNi''} (``ni tratados ni controles''): son aquellas
    unidades que no han sido tratadas y que, en los datos sintéticos, sabemos que no forman
    parte del grupo de control de ninguna cohorte.
\end{itemize}

Para los tratados y los controles, agregamos una característica extra además de la serie
de tiempo, a la que llamamos ``\textit{inicio de programa}''\footnote{La forma en que
incluimos la característica de ``inicio de programa'' en los NiNi se detallará en la
\hyperref[sec:comparacion]{Sección 4.4}.}. En el caso de los tratados, esta variable
indica el momento en que un individuo recibió el tratamiento (relativo a la cantidad de
períodos observados de los individuos). Por ejemplo, si se consideran 20 períodos de
tiempo previos al tratamiento y existen 3 cohortes, entonces los inicios de programa serán
21, 22 y 23. Por otro lado, para cada unidad de control, el inicio de programa representa
el momento en que se aplicó la intervención a los individuos tratados para los cuales
actúa como control.

Siguiendo lo explicado en capítulos anteriores, nuestro trabajo se enmarca dentro del
aprendizaje supervisado. Por lo tanto, también se incluye la etiqueta que le corresponde a
cada individuo: ``1'' para tratados y controles, y ``0'' para los NiNi. La razón por la
cual tratados y controles comparten la misma etiqueta es que buscamos que los modelos
identifiquen, durante su entrenamiento, los patrones presentes en los individuos tratados
para luego poder clasificar como 1 a aquellos que se les asemejen. Estos serán justamente
los que el modelo identifique como potenciales individuos de control.

La motivación detrás de esta construcción artificial es forzar que tanto tratados como
controles compartan una determinada dinámica en su comportamiento temporal, mientras que
los NiNi no. Cabe aclarar que la presencia de la dinámica exacta que modelamos constituye
un supuesto que en la práctica puede resultar difícil de verificar. Nuestro interés radica
en evaluar si al simularla, los algoritmos son capaces de capturar tales patrones.

Como mencionamos anteriormente, para cada individuo \(i\) generamos una serie de tiempo
univariada \(y_i\) de una longitud \(L\). La fórmula que utilizamos para simular cada
valor \(y_{i,t}\) de las series, con \(t = 0, 1, ..., L\) indicando el período generado,
fue la siguiente:
\begin{align}
    y_{i,0} &= \mu + u_{i,0} \cdot \sigma \\
    y_{i,t} &= \phi \cdot y_{i,t-1} + (1 - \phi) \cdot \mu +  u_{i,t} \cdot \sigma \qquad (t \ge 1)
\end{align}
donde:
\[
    \mu = \mu_{EF} + EF_i + \mu_{ET}
\]
es una constante que depende del individuo \(i\) y:
\begin{itemize}[itemsep=0.1cm]
    \item \(\mu_{EF}\) representa la media de los efectos fijos para todos los individuos
    de un mismo grupo. Denotamos con \(\mu_{{EF}_T}\), \(\mu_{{EF}_C}\) y
    \(\mu_{{EF}_{NiNi}}\) la media de los efectos fijos para tratados, controles y NiNi
    respectivamente. En todos nuestros experiementos, tomamos \(\mu_{{EF}_T} =
    \mu_{{EF}_C} = 10\), y el que sí variamos es \(\mu_{{EF}_{NiNi}}\).
    \item \(\mu_{ET}\) representa la media de los efectos temporales en todos los pasos
    de la serie y es la misma para todos los grupos.
    \item \(EF_i\) representa el efecto fijo del individuo \(i\), constante en todos
    los pasos de tiempo.
    \item \(\phi\) es el componente autorregresivo asociado a la variable generada para
    todos los individuos de un mismo grupo. Representa el grado de dependencia temporal
    entre el valor actual de la serie y el anterior. En nuestros experimentos, usamos
    \(\phi = 0.9\) para todos los grupos.
    \item \(u_{i,t}\) es un valor aleatorio proveniente de una distribución normal
    estándar (se genera uno en cada paso de tiempo y para cada individuo).
    \item \(\sigma\) es la desviación estándar del término de ruido de las series,
    y es la misma para todos los grupos. En este trabajo, tomamos \(\sigma=5\).
\end{itemize}

Generamos diversos escenarios, todos con 3 cohortes, que se diferencian entre sí por
ciertos aspectos:
\begin{enumerate}[itemsep=0.05cm, label=\textbf{\arabic*.}]
    \item La longitud de las series de tiempo generadas, que denotamos con \(L\). Esta
    representa la cantidad de períodos observados para cada individuo \textbf{previos al
    inicio del programa}.
    \item La cantidad de períodos previos al inicio del programa en los que se observa un
    comportamiento específico en tratados y controles. A estos períodos, como mencionamos
    previamente, los llamamos ``\textbf{períodos de dependencia} (temporal)'', y
    corresponden siempre a los últimos períodos pre-tratamiento. Denotamos la cantidad de
    períodos de dependencia con \(n_{pd}\), y en nuestros escenarios, se va a cumplir
    siempre que \(0 \le n_{pd} < L\).
    \item En los períodos de dependencia, modelamos un comportamiento al que llamamos
    ``\textbf{decreciente con ruido}'': en dichos períodos, forzamos a que cada nuevo
    valor generado de la serie sea estrictamente menor al anterior, salvo en — a lo sumo —
    una cantidad \(m\) de ocasiones en donde permitimos que haya subidas o incrementos.
    Este \(m\) también varía entre escenarios. Consideramos que esta tendencia es
    representativa de fenómenos reales, y que puede influir en la decisión que los
    individuos reciban un tratamiento o participen de un programa.
    \item La media de los efectos temporales para los indiviudos NiNi
    (\(\mu_{{EF}_{NiNi}}\)). Puntualmente, reducir este valor a la hora de generar los
    datos provoca que la media de la variable dependiente en los NiNi en cada uno de los
    períodos esté por debajo de la de tratados y controles\footnote{Este efecto se verá
    con mayor claridad en el \hyperref[sec:exp4]{experimento 4} del
    \hyperref[chap:resultados]{Capítulo 5}.}.
\end{enumerate}
Una vez fijados los parámetros de un escenario, generamos 100 simulaciones para garantizar
que los resultados obtenidos sean estadísticamente significativos.

A modo de ejemplo, la Figura \ref{fig:treated_series_example} muestra el gráfico de 15
series de tiempo generadas para individuos tratados con \(L=45\), \(n_{pd}=20\) y \(m=6\).
Como tenemos 3 cohortes y de cada indivudo estamos tomando los 45 períodos antes que cada
uno haya recibido el tratamiento (independientemente de la cohorte a la que pertenezcan),
los inicios de programa, aunque no se visualizan explícitamente, son 46, 47 y 48 (uno para
cada cohorte).

\begin{figure}[ht]
    \centering
    \includegraphics[width=0.8\textwidth]{figs/tratados_exp1_sim13.png}
    \caption{Gráfico de 15 series de tiempo generadas para individuos tratados con
    \(L=45\), \(n_{pd}=20\), y \(m=6\). Los tratados están divididos en cohortes, cada una
    de las cuales tiene su correspondiente inicio de programa, por lo que en el eje \(x\)
    se encuentra el período relativo al inicio de programa de cada individuo (por ejemplo,
    -5 indica el quinto período antes que una unidad reciba el tratamiento,
    independientemente de cuándo lo recibió). El eje \(y\) representa el valor de la
    variable en cada período. La línea vertical punteada de color rojo indica el inicio de
    tratamiento de cada individuo, y los períodos con fondo rojo son los períodos de
    dependencia.}
    \label{fig:treated_series_example}
\end{figure}

En todos los escenarios y simulaciones, la distribución de individuos fue la siguiente:
\begin{itemize}[noitemsep]
    \item Cantidad de individuos tratados: 1000, divididos en las 3 cohortes (334 en la
    primera y la segunda, y 332 en la restante).
    \item Cantidad de individuos de control: 1000, divididos en las 3 cohortes (334 en la
    primera y la segunda, y 332 en la restante).
    \item Cantidad de individuos NiNi: 3500.
\end{itemize}

En la \hyperref[sec:comparacion]{Sección 4.4}, explicamos cómo a partir de los datos
generados, construimos los conjuntos de entrenamiento y test, y cómo fueron utilizados
para obtener los resultados.

A continuación, describimos las diferentes arquitecturas de redes neuronales que
utilizamos durante los experimentos y los valores particulares que fueron incluidos en la
búsqueda de hiperparámetros.

\end{document}