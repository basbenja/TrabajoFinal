\documentclass[../../main.tex]{subfiles}
% \graphicspath{{\subfix{../images/}}}

\begin{document}
% Trabajamos bajo el supuesto que el comportamiento dinámico de la variable modelada para
% cada unidad es el que determina si esta ingresa al tratamiento o no.

El primer paso para poder llevar a cabo los experimentos es contar con datos. Nuestro
entorno de simulación está diseñado para generar \textbf{datos sintéticos} que imiten
escenarios reales en donde la asignación al tratamiento es no aleatoria, secuencial y
potencialmente dependiente de resultados pasados.

El proceso de generación de datos involucra la creación de una serie de tiempo univariada
para cada individuo, que incorpora efectos fijos y componentes autoregresivos. Distinguimos
entre tres tipos de individuos:
\begin{itemize}[itemsep=0.1cm]
    \item \textbf{Individuos tratados}: son aquellos que han recibido el tratamiento en
    algún período. Es importante notar que en una situación real, son datos con los que
    contamos.
    \item \textbf{Individuos de control}: son aquellos que en los datos sintéticos, sabemos
    que forman parte del grupo de control. Es importante notar que en un escenario real,
    no sabemos quiénes son estos individuos sino que son los que tratamos de identificar.
    \item \textbf{Individuos ``NiNi''}: son aquellas unidades que no han sido tratadas y
    que en los datos sintéticos, sabemos que no forman parte del grupo de control.
\end{itemize}

Al tratarse de Aprendizaje Supervisado, también incluimos la etiqueta que le corresponde a
cada individuo: 1 para tratados y controles, y 0 para los NiNi.

Los parámetros para la generación de los datos son los siguientes:
\begin{itemize}[itemsep=0.1cm]
    \item \texttt{n\_sample}: cantidad de individuos en la simulación.
    \item \texttt{treated\_pct}: porcentaje de individuos tratados.
    \item \texttt{control\_pct}: porcentaje de individuos de control.
    \item \texttt{T}: cantidad de períodos observados de cada individuo.
    \item \texttt{first\_tr\_period}: primer período de tratamiento (o único, dependiendo
    de la cantidad de cohortes).
    \item \texttt{n\_cohorts}: cantidad de cohortes.
    \item \texttt{phiT}: persistencia auto-regresiva para los tratados.
    \item \texttt{phiC}: persistencia auto-regresiva para los controles.
    \item \texttt{n\_dep\_periods}: cantidad de períodos de dependencia para la
    participación en el tratamiento.
\end{itemize}

Concretamente, la fórmula con la que se generaron los valores de las series de tiempo
fue la siguiente: \textbf{TODO}

El resultado de la simulación es un conjunto de datos de panel compuesto por series de
tiempo univariadas de longitud \texttt{T} para los distintos tipos de individuos:


En el dataset, cada individuo tiene un identificador y su tipo; y aquellos de tipo 1 y 2
tienen como feature extra el período (desde \texttt{first\_tr\_period} hasta
\texttt{first\_tr\_period + n\_cohorts}) en el que ingresaron al programa. A continuación,
se muestran algunos ejemplos, tomando \texttt{T = 6}, \texttt{first\_tr\_period = 3} y
\texttt{n\_cohorts = 2}:

En este punto cabe recordar cuál es nuestra meta: a partir de información sobre los
individuos que fueron tratados, queremos identificar de entre los no tratados, quiénes son
los que podrían formar parte del grupo de control.

También resulta muy importante tener en cuenta que en los datos generados sintéticamente,
sabemos quiénes son los controles, pero en la realidad esto es justamente lo que queremos
identificar.

\subsection{Conjuntos de Entrenamiento y de Test}
Ahora bien, para entrenar y evaluar a nuestros modelos, tomamos en cuenta lo que ocurriría
en un escenario real, en el que sabríamos solamente quiénes son los tratados y quiénes los
no tratados. Por lo tanto, construimos el conjunto de entrenamiento con la totalidad de
los individuos de tipo 1 con etiqueta 1 y algunos de tipo 3 con etiqueta 0, y en el
conjunto de test colocamos a todos los de tipo 2, queriendo predecir en ellos un 1, y al
resto de tipo 3, queriendo predecir en ellos un 0.

Más aún, como queremos identificar a los grupos de control de cada cohorte,

\end{document}