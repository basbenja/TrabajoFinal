\documentclass[../../main.tex]{subfiles}
% \graphicspath{{\subfix{../images/}}}

\begin{document}
Las Redes Neuronales Recurrentes (RNR) son una familia de redes neuronales pensadas
específicamente para trabajar con datos secuenciales. Es por esto que han sido y continúan
siendo muy útiles en tareas como análisis de series temporales y procesamiento de
lenguaje natural.

A diferencia de las redes feedforward que venimos presentando hasta el momento, lo que
caracteriza a estas es que permiten la presencia de cíclos en su grafo, es decir se ven
conexiones que apuntan ``hacia atrás''. Para ver esto, tomemos la más simple de las RNR,
compuesta por una neurona que recibe la entrada correspondiente al tiempo \(t\), produce
una salida y se la envía a sí misma \cite{hands-on-ML-sklearn-tf}, como se puede ver
a la izquierda del siguiente diagrama. De esta forma, en cada paso \(t\), la \textit{neurona
recurrente} recibe no solo la entrada \(x_t\) sino también su propia salida computada
en el paso anterior, \(y_{t-1}\).

\begin{center}
    \begin{tikzpicture}[
        item/.style={circle,draw,thick,align=center},
        itemc/.style={item,on chain,join}
    ]
        % Unenrolled RNN (on the right)
        \begin{scope}[
            start chain=going right,nodes=itemc,every
            join/.style={-latex,very thick},local bounding box=chain
        ]
            \path node (f0) {\(f\)} node (f1) {\(f\)} node (f2) {\(f\)} node[xshift=2em] (ft)
            {\(f\)};
        \end{scope}
        \foreach \X in {0,1,2,t}{
            \draw[very thick,-latex] (f\X.north) -- ++ (0,2em)
            node[above,item,fill=gray!10] (y\X) {\(y_\X\)};
            \draw[very thick,latex-] (f\X.south) -- ++ (0,-2em)
            node[below,item,fill=gray!10] (x\X) {\(x_\X\)};
        }
        \path (x2) -- (xt) node[midway,scale=2,font=\bfseries] {\dots};

        % Equal sign
        \node[left=1em of chain,scale=2] (eq) {\(=\)};

        % Folded RNN (on the left)
        \node[left=2em of eq,item] (f) {\(f\)};
        \path (f.west) ++ (-1em,2em) coordinate (aux);
        \draw[very thick,-latex,rounded corners] (f.east) -| ++ (1em,2em) -- (aux)
        |- (f.west);
        \draw[white,line width=0.8ex] (f.north) -- ++ (0,1.9em);
        \draw[very thick,-latex] (f.north) -- ++ (0,2em)
        node[above,item,fill=gray!10] {\(y_t\)};
        \draw[very thick,latex-] (f.south) -- ++ (0,-2em)
        node[below,item,fill=gray!10] {\(x_t\)};
    \end{tikzpicture}
\end{center}

\subsection{Long Short-Term Memory}

\subsection{Gated Recurrent Unit}

\end{document}