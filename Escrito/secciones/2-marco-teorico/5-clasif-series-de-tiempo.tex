\documentclass[../../main.tex]{subfiles}
% \graphicspath{{\subfix{../images/}}}

\begin{document}
Una \textbf{serie de tiempo} es una secuencia \textbf{ordenada} de valores medidos
sucesivamente en intervalos de tiempo igualmente espaciados
\cite{khan2021-bilstm-for-unitsc}. Este tipo de datos está presente en distintos
fenómenos, como lecturas meteorológicas, registros financieros, señales fisiológicas, y
observaciones industriales \cite{wang2016timeseriesclassificationscratch}. 

% Encerrarlo en un tag de definición ?
Formalmente, una serie de tiempo \textbf{univariada} \(U\) es un conjunto ordenado de
valores reales \(U = (x_1, x_2, ..., x_N\), y su dimensión está dada por la cantidad \(N\)
de valores \cite{khan2021-bilstm-for-unitsc}.

Dicho esto, la \textbf{Clasificación de Series de Tiempo} (CST) consiste en: dado un
conjunto de clases o categorías \(Y\), y un conjunto de datos \(D\) formado por series de
tiempo univariadas \(U_i\) donde a cada una le corresponde una etiqueta \(y(U_i) \in Y\),
el objetivo es encontrar una función \(f\), a la que llamaremos ``clasificador'' o
``modelo'' de forma tal que \(f(U) = y(U)\) \cite{khan2021-bilstm-for-unitsc}. Un buen
modelo será uno que pueda capturar y generalizar el el patrón de las series de datos de
forma tal que sea capaz de clasificar correctamente nuevos datos. El repositorio que se
suele tomar como referencia para evaluar los diferentes modelos de CST es el de la
Universidad de California Riverside, llamado \textit{UCR Time Series Classification
Archive} \cite{UCRArchive2018}.

Durante las últimas décadas, se han propuestos diferentes métodos para afrontar este
problema. Los primeros trabajaban directamente sobre las series de tiempo utilizando
alguna medida de similitud predefinida que pueda capturar la similitud entre ellas, como
por ejemplo la distancia euclideana o la deformación dinámica del tiempo (\textit{Dynamic
Time Warping}) \cite{wang2016timeseriesclassificationscratch}.

Se han desarrollado también algoritmos basados en características o \textit{features}, en
donde cada serie temporal es convertida en un conjunto de features globales, que es
posteriormente usando para definir la semejanza entre pares de series de tiempo
\cite{khan2021-bilstm-for-unitsc}. Algunos de los más conocidos son ...

El gran problema de todas las estrategias mencionadas es que requieren un gran
preprocesamiento de los datos y una ingeniería de atributos (\textit{feature engineering})
\cite{wang2016timeseriesclassificationscratch}, para lo cual generalmente se necesita
tener un buen conocimiento de campo sobre las series sobre las que se está trabajando. Es
por esto que recientemente, las redes neuronales profundas han comenzado a ser utilizadas
para clasificar series temporales \cite{wang2016timeseriesclassificationscratch} ya que
permiten trabajar de forma directa con los datos, y se encargan de detectar
automáticamente cuáles son las características que distinguen a las series. Algunas
arquitecturas utilizadas hasta el momento son el perceptrón multicapa
\cite{wang2016timeseriesclassificationscratch}, redes neuronales convolucionales
\cite{wang2016timeseriesclassificationscratch}, y estas últimas combinadas con unidades
LSTM unidireccionales \cite{Karim_2018} y bidireccionales
\cite{khan2021-bilstm-for-unitsc}.

\end{document}