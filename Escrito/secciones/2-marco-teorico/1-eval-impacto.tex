\documentclass[../../main.tex]{subfiles}
% \graphicspath{{\subfix{../images/}}}

\begin{document}
\subsection{¿Qué es una Evaluación de Impacto?}
\begin{comment}
El uso de métodos cuantitativos para medir el impacto de programas sociales ha cobrado un gran interés en los últimos años \cite{bernal}. Las evaluaciones de impacto han comenzado a desempeñar un papel preponderante en el diseño de políticas públicas \cite{bernal}.
\end{comment}

Los programas y políticas de desarrollo suelen estar diseñados para cambiar resultados, como aumentar los ingresos, mejorar el aprendizaje o reducir las enfermedades \cite{gertler-2016}. Saber si estos cambios se logran o no es una pregunta crucial para las políticas públicas \cite{gertler-2016}, y aquí es donde entra en juego lo que se conoce como \textbf{evaluación de impacto}.

Una evaluación de impacto mide los cambios en el bienestar de los individuos que se pueden atribuir a un proyecto, un programa o una política específicos \cite{gertler-2016}. Su sello distintivo radica en que pueden proporcionar \textbf{evidencia} robusta y creíble sobre si un programa concreto ha alcanzado o está alcanzando sus resultados deseados \cite{gertler-2016}. 

Este tipo de evaluaciones pone un fuerte énfasis en los resultados y busca responder una pregunta específica de causa y efecto: ¿cuál es el impacto (o efecto causal) de un programa en un resultado de interés? \cite{gertler-2016}. Es decir, se focalizan en los cambios directamente atribuibles al tratamiento \cite{gertler-2016} sobre un conjunto de \textbf{variables de resultado} en un conjunto de individuos \cite{bernal}. Estas variables son aquellas sobre las cuales se espera que el programa tenga un efecto en los beneficiarios \cite{bernal}.

Por lo tanto, el objetivo final de la evaluación de impacto consiste en establecer la diferencia entre la variable de resultado del individuo participante en el programa en presencia del programa, y la variable de resultado de ese individuo en ausencia del programa \cite{bernal}, lo que se conoce como \textbf{efecto del tratamiento}. Sin embargo, es evidente que la respuesta a ``¿qué habría pasado con los beneficiarios en ausencia del programa?'' se refiere a una situación que no es observable. Este resultado hipotético se denomina \textbf{contrafactual} y es lo que se debe estimar en cualquiera evaluación de impacto.

Algunas de las principales razones por las que se debería promover el uso de estas evaluaciones como herramientas de gestión provienen del hecho que permiten mejorar la rendición de cuentas, la inversión de recursos públicos o la efectividad de una política, obtener financiamiento, como así también probar modalidades de programas alternativos o innovaciones de diseño \cite{gertler-2016} y revelar la realidad de muchas políticas públicas para de esta forma contribuir a la fiscalización mediática \cite{bernal}.

Un ejemplo claro de por qué son necesarias las evaluaciones de impacto es el que describe Howard White con respecto al Programa Integrado de Nutrición en Bangladesh (PINB) \cite{white2009theory}. Este programa identificaba, mediante mediciones en campo, a los niños desnutridos y los asignaba a un tratamiento que incluía alimentación suplementaria a los menores y educación nutricional a las madres \cite{bernal}. Inicialmente, el programa fue considerado como un éxito ya que los datos de monitoreo mostraban caídas sustanciales en los niveles de desnutrición. El Banco Mundial decidió, con base en esta evidencia y previo a cualquier tipo de evaluación, aumentar los recursos destinados al programa. Sin embargo, las primeras evaluaciones de impacto, realizadas por el Grupo Independiente de Evaluación del mismo Banco Mundial y por la ONG inglesa \textit{Save the Children}, mostraron que la mejoría de los indicadores de los beneficiarios era similar o inferior a la de otros niños con características comparables que no hacían parte del programa \cite{bernal}. Estos resultados reflejaron que las percepciones de los administradores del programa y de las entidades financiadoras eran erradas, y sugirieron algunos correctivos al programa \cite{bernal}.

\begin{comment}
    AGREGAR EL EJEMPLO DE MEXICO QUE APARECE EN EL GERTLER
\end{comment}

% Preparación de una evaluación?

\subsection{Estimación del Tratamiento}
El marco teórico estándar para formalizar el problema de la evaluación de impacto se basa en el modelo de resultado \textit{potencial} o modelo de Roy-Rubin (\cite{rubin1974}) \cite{bernal}. 

Formalmente, definimos dos elementos para cada individuo \(i = 1,...,N\), donde \(N\) denota la población total:
\begin{itemize}
    \item Por un lado, el indicador de tratamiento \(D_i\), tal que \(D_i = 1\) implica que el individuo \(i\) participó del tratamiento, y \(D_i = 0\) en caso contrario.
    \item Por otro lado, las variables de resultado las definimos como \(Y_i(D_i) = Y_i|D_i\) - se lee como ``el valor de \(Y_i\) \textit{dado} \(D_i\)''. De esta forma, \(Y_i(1)\) es la variable de resultado si el individuo \(i\) es tratado, e \(Y_i(0)\) es la variable de resultado si el individuo \(i\) no es tratado.
\end{itemize}

Con esto, el \textbf{efecto del tratamiento} para un individuo \(i\) se puede escribir como:
\begin{equation}
    \tau_i = Y_i(1) - Y_i(0) = (Y_i|D_i=1) - (Y_i|D_i=0)
    \label{eq:ite} % ite = individual treatment effect
\end{equation}

Según esta fórmula, el impacto causal (\(\tau_i\)) de un programa (\(D_i\)) en una variable resultado (\(Y_i\)) para un individuo \(i\) es la diferencia entre la variable con el programa (\(Y_i(1)\)) y la misma variable sin el programa (\(Y_i(0)\)). 

De nuevo, el problema fundamental de la evaluación de impacto es que se intenta medir una variable en un mismo momento del tiempo para la misma unidad de observación pero en dos realidad diferentes \cite{gertler-2016}. Sin embargo, claramente solo se da uno de los dos resultados potenciales \(Y_i(1)\) o \(Y_i(0)\), pero no ambos. Es decir, en los datos queda solamente registrado \(Y_i(1)\) si \(D_i=1\) y \(Y_i(0)\) si \(D_i=0\); no se dispone de \(Y_i(1)\) si el individuo no fue tratado (\(D_i=0\)), ni tampoco de \(Y_i(0)\) si el individuo fue tratado (\(D_i=1\)). De esta manera, el \textbf{resultado observado} de \(Y_i\) se puede expresar como:

\begin{equation}
    Y_i = D_i Y_i(1) + (1-D_i)Y_i(0) =
    \begin{cases}
        Y_i(1) \text{ si } D_i=1 \\
        Y_i(0) \text{ si } D_i=0
    \end{cases}
    \label{eq:observed-result}
\end{equation}

En el caso de la ecuación \ref{eq:ite}, al enfocarse en los individuos tratados, el término \(Y_i(0) = Y_i|D_i=0\) representa el contrafactual, es decir \textit{cuál habría sido el resultado si la unidad no hubiera participado en el programa}. Al ser imposible observar directamente el contrafactual, es necesario \textbf{estimarlo}. La forma más directa de solucionar este problema sería hallando un ``clon perfecto'' para cada uno de los individuos participantes del programa, lo cual resulta bastante difícil. Por lo tanto, el primer paso para lograr esta estimación consiste en \textbf{desplazarse desde el nivel individual al nivel del grupo} \cite{gertler-2016}, concentrando el análisis en el impacto o efecto \textit{promedio} (y no individual).

En primera instancia, se puede estimar el \textit{impacto promedio del programa \textbf{sobre la población}} (o \(ATE\)):

\begin{equation}
    ATE = \mathbb{E}\left[Y_i(1)-Y_i(0)\right]
\end{equation}

El \(ATE\) se interpreta como el cambio promedio en la variable de resultado cuando un individuo escogido al azar pasa aleatoriamente de ser participante a ser no participante \cite{bernal}. Esta medida es relevante en el caso de la evaluación de un programa universal. Sin embargo, en la mayoría de los casos, el tratamiento solo está disponible para un subconjunto de la población. En este caso, es posible utilizar un estimador que únicamente promedio el efecto sobre la población elegible \cite{bernal}. 

De esta forma, tenemos el \textit{impacto promedio del programa \textbf{sobre los tratados}} (o \(ATT\)), que es en general el parámetro de mayor interés en una evaluación de impacto, y representa el efecto esperado del tratamiento en el subconjunto de individuos que fueron efectivamente tratados:

\begin{equation}
    ATT = \mathbb{E} \left[Y_i(1)-Y_i(0)|D_i=1\right] = \mathbb{E} \left[Y_i(1)|D_i=1\right] - \mathbb{E} \left[Y_i(0)|D_i=1\right]
    \label{eq:ATT}
\end{equation}

Es decir, el \(ATT\) es la diferencia entre la media de la variable de resultado en el grupo de los participantes y la media que hubieran obtenido los participantes si el programa no hubiera existido \cite{bernal}.

\subsection{El Grupo de Control como Estimador del Contrafactual}
En la ecuación \ref{eq:ATT}, el valor \(\mathbb{E} \left[Y_i(1)|D_i=1\right]\) es un resultado observable mientras que \(\mathbb{E} \left[Y_i(0)|D_i=1\right]\) es el promedio contrafactual que debemos aproximar. Para realizar esto, se construye lo que se conoce como \textbf{grupo de control}. Este está formado por individuos que \textbf{no participan} del programa pero es estadísticamente idéntico \cite{gertler-2016} al \textbf{grupo de tratamiento}, compuesto por aquellos que sí participan del programa.

\begin{figure}[h!]
    \centering
    \includegraphics[width=0.6\textwidth]{figs/grupo-de-control.png}
    \caption{}
    \label{fig:control-group}
\end{figure}

Por lo tanto, en la práctica, el reto de una evaluación de impacto es definir un \textit{buen} grupo de control, es decir uno que sea estadísticamente idéntico al de tratamiento, en promedio, en ausencia del programa \cite{gertler-2016}. Si los dos grupos son idénticos, con la única excepción que un grupo participa del programa y el otro no, es posible estar seguros que cualquier diferencia en los resultados tendría que deberse al programa \cite{gertler-2016}.

Para que un grupo de comparación sea válido, debe satisfacer lo siguiente:
\begin{itemize}
    \item Las características \textit{promedio} del grupo de tratamiento y del grupo de comparación deben ser idénticas en ausencia del programa. Cabe resaltar que no es necesario que las unidades individuales en el grupo de tratamiento tengan clones perfectos en el grupo de control.
    \item No debe ser afectado por el tratamiento de forma directa ni indirecta.
    \item El grupo de comparación debería reaccionar de la misma manera que el grupo de tratamiento si fuera objeto del programa.
\end{itemize}

Cuando el grupo de comparación no produce una estimación precisa del contrafactual, no se puede establecer el verdadero impacto del programa.

A continuación, se presentan dos situaciones en las que el grupo de comparación seleccionado conduce a una estimación incorrecta del contrafactual.

\subsubsection{Comparaciones antes-después}
Este tipo de comparaciones intenta establecer el impacto de un programa a partir de un seguimientos de los cambios en los resultados en los participantes del programa a lo largo del tiempo. Consideran el contrafactual como el resultado para el grupo de tratamiento antes que comience la intervención. Esta comparación supone que si el programa no hubiera existido, el resultado para los participantes del programa habría sido igual a su situación antes del programa, lo cual en la mayoría de los tratamientos este supuesto no puede sostenerse \cite{gertler-2016}.

\subsubsection{Comparaciones de inscritos y no inscritos}
Una de las formas más directas de solucionar el problema del contrafactual podría ser simplemente utilizar el promedio de la variable de resultado entre los individuos no participantes del programa pero elegibles para participar. Es decir, se estaría usando a \(\mathbb{E} \left[Y_i(0)|D_i=0\right]\) como una aproximación de \(\mathbb{E} \left[Y_i(0)|D_i=1\right]\) \cite{bernal}. Sin embargo, esto podría generar estimaciones inexactas del efecto del programa, dado que los participantes y los no participantes generalmente son diferentes (en características observadas y no observadas), aún en ausencia del programa, y por tal motivo es que unos \textit{eligen} participar y otros no \cite{bernal}. Este problema se conoce como \textbf{sesgo de autoselección}, ya que el grupo se \textit{autoselecciona} para no participar de un programa. Más concretamente, el sesgo de autoselección se produce cuando los motivos por los que un individuo participa en un programa están correlaciones con los resultados \cite{gertler-2016}, esto no es tenido en cuenta a la hora de estimar el impacto.

A modo de ejemplo, se puede pensar en un programa de nutrición infantil. Podría ocurrir que las madres de familia participantes del programa sean más proactivas respecto al desarrollo de sus hijos, por lo cual se preocuparon en lograr la participación en el programa. El problema de autoselección en este caso radica en que la motivación de las madres (que no se observa y es difícil de medir) afecta no solo la probabilidad de participar en el programa, sino también el estado nutricional de los niños ya que estas podrían vigilar mejor la dieta de sus hijos. De esta forma, la diferencia observada en el estado nutricional de los niños de los dos grupos podría deberse parcialmente a la diferencia en el nivel de compromiso de las madres, y no exclusivamente a que un grupo participa en el programa y el otro no \cite{bernal}.

\bigskip
HOLA HOLA 

\begin{comment}
Es importante en este punto identificar claramente los diferentes resultados que hemos mencionado hasta el momento, los cuales se encuentran en la tabla \ref{tab:different-estimations}.

\begin{table}[h]
    \centering
    \begin{tabular}{p{7cm}m{7cm}}  % Set fixed column widths
        \hline
        \textbf{Lo que se desea medir} (efecto promedio sobre los tratados, \(ATT\)) & \(\mathbb{E} \left[Y_i(1)-Y_i(0)|D_i=1\right]\) \\
        \hline
        \textbf{Lo que observamos} (el promedio de la diferencia entre los resultados de los tratados y el grupo de control) & \(\mathbb{E} \left[Y_i(1)-C_i(0)\right]\) \\
        \hline
        \textbf{El promedio de la diferencia potencial entre lo que observamos y lo que se desea medir} (``sesgo de selección'') & \(\mathbb{E} \left[\left(Y_i(1)-Y_i(0)\right) - \left(Y_i(1)-C_i(0)\right)\right] = \mathbb{E} \left[C_i(0)-Y_i(0)\right]\) \\
        \hline
    \end{tabular}
    \caption{Your caption here}
    \label{tab:different-estimations}
\end{table}
\end{comment}


% Los distintos tipos de experimentos?
\subsection{Evaluaciones Experimentales o Aleatorias}
\subsection{Evaluaciones Cuaisexperimentales}
\subsection{Evaluaciones No Experimentales}
\subsubsection{Propensity Score Matching}


\end{document}