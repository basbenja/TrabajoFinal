\documentclass[../../main.tex]{subfiles}

\begin{document}

Las Redes Neuronales Artificiales (RNAs) son un modelo específico dentro del ML, cuya
estructura y funcionamiento estuvieron inspirados inicialmente por el intento de ser
modelos computacionales del aprendizaje biológico, es decir modelos de cómo el aprendizaje
podría ocurrir en el cerebro \cite{deep-learning}. Durante los últimos años, ha sido el
área que más desarrollo e impacto ha tenido, principalmente gracias a su versatilidad,
potencia y escalabilidad, cualidades que hacen que estos modelos sean capaces de
enfrentar problemas grandes y complejos \cite{hands-on-ML-sklearn-tf}, y que sobre todo
parecían extremadamente difíciles de resolver, como son el reconocimiento de voz y la
detección de objetos.

A continuación, damos un breve contexto histórico y luego ahondamos en su funcionamiento.

\subsection{Breve Contexto Histórico}
Aunque su gran éxito ha sido reciente, lo cierto es que la idea de las RNAs data desde
1943, cuando Warren McCulloch y Walter Pitts presentaron en su artículo \textit{A Logical
Calculus of Ideas Immanent of Nervous Activity} \cite{mculloch-pitts-1943} un modelo
computacional simplificado utilizando lógica proposicional acerca de cómo funcionan las
neuronas de cerebros animales en conjunto para llevar a cabo cómputos complejos
\cite{hands-on-ML-sklearn-tf}. Presentaron una versión muy simplificada de la neurona
biológica, que solamente tenía una o más entradas binarias y una salida binaria.

Posteriormente, en 1958, Frank Rosenblatt presentó una de las formas o
\textit{arquitecturas} más simples de RNAs: el ``Perceptrón''
\cite{rosenblatt1958perceptron}. Este servía principalmente para resolver problemas de
clasificación en donde los datos son linealmente separables. Sin emabrgo, su aporte más
notorio fue que definió un algoritmo para el entrenamiento del Perceptrón que le permitía
mejorar automáticamente sus parámetros internos para poder llegar a la solución óptima.

Más tarde, se descubrió que los problemas que no podían ser resueltos por el Perceptrón,
sí podían ser resueltos ``apilando'' múltiples perceptrones, lo cual llevó a la invención
del ``Perceptrón Multicapa'' (PMC), también conocido actualmente como ``Red Neuronal de
Propagación Directa''\footnotemark (del inglés \textit{Feedforward Neural Networks}, FFNN)
\cite{deep-learning}, las cuales conforman el punto de partida de las redes neuronales
actuales. \footnotetext{Si bien estos términos se suelen usar indistintamente, la realidad
es que las redes neuronales feedforward tienen algunas diferencias con  respecto al
Perceptrón Multicapa.}

Para explicar la idea y los elementos presentes detrás de estos algoritmos, tomaremos como
referencia las redes neuronales Feedforward, y en particular las llamadas ``totalmente
conectadas'' (\textit{fully connected}), que definiremos a continuación.

\subsection{Red Neuronal Feedforward}
Como dijimos anteriormente, las redes neuronales son un modelo particular de Aprendizaje
Automático. Aquí, las funciones hipótesis se caracterizan por incorporar \textbf{no
linealidad} y toman la forma de circuitos algebraicos complejos con conexiones que pueden
tener diferentes ``intensidades'' \cite{ai-a-modern-approach}. La idea principal en estos
circuitos es que el ``camino'' recorrido al realizar el cómputo tenga varios pasos como
para permitir que las variables de entrada puedan interactuar de formas complejas. Esto
hace que sean lo suficientemente expresivos como para poder capturar la complejidad de los
datos del mundo real \cite{ai-a-modern-approach}.

Más concretamente, estos modelos son llamados \textit{redes} porque el espacio de
funciones que proveen está formado en realidad por la composición de varias funciones
\cite{deep-learning}. Por ejemplo, podríamos tener la composición de tres funciones
\(f^{(1)}\), \(f^{(2)}\) y \(f^{(3)}\) para formar la siguiente función:
\begin{equation}
    f(\bm{x}) = f^{(3)}(f^{(2)}(f^{(1)}(\bm{x})))
    \label{eq:fun-composition}
\end{equation}
donde \(\bm{x}\) es un vector de dimensión \(n\) de números reales: \(\bm{x}=(x_1,
x_2, ..., x_n)\), es decir \(\bm{x} \in \mathbb{R}^n\).

Usualmente, se dice que las redes están organizadas en \textbf{capas}. De esta forma, en
la ecuación anterior, a \(\bm{x}\) se la conoce como \textbf{capa de entrada}, a las
funciones \(f^{(1)}\) y \(f^{(2)}\) como \textbf{capa ocultas o intermedias}, y a
\(f^{(3)}\), que es la que produce el resultado final, como \textbf{capa de salida}. La
longitud de esta cadena de funciones es la que va a dar la \textbf{profundidad} de la red.

Las RNAs tienen una capa de entrada y una de salida, pero el número de capas ocultas
depende de quien la diseñe. Cuando tienen una capa oculta, se las llama ``superficiales''
(o poco profundas, del inglés \textit{shallow}), y cuando tienen más de una,
\textit{profundas}. Es por esto que al hablar de redes neuronales, muchas veces se
hace referencia al término \textbf{Aprendizaje Profundo}.

Recordemos que uno de los elementos presentes en el Aprendizaje Supervisado es el conjunto
de entrenamiento, compuesto por pares de entrada-etiqeuta. Ahora bien, lo interesante de
estos modelos es que si bien este conjunto especifica qué tiene que producir la capa de
salida ante cada entrada particular, no determina cuál debe ser el comportamiento de las
otras capas \cite{deep-learning}. En cambio, es el algoritmo de aprendizaje el que tiene
que decidir cómo usarlas para lograr una buena aproximación de la función desconocida.

Una forma muy común y más intuitiva de pensar estos modelos es a través de grafos
dirigidos cuyas flechas describen cómo están compuestas las funciones y cómo fluye la
información a través de ellas. Si hay una flecha que une a dos nodos, diremos que están
``conectados''. Por ejemplo, la Ecuación \ref{eq:fun-composition} se representaría de la
siguiente manera:
\begin{center}
    \begin{tikzpicture}[
        node distance=2cm,
        every node/.style={draw, circle, minimum size=1cm, align=center},
        thickborder/.style={draw, circle, minimum size=1cm, align=center, line width=0.5mm}
    ]
        % Nodes
        \node [node hidden] (n1) [thickborder] {\( f^{(1)} \)};
        \node [node hidden] (n2) [thickborder, right=of n1] {\( f^{(2)} \)};
        \node [node out] (n3) [thickborder, right=of n2] {\( f^{(3)} \)};

        \node [node in] (x) [left=of n1 ] {\(\bm{x}\)};
        \node [right=of n3, draw=none] (y) {\(\bm{\hat{y}}\)};

        % Edges
        \draw [connect arrow] (x.east) -- (n1.west);
        \draw [connect arrow] (n1.east) -- (n2.west);
        \draw [connect arrow] (n2.east) -- (n3.west);
        \draw [connect arrow] (n3.east) -- (y.west);
    \end{tikzpicture}
\end{center}

En este caso, la entrada \(\bm{x}\) va a la función \(f^{(1)}\), la salida de
\(f^{(1)}(\bm{x})\) va a \(f^{(2)}\), y la salida de \(f^{(2)}(f^{(1)}(\bm{x}))\)
va directamente a \(f^{(3)}\) para de esa forma producir el resultado final \(\bm{\hat{y}} =
f(\bm{x}) = f^{(3)}(f^{(2)}(f^{(1)}(\bm{x})))\).

Es justamente el comportamiento anterior el que caracteriza a las redes neuronales de tipo
\textbf{\textit{Feedforward}}: los datos y resultados fluyen en una sola dirección; cada
nodo computa su resultado y se lo pasa a su sucesor (o sucesores, como veremos más
adelante). En el grafo, esta situación se refleja en el hecho que no hay ciclos, por lo
que las redes de este tipo se representan por medio de grafos dirigidos y acíclicos.

Ahora bien, ¿qué es exactamente una capa? En la terminología de redes neuronales, una capa
es un conjunto de \textbf{unidades} o, tomando en cuenta su inspiración biológica,
\textbf{neuronas}, que actúan en paralelo. Cada unidad representa una función que toma un
vector y retorna un escalar, y se asemejan a las neuronas biológicas en el sentido que
reciben entradas (o estímulos) de otras unidades y computan su propio valor de activación
\cite{deep-learning}.

De esta forma, la capa de entrada va a tener tantas neuronas como la dimensión de los
datos de entrada (\(\bm{x}\)). Es decir, si \(\bm{x}\) es de dimensión \(n\), entonces la
capa de entrada va a tener \(n\) unidades. Sin embargo, la cantidad de neuronas de cada
capa oculta depende del diseño de la red, y la de la capa de salida depende sobre todo del
problema que se esté tratando de resolver. Si se trata por ejemplo de un problema de
clasificación, entonces la capa de salida va a tener en general tantas neuronas como
categorías existan en el dominio del problema.

Teniendo el concepto de neuronas, podemos introducir el de redes \textbf{totalmente
conectadas} (\textit{fully connected}), que son aquellas en las que cada unidad de una
capa se conecta con todas las de la capa siguiente, ``pasándole'' su valor computado
a todas ellas.

Con esto en mente, podemos concretizar un poco más el ejemplo con el que venimos
trabajando suponiendo que \(\bm{x}\) es un vector de dimensión 3, las capas ocultas dadas
por \(f^{(1)}\) y \(f^{(2)}\) tienen 4 y 3 neuronas respectivamente y la capa de salida
tiene 2. Así, suponiendo que nuestra red es fully connected, nuestro grafo resultaría en
el de la Figura \ref{fig:ff-neural-network}, donde el superíndice de cada nodo indica el
número de capa y el subíndice hace referencia al número de neurona en esa capa.
\begin{figure}
    \centering
    % NEURAL NETWORK with coefficients, arrows
    \begin{tikzpicture}[
        x=2.2cm,
        y=1.4cm
    ]
        \readlist\Nnod{3,4,3,2} % array of number of nodes per layer

        \foreachitem \N \in \Nnod{ % loop over layers
            \edef\lay{\Ncnt} % alias of index of current layer
            \message{\lay,}
            \pgfmathsetmacro\prev{int(\Ncnt-1)} % number of previous layer
            \foreach \i [evaluate={\y=\N/2-\i; \x=\lay; \n=\nstyle;}] in {1,...,\N}{ % loop over nodes
                % NODES
                \ifnum\lay = 1 % input layer
                    \node[node \n] (N\lay-\i) at (\x,\y) {$x_\i$};
                \else
                    \node[node \n] (N\lay-\i) at (\x,\y) {$f_\i^{(\prev)}$};
                \fi
                % CONNECTIONS
                \ifnum\lay > 1 % connect to previous layer
                    \foreach \j in {1,...,\Nnod[\prev]}{ % loop over nodes in previous layer
                        \draw[connect arrow] (N\prev-\j) -- (N\lay-\i); % connect arrows directly
                    }
                \fi % else: nothing to connect first layer
            }
        }

        % Add output nodes to the left of each final layer neuron
        \node[right=of N\Nnodlen-1, circle] (y1) {\(\hat{y}_1\)};
        \node[right=of N\Nnodlen-2, circle] (y2) {\(\hat{y}_2\)};

        % Connect final layer neurons to their corresponding output nodes
        \draw[connect arrow] (N\Nnodlen-1) -- (y1);
        \draw[connect arrow] (N\Nnodlen-2) -- (y2);
    \end{tikzpicture}
    \caption{Red Neuronal de tipo Feedforward y totalmente conectada, con la capa de
    entrada formada por 3 neuronas, dos capas ocultas, cada una formada por 4 y 3 neuronas
    respectivamente, y la capa de salida, formada por dos neuronas. En este caso, la red
    ``acepta'' entradas \(\bm{x} \in \mathbb{R}^3\) y produce salidas \(\bm{\hat{y}} \in
    \mathbb{R}^2\). Adaptado de \cite{tikz-neural-networks}.}
    \label{fig:ff-neural-network}
\end{figure}

A partir de la Figura \ref{fig:ff-neural-network}, se puede ver que cada neurona de la
capa de entrada representa un elemento del vector de entrada, pero las neuronas de tanto
la capa oculta como la de salida reciben las salidas de las neuronas de la capa anterior.
Veamos entonces qué hace concretamente una neurona o unidad.

Una neurona simplemente calcula una suma pesada de sus entradas, provenientes de las
unidades de la capa anterior, y luego aplica una función \textbf{no lineal} para producir
su salida. Esta función se denomina \textbf{función de activación}, y el hecho que sea no
lineal es importante ya que de no ser así, cualquier composición de unidades podría
representarse mediante una función lineal \cite{ai-a-modern-approach}. Como mencionamos
anteriormente, es justamente esta no linealidad lo que permite a estos modelos representar
funciones arbitrarias \cite{ai-a-modern-approach} y complejas. En general, se asume que
todas las neuronas de una capa tienen la misma función de activación, pero puede ocurrir
que diferentes capas tengan diferentes funciones de activación.

Más precisamente, una función de activación es una función no lineal que toma cualquier
valor real como entrada y produce como resultado un número en un determinado rango.
Algunas funciones de activación comunes son las siguientes:
\begin{itemize}[itemsep=0.1cm]
    \item \textbf{Sigmoide}: produce un valor entre 0 y 1. Es por ello que se suele usar
    en problemas de clasificación binaria para que una única neurona en la capa de salida
    represente la probabilidad de que la entrada pertenezca a la clase ``positiva''
    (la que corresponde a la etiqueta 1).
    \[\sigma(x)=\frac{1}{1+e^{-x}}\]
    \item \textbf{ReLU} (abreviatura de \textit{Rectified Linear Unit}): produce un
    valor entre 0 e \(\infty\). Se utiliza mayormente en las neuronas de capas ocultas.
    \[\text{ReLU}(x) = \text{max}(0, x)\]
    \item \textbf{Tangente hiperbólica}: produce un valor entre -1 y 1. Lo particular de
    esta es que mantiene el signo de la entrada.
    \[\text{tanh}(x)=\frac{e^x - e^{-x}}{e^x + e^{-x}}\]
\end{itemize}
Actualmente, la más utilizada es la ReLU y variantes de ella.

Para formalizar el cómputo de una neurona, es necesario introducir cierta notación.
Denotaremos con:
\begin{itemize}[itemsep=0.1cm]
    \item \(s^{(k)}_j\) a la salida de la unidad \(j\) de la capa \(k\)
    \item \(a^{(k)}_j\) a la función no lineal de la unidad \(j\) de la capa \(k\)
    \item \(w^{(k)}_{i,j}\) la intensidad o \textbf{peso} de la conexión entre la
    neurona \(i\) de la capa \(k\) y la \(j\) de la capa \((k+1)\),
\end{itemize}
tenemos que:
\begin{equation}
    s^{(k)}_j = a^{(k)}_j \left( \sum_i w^{(k-1)}_{i,j} s^{(k-1)}_i \right)
    \label{eq:neuron}
\end{equation}
donde el ínidice \(i\) de la sumatoria va a recorrer todas las neuronas de la capa
anterior.

%Gráficamente:
% Poner gráfico basado en la figura figs/neurona_orig.png



% Hablar en este punto que la función de activación se aplica en problemas de clasificación
% es la sigmoide en la capa de salida?

Volviendo a la Fórmula \ref{eq:neuron}, si en nuestro ejemplo tomamos la neurona
\(f^{(2)}_1\), tenemos que su salida estará dada por:
\begin{align*}
    s^{(2)}_1 &= a^{(2)}_1 \left( \sum_{i=1}^{3} w^{(1)}_{i,1} s^{(1)}_i \right) \\
              &= a^{(2)}_1 \left( w^{(1)}_{1,1}s^{(1)}_1 +  w^{(1)}_{2,1}s^{(1)}_2 + w^{(1)}_{3,1}s^{(1)}_3 \right)
\end{align*}

En las redes, se estipula que cada unidad tiene una entrada extra desde una neurona
\textit{dummy} de la capa anterior, para la cual utilizaremos el subíndice 0. El valor de
salida de esta neurona se fija en 1, y el peso asociado con una neurona \(j\) de una capa
\(k\) será \(w^{(k)}_{0,j}\). Este peso se suele llamar
\textit{\textbf{bias}}\footnotemark\ y permite que la entrada a dicha neurona sea distinta
de 0 incluso cuando todas las salidas de la capa anterior sean 0
\cite{ai-a-modern-approach}. Agregándolo, podemos escribir la Ecuación \ref{eq:neuron} de
forma vectorizada:
\begin{equation}
    s^{(k)}_j = a^{(k)}_j \left( \bm{w}^{(k-1)}_j \left( \bm{s}^{(k-1)}_j \right)^T \right)
\end{equation}
donde: \vspace{-0.25cm}
\begin{itemize}
    \item \(\bm{w}^{(k-1)}_j\) es el vector de todos los pesos que salen de las neuronas de la
    capa \((k-1\)) y se dirigen a la unidad \(j\) de la capa \(k\) (incluyendo \(w_{0,j}\))
    \item \(\bm{s}^{(k-1)}_j\) es el vector de todas las salidas de la capa anterior que se
dirigen a la unidad \(j\) de la capa \(k\) (incluyendo el 1 fijo de la neurona dummy).
\end{itemize}
\footnotetext{En la bibliografía sobre redes neuronales, también se suele presentar a este
peso como un elemento ``aparte'' de la neurona, no como un peso extra, y se lo denota como
\(b^{(k)}_j\).}

\vspace{-0.2cm}
Con esta notación, si tomamos nuevamente a \(f^{(2)}_1\), los vectores involucrados van a
ser \(\bm{w}^{(1)}_1\) y \(\bm{s}^{(1)}_1\), dados en este caso por:
\begin{quote}
    \(\bm{w}^{(1)}_1 = (w^{(1)}_{0,1}, w^{(1)}_{1,1}, w^{(1)}_{2,1}, w^{(1)}_{3,1}, w^{(1)}_{4,1})\),
    \(\bm{s}^{(1)}_1 = (1, s^{(1)}_1, s^{(1)}_2, s^{(1)}_3, s^{(1)}_4)\)
\end{quote}
Esto se puede ver mejor gráficamente haciendo foco en dicha neurona, como lo ilustra
la Figura \ref{fig:neuron-weights}.
\begin{figure}[ht]
    \centering
    \begin{tikzpicture}[x=2.7cm,y=1.6cm]
        \def\NI{4} % number of nodes in input layers
        \def\NO{3} % number of nodes in output layers
        \def\yshift{0.4} % shift last node for dots

        % INPUT LAYER
        \foreach \i in {0,...,\NI} {
            \pgfmathsetmacro{\y}{\NI/2 - \i}
            \ifnum\i=0
                \node[node hidden,outer sep=0.6] (NI-\i) at (0,\y) {1};
            \else
                \node[node hidden,outer sep=0.6] (NI-\i) at (0,\y) {\(s^{(1)}_{\i}\)};
            \fi
        }

        % OUTPUT LAYER
        \foreach \i in {\NO,...,1}{ % loop over nodes
            \pgfmathsetmacro{\y}{\NO/2 - \i}
            \ifnum\i=1 % high-lighted node
                \node[node hidden] (NO-\i) at (1,\y) {\(f^{(2)}_{\i}\)};
                \foreach \j in {0,...,\NI}{ % loop over nodes in previous layer
                    \draw[connect arrow] (NI-\j) -- (NO-\i)
                        node[pos=0.50, fill=white] {\contour{white}{\footnotesize{\(w^{(1)}_{\j,\i}\)}}};
                }
            \else % other light-colored nodes
                \node[node,blue!20!black!80,draw=myblue!20,fill=myblue!5]
                (NO-\i) at (1,\y) {\(f^{(2)}_{\i}\)};
                \foreach \j in {0,...,\NI}{ % loop over nodes in previous layer
                    \draw[connect arrow,myblue!20] (NI-\j) -- (NO-\i);
                }
            \fi
        }

        % \node[below=16,right=5,mydarkblue,scale=0.95] at (NO-1)
        \def\agr#1{{s_{#1}^{(1)}}}
        \node[right=of NO-1] at (0.8, -0.45){
            $\begin{aligned}
                &= a^{(2)}_1\left( \color{black}
                        w^{(1)}_{0,1} + w_{1,1}\agr{1} + w_{2,1}\agr{2} + w_{3,1}\agr{3} + w_{4,1}\agr{4}
                    \right) \\
                &= a^{(2)}_1\left( \color{black}
                    \begin{bmatrix}
                        w^{(1)}_{0,1} & w^{(1)}_{1,1} & w^{(1)}_{2,1} & w^{(1)}_{3,1} & w^{(1)}_{4,1}
                    \end{bmatrix}
                    \begin{bmatrix}
                        1 \\ s^{(1)}_1 \\ s^{(1)}_2 \\ s^{(1)}_3 \\ s^{(1)}_4
                    \end{bmatrix}
                    \right)
            \end{aligned}$
        };
    \end{tikzpicture}
    \caption{Entradas a la neurona \(f^{(2)}_1\), junto con sus pesos asociados. Se
    incluyen tanto la neurona dummy de la capa anterior como su bias. En este caso, se
    usan indistintamente las letras \(s\) y \(f\) para denotar a las neuronas. Adaptado de
    \cite{tikz-neural-networks}.}
    \label{fig:neuron-weights}
\end{figure}


Así como utilizamos vectores para describir el cómputo de una neurona, podemos emplear
matrices para describir el comportamiento de toda una capa. Para ello, tomemos la siguiente
notación, para lo cual resulta conveniente fijar una capa \(k\), que tiene \(m\)
neuronas:
\begin{itemize}[itemsep=0.1cm]
    \item \(\bm{s}^{(k)}\) es el vector columna formado por las salidas de la capa
    \(k\). Es decir: \(\bm{s}^{(k)} = (s_0^{(k)}, s_1^{(k)}, ..., s_m^{(k)})^T = (1,
    s_1^{(k)}, ..., s_m^{(k)})^T\), de dimensión \(m \times 1\).
    \item \(\bm{a}^{(k)}\) es la función de activación de la capa \(k\), con una aplicación
    elemento a elemento. Es decir: \(\bm{a}^{(k)}(x_1, x_2, ..., x_m) = (a^{(k)}(x_1),
    a^{(k)}(x_2), ..., a^{(k)}(x_m))\), donde \(a^{(k)}\) es la función de activación
    de todas las neuronas de la capa \(k\) con aplicación a un número.
    \item \(\bm{W}^{(k)}\) es la matriz de pesos que salen de la capa \(k\). Cada fila
    \(j\) de esta matriz corresponde a los pesos que sale de todas las neuronas de la capa
    \(k\) (incluyendo el bias) y se dirigen a la neurona \(j\) de la capa \((k+1)\). Es
    decir cada fila es \(\bm{w}^{(k)}_j\) con \(j = 1,...,n\), y \(n\) la cantidad de
    neuronas de la capa \((k+1)\) (sin contar la dummy, que ``aparece después''). Así,
    esta matriz tiene dimensiones \(n \times m\).
\end{itemize}
Con esto, tenemos que la salida de una capa \(k\) está dada por:
\[
    \bm{s}^{(k)} = \bm{a}^{(k)} \left( \bm{W}^{(k-1)} \bm{s}^{(k-1)} \right)
\]

Con todo esto en mente, veamos cuál es la función que describe la red neuronal presentada
como ejemplo:
\begin{align*}
    \bm{\hat{y}} =\ & \bm{s}^{(3)} \\
        =\ & \bm{a}^{(3)} \left( \bm{W}^{(2)} \bm{s}^{(2)} \right) \\
        =\ & \bm{a}^{(3)} \left(
            \bm{W}^{(2)} \bm{a}^{(2)} \left(
                \bm{W}^{(1)} \bm{s}^{(1)}
            \right)
        \right) \\
        =\ & \bm{a}^{(3)} \left(
            \bm{W}^{(2)} \bm{a}^{(2)} \left(
                \bm{W}^{(1)} \bm{a}^{(1)} \left( \bm{W}^{(0)} \bm{s}^{(0)} \right)
            \right)
        \right) \\
        =\ & \bm{a}^{(3)} \left(
            \bm{W}^{(2)} \bm{a}^{(2)} \left(
                \bm{W}^{(1)} \bm{a}^{(1)} \left( \bm{W}^{(0)} \bm{x}^T \right)
            \right)
        \right)
\end{align*}

Y si queremos profundizar aún más esta ecuación para ver dónde aparece cada peso:
\begin{align*}
    \bm{\hat{y}}
    &=\ \bm{a}^{(3)} \left(
        \bm{W}^{(2)} \bm{a}^{(2)} \left(
            \bm{W}^{(1)} \bm{a}^{(1)} \left( \bm{W}^{(0)} \bm{x}^T \right)
        \right)
    \right) \\
    &=\ \bm{a}^{(3)} \left(
            \bm{W}^{(2)} \bm{a}^{(2)} \left(
                \bm{W}^{(1)} \bm{a}^{(1)} \left(
                    \bm{W}^{(0)} \begin{bmatrix} 1 \\ x_1 \\ x_2 \\ x_3 \end{bmatrix}
            \right)
        \right)
    \right) \\
    &=\ \bm{a}^{(3)} \left(
            \bm{W}^{(2)} \bm{a}^{(2)} \left(
                \bm{W}^{(1)} \bm{a}^{(1)} \left(
                    \begin{bmatrix}
                        w_{0,1}^{(0)} & w_{1,1}^{(0)} & w_{2,1}^{(0)} & w_{3,1}^{(0)} \\
                        w_{0,2}^{(0)} & w_{1,2}^{(0)} & w_{2,2}^{(0)} & w_{3,2}^{(0)} \\
                        w_{0,3}^{(0)} & w_{1,3}^{(0)} & w_{2,3}^{(0)} & w_{3,3}^{(0)} \\
                        w_{0,4}^{(0)} & w_{1,4}^{(0)} & w_{2,4}^{(0)} & w_{3,4}^{(0)}
                    \end{bmatrix}
                    \begin{bmatrix} 1 \\ x_1 \\ x_2 \\ x_3 \end{bmatrix}
            \right)
        \right)
    \right) \\
    &=\ \bm{a}^{(3)} \left(
            \bm{W}^{(2)} \bm{a}^{(2)} \left(
                \bm{W}^{(1)} \bm{a}^{(1)} \left(
                    \begin{bmatrix}
                        w_{0,1}^{(0)} + w_{1,1}^{(0)} x_1 + w_{2,1}^{(0)} x_2 + w_{3,1}^{(0)} x_3 \\
                        w_{0,2}^{(0)} + w_{1,2}^{(0)} x_1 + w_{2,2}^{(0)} x_2 + w_{3,2}^{(0)} x_3 \\
                        w_{0,3}^{(0)} + w_{1,3}^{(0)} x_1 + w_{2,3}^{(0)} x_2 + w_{3,3}^{(0)} x_3 \\
                        w_{0,4}^{(0)} + w_{1,4}^{(0)} x_1 + w_{2,4}^{(0)} x_2 + w_{3,4}^{(0)} x_3
                    \end{bmatrix}
            \right)
        \right)
    \right) \\
    &=\ \bm{a}^{(3)} \left(
            \bm{W}^{(2)} \bm{a}^{(2)} \left(
                \bm{W}^{(1)}
                \begin{bmatrix}
                    a^{(1)} \left( w_{0,1}^{(0)} + w_{1,1}^{(0)} x_1 + w_{2,1}^{(0)} x_2 + w_{3,1}^{(0)} x_3 \right) \\
                    a^{(1)} \left( w_{0,2}^{(0)} + w_{1,2}^{(0)} x_1 + w_{2,2}^{(0)} x_2 + w_{3,2}^{(0)} x_3 \right) \\
                    a^{(1)} \left( w_{0,3}^{(0)} + w_{1,3}^{(0)} x_1 + w_{2,3}^{(0)} x_2 + w_{3,3}^{(0)} x_3 \right) \\
                    a^{(1)} \left( w_{0,4}^{(0)} + w_{1,4}^{(0)} x_1 + w_{2,4}^{(0)} x_2 + w_{3,4}^{(0)} x_3 \right)
                \end{bmatrix}
        \right)
    \right) \\
    &=\ \left( \dots \right)
\end{align*}

Las librerías que implementan estos modelos utilizan esta notación matricial ya que
suelen estar optimizadas para cálculos con matrices.

Algo a notar es que en las redes neuronales, aparecen nuevos hiperparámetros, que ya no
tienen tanto que ver con el proceso de entrenamiento en sí, sino más bien con el diseño y
la capacidad de la red, como son la cantidad de capas ocultas, la cantidad de neuronas en
cada capa oculta, e incluso la función de activación de cada capa. Para todos estos,
se puede encontrar el valor óptimo utilizando las técnicas mencionadas en la sección
anterior.

\subsubsection{Entrenamiento}
En el caso de las redes neuronales, los parámetros a optimizar van a ser las intensidades
de las conexiones entre las neuronas, que también venimos llamando pesos. Los
optimizadores que se utilizan actualmente se basan en la regla del DG, aunque con algunas
mejoras.

Como mencionamos en la sección anterior, lo costoso de la regla clásica del DG presentada
hasta el momento es que requiere calcular el gradiente de la función de pérdida
considerando \textit{todas} las muestras del conjunto de entrenamiento. Para reducir esta
carga, en la práctica se recurre a una aproximación: en lugar de usar todas las muestras,
se selecciona aleatoriamente un subconjunto de ellas, llamado \textbf{lote}
(\textit{batch}), y se calcula el gradiente usando únicamente ese lote. Luego, los pesos
se actualizan con base en esta estimación parcial. Luego, se actualizan los pesos en base
a este estimación. Cabe aclarar que el tamaño de lote, es decir la cantidad de ejemplos
que se eligen cada vez, se mantiene fijo en todo el entrenamiento y constituye un
hiperparámetro. Esta técnica suele llamarse optimización por \textbf{mini-lotes} o
\textbf{estocástica}, ya que introduce cierto grado de aletoriedad en el cálculo
del gradiente.

Otra mejora aplicada sobre el método del DG estándar, diseñada para acelerar el
aprendizaje, es la técnica conocida como \textbf{momentum}. A grandes rasgos, consiste en
actualizar los pesos mirando no solo el gradiente de la iteración actual, sino también
dándole un peso a la acumulación de gradientes de las iteraciones anteriores. Esto permite
sobre todo acelerar las actualizaciones cuando se encuentran sucesivos gradientes que
apuntan en la misma dirección \cite{deep-learning}. Un optimizador ampliamente utilizado
actualmente, y que es el que empleamos en este trabajo, es \textbf{Adam} \cite{adampaper}.
Este combina momentum con una tasa de aprendizaje adaptativa para cada parámetro.

Para entender cómo progresa el entrenamiento, es útil definir el concepto de definir el
concepto de \textbf{época}. Una época se refiere al proceso completo en el cual el modelo
se entrena utilizando \textbf{todos} los datos disponibles en el conjunto de entrenamiento
una vez. A grandes rasgos, los pasos involucrados en una época son los siguientes:
\begin{enumerate}[itemsep=0.05cm,label=\textbf{\arabic*.}]
    \item Mezclar el conjunto de entrenamiento. Este paso en realidad es opcional
    pero evita que el modelo aprenda patrones no deseados debido al orden de los datos.
    \item Seleccionar un lote del tamaño predefinido del conjunto de datos de entrenamiento.
    \item Para cada ejemplo del lote:
    \vspace{-0.2cm}
    \begin{enumerate}[noitemsep]
        \item Computar la predicción del modelo.
        \item Calcular la pérdida.
    \end{enumerate}
    \item Promediar el error a lo largo de todos los ejemplos del lote. Este valor escalar
    no se utiliza directamente en el entrenamiento, pero se emplea comúnmente para
    monitorear el progreso del aprendizaje.
    \item Calcular el gradiente utilizando solamente las muestras del lote.
    \item Actualizar los pesos.
    \item Volver a 2.
    \item Una vez que se han recorrido todos los lotes, finaliza la época.
\end{enumerate}

Ahora bien, un paso fundamental en el entrenamiento de las redes neuronales y del que vale
la pena profundizar es el del cálculo del gradiente. De hecho, no encontrar una solución
eficiente para lograrlo detuvo el avance de estos modelos durante varios años. El algoritmo
que vino a dar respuesta y que es el utilizado hasta hoy es conocido como
\textbf{retropropagación} (del inglés \textit{backpropagation}) y fue presentado en el año
1986 \cite{backprop-1986}.

El backpropagation se basa fundamentalmente en la regla de la cadena para derivadas de
funciones compuestas y se divide en dos etapas: primero, el paso hacia adelante
(\textit{forward pass}) y luego, el paso hacia atras (\textit{backward pass}). Para
entender estas etapas, supondremos que solamente una entrada es provista a la red.

En el forward pass, la red simplemente lleva a cabo una predicción para la entrada,
realizando todo el cómputo intermedio necesario para llegar a producir una salida.

El backward pass comienza por calcular el error cometido por la red para la entrada dada,
y consiste en propagar este error desde la capa de salida hasta la de entrada, midiendo la
contribución al error de cada conexión \cite{hands-on-ML-sklearn-tf}. Este paso se basa
fuertemente en la idea que un pequeño cambio en uno de los pesos causa un efecto cadena
sobre el cómputo restante de la red \cite{prince2024understanding}. Para verlo, supongamos
que tenemos una red neuronal con tres capas ocultas, que denotaremos con \(\bm{h}^{(1)}\),
\(\bm{h}^{(2)}\) y \(\bm{h}^{(3)}\) (la cantidad de neuronas en cada capa es irrelevante),
y veamos cómo computaríamos el efecto de un cambio en un peso
\cite{prince2024understanding}:
\begin{itemize}
    \item Para calcular cómo un cambio en un peso que se dirige a \(\bm{h}^{(3)}\)
    modifica el valor de la pérdida, necesitamos saber (i) cómo un cambio en \(\bm{h}^{(3)}\)
    modifica la salida \(\bm{f}\) del modelo y (ii) cómo un cambio en la salida
    modifica la pérdida.
    \item Para calcular cómo un cambio en un peso que se dirige a \(\bm{h}^{(2)}\)
    modifica el valor de la pérdida, necesitamos saber (i) cómo un cambio en
    \(\bm{h}^{(2)}\) afecta a \(\bm{h}^{(3)}\), (ii) cómo un cambio en \(\bm{h}^{(3)}\)
    modifica la salida \(\bm{f}\) del modelo y (iii) cómo un cambio en la salida
    modifica la pérdida.
    \item Para calcular cómo un cambio en un peso que se dirige a \(\bm{h}^{(1)}\)
    modifica el valor de la pérdida, necesitamos saber (i) cómo un cambio en
    \(\bm{h}^{(1)}\) afecta a \(\bm{h}^{(2)}\), (ii) cómo un cambio en
    \(\bm{h}^{(2)}\) afecta a \(\bm{h}^{(3)}\), (iii) cómo un cambio en \(\bm{h}^{(3)}\)
    modifica la salida \(\bm{f}\) del modelo y (iv) cómo un cambio en la salida
    modifica la pérdida.
\end{itemize}
Mientras nos movemos hacia las primeras capas de la red, vemos que la mayoría de los
términos que se necesitan ya han sido calculados en pasos anteriores, por lo que no es
necesario recomputarlos. Justamente calcular los efectos de los cambios - que son
básicamente las derivadas parciales - de esta manera es conocido como el backward pass.

El paso final del backpropagation es actualizar los pesos con los gradientes calculados
en el backward pass y utilizando la regla del DG.

\bigskip
En las siguientes secciones, hablaremos sobre dos tipos particulares de redes neuronales
de las que hacemos uso en este trabajo: las Redes Neuronales Convolucionales y las Redes
Neuronales Recurrentes. Cada una fue diseñada originalmente para trabajar con un tipo
específico de datos. Las convolucionales son ideales para procesar datos estructurados en
forma de grilla, mientras que las recurrentes son adecuadas para secuencias temporales.
Presentaremos la intuición sobre detrás de ellas y nos concentraremos en su aplicación
sobre series de tiempo, relevante para nuestro trabajo.

\end{document}