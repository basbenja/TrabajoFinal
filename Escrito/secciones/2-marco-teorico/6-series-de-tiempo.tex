\documentclass[../../main.tex]{subfiles}
% \graphicspath{{\subfix{../images/}}}

\begin{document}
Como mencionamos anteriormente y como veremos en las secciones posteriores, nuestros
experimentos fueron llevados a cabo sobre datos de series temporales, muy propios del área
de la econometría, dentro de la cual se engloba la evaluación de impacto. Por lo tanto,
dedicaremos esta sección a hablar sobre este tipo de datos, haciendo hincapié sobre lo que
es relevante para este trabajo.

Una serie de tiempo es simplemente una secuencia de observaciones de una o más variables a
lo largo del tiempo. Cuando es solamente una la variable observada, como en nuestro caso,
se dice que la serie es univariada. Y, cuando se trabaja con una base de datos en donde a
cada unidad bajo estudio le corresponde una serie de tiempo, se dice que es un conjunto de
datos \textbf{de panel}.

En nuestro caso, trabajamos con datos de panel con series de tiempo univariadas, en donde
cada una fue generada artificialmente intentando simular escenarios reales de eventos
sociales, en donde siempre está presente la aleatoriedad y la dependencia con valores del
pasado. Puntualmente, modelamos las series con \textbf{procesos autorregresivos de orden
uno con efectos fijos y efectos temporales}. A continuación, explicaremos qué signfica
esto matemáticamente. Para ello, utilizaremos la siguiente notación:
\begin{itemize}
    \item \(y\) es la variable de la cual estamos generando la serie de tiempo. Se
    la suele llamar variable dependiente, ya que su valor depende de otros factores.
    Por ejemplo, la inflación, los ingresos mensuales de una empresa, etcétera.
    \item \(y_t\) es el valor de \(y\) en el período \(t\), con \(t=0,1,...\).
    \item \(\{y_t:t=0,1,..\}\) representa los valores de la serie de tiempo en los
    distintos períodos.
    \item \(\{y_it:t=0,1,..\}\) es la serie de tiempo para la unidad \(i\).
\end{itemize}

Un \textbf{proceso autorregresivo de orden uno}, abreviado como \textbf{AR(1)} es un
modelo de series de tiempo en donde el valor actual de la serie depende linealmente de su
valor más reciente más una perturbación impredecible \cite{intro-econometria}. La fórmula
de un proceso autorregresivo de orden uno es la siguiente:
\[
    y_t = \rho y_{t-1} + e_t
\]
donde a \(\rho\) se lo denomina coeficiente autorregresivo y \(e_t\) para \(t=0,1,...,T\)
es una secuencia de valores independiente e idénticamente distribuidos con media cero y
varianza \(\sigma_e^2\).

Por otro lado, cuando se trabaja con datos de panel, se pueden clasificar los factores no
observables que influyen sobre el valor de la variable dependiente en dos tipos: aquellos
que son constantes y aquellos que varían con el tiempo.




% Una \textbf{serie de tiempo} es una secuencia \textbf{ordenada} de valores medidos
% sucesivamente en intervalos de tiempo igualmente espaciados
% \cite{khan2021-bilstm-for-unitsc}. Este tipo de datos está presente en distintos
% fenómenos, como lecturas meteorológicas, registros financieros, señales fisiológicas, y
% observaciones industriales \cite{wang2016timeseriesclassificationscratch}.

% % Encerrarlo en un tag de definición ?
% Formalmente, una serie de tiempo \textbf{univariada} \(U\) es un conjunto ordenado de
% valores reales \(U = (x_1, x_2, ..., x_N)\), y su dimensión está dada por la cantidad \(N\)
% de valores \cite{khan2021-bilstm-for-unitsc}.

% \subsection{Clasificación de Series de Tiempo con Aprendizaje Automático}
% Teniendo una definición de una serie de tiempo, el problema de \textbf{Clasificación de
% Series de Tiempo} (CST) consiste en: dado un conjunto de clases o categorías \(Y\), y un
% conjunto de datos \(D\) formado por series de tiempo univariadas \(U_i\) donde a cada una
% le corresponde una etiqueta \(y(U_i) \in Y\), el objetivo es encontrar una función \(f\),
% a la que llamaremos ``clasificador'' o ``modelo'' de forma tal que \(f(U) = y(U)\)
% \cite{khan2021-bilstm-for-unitsc}. Como venimos explicando, un buen modelo será uno que
% pueda capturar y generalizar el el patrón de las series de datos de forma tal que sea
% capaz de clasificar correctamente nuevos datos. El repositorio que se suele tomar como
% referencia para evaluar los diferentes modelos de CST es el de la Universidad de
% California Riverside, llamado \textit{UCR Time Series Classification Archive}
% \cite{UCRArchive2018}.

% Durante las últimas décadas, se han propuestos diferentes métodos para afrontar este
% problema. Los primeros trabajaban directamente sobre las series de tiempo utilizando
% alguna medida de similitud predefinida que pueda capturar la similitud entre ellas, como
% por ejemplo la distancia euclideana o la deformación dinámica del tiempo (\textit{Dynamic
% Time Warping}) \cite{wang2016timeseriesclassificationscratch}.

% Se han desarrollado también algoritmos basados en características o \textit{features}, en
% donde cada serie temporal es convertida en un conjunto de features globales, que es
% posteriormente usando para definir la semejanza entre pares de series de tiempo
% \cite{khan2021-bilstm-for-unitsc}. Algunos de los más conocidos son ...

% El gran problema de todas las estrategias mencionadas es que requieren un gran
% preprocesamiento de los datos y una ingeniería de atributos (\textit{feature engineering})
% \cite{wang2016timeseriesclassificationscratch}, para lo cual generalmente se necesita
% tener un buen conocimiento de campo sobre las series sobre las que se está trabajando.

% Es por esto que recientemente, las redes neuronales profundas han comenzado a ser
% utilizadas para clasificar series temporales
% \cite{wang2016timeseriesclassificationscratch} ya que permiten trabajar de forma directa
% con los datos, y se encargan de detectar automáticamente cuáles son las características
% que distinguen a las series. Algunas arquitecturas utilizadas hasta el momento son el
% perceptrón multicapa \cite{wang2016timeseriesclassificationscratch}, redes neuronales
% convolucionales \cite{wang2016timeseriesclassificationscratch}, y estas últimas combinadas
% con unidades LSTM unidireccionales \cite{Karim_2018} y bidireccionales
% \cite{khan2021-bilstm-for-unitsc}.

\end{document}