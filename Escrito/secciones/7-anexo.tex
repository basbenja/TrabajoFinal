\documentclass[../main.tex]{subfiles}

\begin{document}

\section*{Resultados de pruebas de hipótesis}
Como dijimos, utilizamos pruebas de hipótesis\footnote{Para una explicación más detallada
sobre pruebas de hipótesis, se puede consultar el Capítulo 3 de \cite{giuliodori-2022}.}
de diferencia de medias para verificar si las diferencias observadas en el puntaje \(F_1\)
entre cada arquitectura y el PSM son estadísticamente significativas.

Contamos con un modelo A y un modelo B, evaluados independientemente sobre el conjunto de
test de cada una de 100 simulaciones. Ante este escenario, aplicamos la prueba t de
Student para muestras independientes con varianza poblacional desconocida. El estadístico
de prueba sigue una distribución \(t\) de Student con \(n_A + n_B - 2\) grados de libertad,
donde \(n_A\) y \(n_B\) son los tamaños muestrales de cada modelo (ambos iguales a 100 en
este caso).

Concretamente, si denotamos con \(\mu^A_{F_1}\) y \(\mu^B_{F_1}\) a la media poblacional
del puntaje \(F_1\) obtenido con los modelos A y B respectivamente, entonces las hipótesis
nula (\(H_0\)) y alternativa (\(H_1\)) son:
\begin{itemize}[itemsep=0.05cm]
    \item \(H_0\): \(\mu^A_{F_1} = \mu^B_{F_1}\)
    \item \(H_1\): \(\mu^A_{F_1} \ne \mu^B_{F_1}\)
\end{itemize}

En nuestro caso, tomamos un nivel de significancia \(\alpha = 0.05\). De esta forma, si el
\(p\)-valor obtenido en la prueba es menor a \(0.05\) o si el valor absoluto del
estadístico observado \(t_{\text{obs}}\) es mayor a\footnote{Este valor se obtuvo
utilizando la función
\href{https://docs.scipy.org/doc/scipy/reference/generated/scipy.stats.t.html}{\texttt{scipy.stats.t.ppf}}.}
1.972, se rechaza la hipótesis nula, concluyendo que la diferencia observada en el
desempeño de los modelos es estadísticamente significativa.

En la Tabla \ref{tab:pruebas_hipotesis}, se presentan los \(p\)-valores y los estadísticos
observados en los diferentes experimentos al comparar las muestras de \(F_1\) obtenidas
con cada red frente a las obtenidas con el PSM. En todos los casos, se rechaza la
hipótesis nula, lo que permite concluir que las medias poblacionales son
significativamente diferentes.

\newpage

\begin{table}[H]
    \centering
    \begin{tabular}{|c|c|c|c|}
    \hline
        \textbf{Experimento} & \textbf{Arquitectura} & \(t_{\text{obs}}\) & \(p\)-valor \\
    \hline\hline
        \multirow{3}{*}{\textbf{1}}
        & LSTM             & -48.69896 & \( 3.66136 \times 10^{-112} \) \\
        & Convolucional    & -71.20134 & \( 4.91162 \times 10^{-143} \) \\
        & LSTM + Conv      & -73.90965 & \( 3.95276 \times 10^{-146} \) \\
        \hline
        \multirow{3}{*}{\textbf{2}}
        & LSTM             & -86.86190  & \( 1.33819 \times 10^{-159} \) \\
        & Convolucional    & -100.94082 & \( 3.12119 \times 10^{-172} \) \\
        & LSTM + Conv      & -97.97543  & \( 1.01923 \times 10^{-169} \) \\
        \hline
        \multirow{3}{*}{\textbf{3}}
        & LSTM             & -118.25474 & \( 1.26872 \times 10^{-185} \) \\
        & Convolucional    & -131.15133 & \( 2.06197 \times 10^{-194} \) \\
        & LSTM + Conv      & -139.31947 & \( 1.49411 \times 10^{-199} \) \\
        \hline
        \multirow{3}{*}{\textbf{4}}
        & LSTM             & -31.31390 & \( 1.25482 \times 10^{-78} \)  \\
        & Convolucional    & -53.40744 & \( 1.53692 \times 10^{-119} \) \\
        & LSTM + Conv      & -55.37216 & \( 1.87972 \times 10^{-122} \) \\
        \hline
        \multirow{3}{*}{\textbf{5}}
        & LSTM             & -63.45210 & \( 1.53054 \times 10^{-133} \) \\
        & Convolucional    & -78.34732 & \( 5.59729 \times 10^{-151} \) \\
        & LSTM + Conv      & -84.41686 & \( 3.29239 \times 10^{-157} \) \\
        \hline
        \multirow{3}{*}{\textbf{6}}
        & LSTM             &  -83.30212 & \( 4.26036 \times 10^{-156} \) \\
        & Convolucional    & -111.66130 & \( 9.20778 \times 10^{-181} \) \\
        & LSTM + Conv      & -101.00631 & \( 2.75173 \times 10^{-172} \) \\
        \hline
        \multirow{3}{*}{\textbf{7}}
        & LSTM             & -102.06262 & \( 3.64567 \times 10^{-173} \) \\
        & Convolucional    & -111.91466 & \( 5.91960 \times 10^{-181} \) \\
        & LSTM + Conv      & -110.07937 & \( 1.48510 \times 10^{-179} \) \\
        \hline
    \end{tabular}
    \caption{Estadísticos observados y \(p\)-valores en las pruebas de diferencia de
    medias para el puntaje \(F_1\), comparando cada red con el PSM en cada experimento.
    La prueba de hipótesis realizada fue la \(t\) de Student para muestras independientes,
    asumiendo varianzas poblaciones iguales. En todos los casos, el \(p\)-valor es menor
    al nivel de significancia \(\alpha = 0.05\), por lo que se rechaza la hipótesis nula
    que establece que las medias poblacionales del puntaje \(F_1\) en los dos modelos comparados
    en cada prueba son iguales.}
    \label{tab:pruebas_hipotesis}
\end{table}


\end{document}