\documentclass[../../main.tex]{subfiles}

\begin{document}
En lo que sigue, presentamos los resultados obtenidos con las diferentes arquitecturas de
redes neuronales y con el método de PSM en los escenarios modelados. Para cada uno,
detallamos los parámetros utilizados: períodos observados (\(L\)), períodos de dependencia
(\(n_{pd}\)), cantidad máxima de incrementos en los períodos de dependencia (\(m\)) y
media de los efectos fijos de los NiNi (\(\mu_{{EF}_{NiNi}}\)).

Recordemos que, fijados los parámetros de un escenario, se generaron 100 simulaciones del
mismo con el objetivo de obtener resultados estadísticamente significativos. Tanto para
las diferentes arquitecturas de redes como para el PSM, se midió su rendimiento en cada
una de las 100 simulaciones, y luego se calculó el promedio.

En el caso de las redes neuronales, cada simulación incluyó una búsqueda de
hiperparámetros sobre 16 combinaciones distintas, realizada automáticamente por Optuna. El
resultado en cada simulación corresponde a la red entrenada con la mejor combinación de
hiperparámetros hallada para esa simulación.

Mostramos los promedios del puntaje \(F_1\), la precisión y la cobertura en el conjunto de
test a lo largo de las 100 simulaciones. Cabe aclarar que, siguiendo lo explicado en la
Sección \ref{sec:comparacion}, dada una métrica determinada, el resultado obtenido en una
simulación es el promedio de esa métrica entre las cohortes. Además, para las redes
neuronales, de cada hiperparámetro mostramos el valor que fue seleccionado como el óptimo
con mayor frecuencia en las simulaciones correspondientes a un escenario.

Dado que nuestra métrica principal es el puntaje \(F_1\), verificamos si las diferencias
observadas en dicha métrica entre cada arquitectura y el PSM son estadísticamente
significativas mediante pruebas de hipótesis de diferencia de medias. Los detalles y
resultados de estas pruebas se encuentran en el \hyperref[chap:anexo]{Anexo}. En todos los
casos, rechazamos la hipótesis nula que plantea la igualdad de las medias poblacionales,
lo cual implica que las diferencias observadas resultan estadísticamente significativas.

\end{document}