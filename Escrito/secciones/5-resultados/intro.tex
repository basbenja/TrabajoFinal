\documentclass[../../main.tex]{subfiles}

\begin{document}
En lo que sigue, detallamos las diferentes características presentes en cada uno de
los escenarios junto con los resultados obtenidos con las diferentes arquitecturas
de redes neuronales como con el método de PSM.

Recordemos que para cada experimento realizamos 100 simulaciones por modelo, tanto para
las diferentes arquitecturas de redes como para el PSM. En el caso de las redes
neuronales, cada simulación incluyó una búsqueda de hiperparámetros sobre 16 combinaciones
distintas, realizada automáticamente por Optuna. El resultado en cada simulación
corresponde a la red entrenada con la mejor combinación de hiperparámetros hallada para
esa simulación.

Mostramos los promedios del puntaje \(F_1\), la precisión y la sensibilidad en el conjunto
de test a lo largo de las 100 simulaciones. Cabe aclarar que, siguiendo lo explicado en la
Sección \ref{sec:comparacion}, dada una métrica determinada, el resultado obtenido en una
simulación es el promedio obtenido en las cohortes. Además, en las redes neuronales, de
cada hiperparámetro, mostramos el valor que fue seleccionado como el óptimo con mayor
frecuencia en las simulaciones.

Como mencionamos previamente, denotamos con:
\begin{itemize}
    \item \(L\) a la cantidad de períodos observados antes del inicio del programa.
    \item \(n_{pd}\) a la cantidad de períodos de dependencia, con \(0 \le n_{pd} < L\).
    \item \(m\) a la cantidad de veces, dentro de los períodos de dependencia, en los que
    permitimos que el valor de la variable sea mayor al inmediato anterior.
\end{itemize}
Y recordemos que en cada escenario simulamos la existencia de 3 cohortes.
\end{document}