\documentclass[../../main.tex]{subfiles}

\begin{document}
En lo que sigue, detallamos las diferentes características presentes en cada uno de los
escenarios (períodos observados, períodos de dependencia, máxima cantidad de incrementos
en los períodos de dependencia y media de los efectos fijos de los NiNi) junto con los
resultados obtenidos con las diferentes arquitecturas de redes neuronales como con el
método de PSM.

Recordemos que fijados los parámetros de un escenario, generamos 100 simulaciones
del mismo para poder garantizar resultados estadísticamente significativos. Tanto para
las diferentes arquitecturas de redes como para el PSM, se midió su rendimiento en
cada una de las 100 simulaciones para luego poder promediarlo.

En el caso de las redes neuronales, cada simulación incluyó una búsqueda de
hiperparámetros sobre 16 combinaciones distintas, realizada automáticamente por Optuna. El
resultado en cada simulación corresponde a la red entrenada con la mejor combinación de
hiperparámetros hallada para esa simulación.

Mostramos los promedios del puntaje \(F_1\), la precisión y la cobertura en el conjunto
de test a lo largo de las 100 simulaciones. Cabe aclarar que, siguiendo lo explicado en la
Sección \ref{sec:comparacion}, dada una métrica determinada, el resultado obtenido en una
simulación es el promedio de esa métrica entre las cohortes. Además, para las redes
neuronales, de cada hiperparámetro, mostramos el valor que fue seleccionado como el óptimo
con mayor frecuencia.

Dado que nuestra métrica principal es el puntaje \(F_1\), para verificar que las
diferencias observadas en dicha métrica entre cada arquitectura y el PSM son
estadísticamente signficativas, llevamos a cabo pruebas de hipótesis de diferencia de
medias. Los detalles y resultados de estas pruebas se encuentran en el
\hyperref[chap:anexo]{Anexo}. En todos los casos, rechazamos la hipótesis nula que
establece que las medias poblaciones son iguales, lo cual implica que las diferencias
observadas son estadísticamente signficativas.

% Como mencionamos previamente, denotamos con:
% \begin{itemize}[itemsep=0.1cm]
%     \item \(L\) a la cantidad de períodos observados antes del inicio del programa.
%     \item \(n_{pd}\) a la cantidad de períodos de dependencia, con \(0 \le n_{pd} < L\).
%     \item \(m\) a la cantidad de veces, dentro de los períodos de dependencia, en los que
%     permitimos que el valor de la variable sea mayor al inmediato anterior.
%     \item \(\mu_{{EF}_{NiNi}}\) a la media de los efectos fijos para los individuos
%     NiNi.
% \end{itemize}
\end{document}