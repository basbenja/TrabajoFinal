\documentclass[../../main.tex]{subfiles}

\begin{document}

\section{Experimento 1: 45 períodos observados, 20 de dependencia}
En este primer escenario, tomamos 45 períodos observados previos al inicio del programa
para cada individuo con 20 períodos de dependencia, en donde permitimos un máximo de 6
``subidas''. Así, la proporción períodos de dependencia sobre períodos observados es de
0.44, y de subidas sobre períodos de dependencia de 0.3. Al tener 45 períodos observados y
3 cohortes, los inicios de programa son 46, 47 y 48.

% Todos los parámetros utilizados se encuentran en la Tabla \ref{tab:params_exp1}
% \begin{table}[H]
%     \centering
%     \begin{tabular}{|c|c|c|}
%         \hline
%         \(L\) & \(n_{pd}\) & \(m\)  \\ \hline\hline
%         45 & 20 & 6 \\
%         \hline
%     \end{tabular}
%     \caption{Parámetros para la generación de datos utilizados en el Experimento 1.}
%     \label{tab:params_exp1}
% \end{table}

Consideramos que este escenario es el que más ``favorece'' a las redes ya que es en el que
mayor cantidad de períodos observados y de períodos de dependencia hay, por lo que lo
tomaremos como punto de referencia para comparar lo ocurrido en los otros. La Figura
\ref{fig:time_series_exp1} muestra algunos ejemplos de las series de tiempo generadas para
cada uno de los grupos, en donde se ve claramente la tendencia que estamos tratando de
capturar en tratados y controles. En ellos, en los últimos 20 de los 45 períodos
considerados se observa un comportamiento ``decreciente con ruido'', que en los NiNi no
está presente.

\begin{figure}[H]
    \centering
    \begin{minipage}{0.48\textwidth}
        \centering
        \includegraphics[scale=0.3]{figs/Exp1/plot_time_series_Tratados.png}
    \end{minipage}
    \hfill
    \begin{minipage}{0.48\textwidth}
        \centering
        \includegraphics[scale=0.3]{figs/Exp1/plot_time_series_de_Control.png}
    \end{minipage}
    \vspace{0.5em}
    \begin{minipage}{0.6\textwidth}
        \centering
        \includegraphics[scale=0.3]{figs/Exp1/plot_time_series_Ninis.png}
    \end{minipage}
    \caption{Ejemplos de series de tiempo generadas para cada grupo en el Experimento 1.
    Los períodos con un fondo rojo indican los períodos de dependencia para tratados y
    controles, en donde se modela la tendencia ``decreciente con ruido''. La línea
    punteada roja indica el inicio de tratamiento para cada individuo.}
    \label{fig:time_series_exp1}
\end{figure}


La Tabla \ref{tab:results_exp1} muestra los resultados obtenidos en las métricas \(F_1\),
precisión y sensibilidad por las diferentes redes y el PSM. Cada valor representa el
promedio de esa métrica en las 100 simulaciones en el conjunto de test.

\begin{table}[H]
    \centering
    \renewcommand{\arraystretch}{1.2}
    \begin{tabular}{|c|c|c|c|}
        \hline
         & \textbf{Puntaje} \(F_1\) & \textbf{Precisión} & \textbf{Sensibilidad} \\ \hline\hline
        \textbf{LSTM}
            & 0.809842 & 0.724083 & 0.922185 \\ \hline
        \textbf{Convolucional}
            & \textbf{0.839449} & \textbf{0.762929} & \textbf{0.934678} \\ \hline
        \textbf{LSTM + Convolucional}
            & 0.836937 & 0.760843 & 0.931613 \\ \hline
        \textbf{PSM}
            & 0.658819 & 0.65998 & 0.65767 \\
        \hline
    \end{tabular}
    \caption{Promedio de las métricas \(F_1\), precisión y sensibilidad sobre la
    clase positiva (controles) en el conjunto de test en las 100 simulaciones del
    Experimento 1.}
    \label{tab:results_exp1}
\end{table}

En primer lugar, se puede ver que el desempeño de todas las redes fue superior al del PSM,
y que los resultados obtenidos con ellas son satisfactorios. En segundo lugar, el
desempeño de las redes fue parecido entre ellas; más aún entre la Convolucional y la LSTM
+ Convolucional, siendo la primera apenas mejor. Algo que también se puede observar es que
la sensibilidad obtenida con todas las redes es bastante alta, lo cual indica que la
mayoría de los controles fueron identificados correctamente por la red; mientras que la
precisión fue consistentemente menor, aunque con valores aceptables.

La Tabla \ref{tab:hyperparams_exp1} muestra, para cada arquitectura, el valor de cada
hiperparámetro que fue seleccionado con mayor frecuencia como el óptimo duarante la
búsqueda en las 100 simulaciones. Consideramos que de utilizar alguna de las
arquitecturas, estos valores deberían ser los empleados para entrenarla.

\begin{table}[H]
    \centering
    \renewcommand{\arraystretch}{1.2}
    \begin{tabular}{|c|c|c|c|c|}
        \hline
            & \makecell{\textbf{Tamaño}\\\textbf{de lote}}
            & \makecell{\textbf{Neuronas en}\\\textbf{capas ocultas}}
            & \makecell{\textbf{Tasa de}\\\textbf{aprendizaje}}
            & \textbf{Dropout} \\ \hline\hline
        \textbf{LSTM}
            & 128 (44\%) & 128 (69\%) & 0.001 (98\%) & 0.3 (54\%) \\ \hline
        \textbf{Convolucional}
            & 32 (46\%)  & -          & 0.001 (90\%) & 0.3 (66\%) \\ \hline
        \makecell{\textbf{LSTM +}\\\textbf{Convolucional}}
            & 32 (37\%)  & 64 (37\%)  & 0.001 (86\%) & 0.3 (72\%) \\
        \hline
    \end{tabular}
    \caption{Valores de hiperparámetros seleccionados con mayor frecuencia en las 100
    simulaciones en cada arquitectura. Cada celda contiene dicho valor y entre paréntesis
    el porcentaje de simulaciones en el resultó ser el mejor, de acuerdo a la optimización
    realizada por Optuna mediante validación cruzada.}
    \label{tab:hyperparams_exp1}
\end{table}

Se puede observar a simple vista que para todas las arquitecturas, la tasa de aprendizaje
de 0.001 resultó ser ampliamente la mejor y que en la mayoría de los casos se eligió un
dropout de 0.3, el cual era el menor en el espacio de búsqueda establecido. Otro detalle
es que el tamaño de lote parece no haber sido tan relevante, ya que en cada modelo, el
valor más seleccionado lo fue en menos de la mitad de las simulaciones. Algo que resulta
interesante es que en la LSTM ``sola'', el número de neuronas por capa más elegido fue
128, el máximo en el espacio de búsqueda, mientras que cuando se la combinó con la
Convolucional, este número se redujo a la mitad.

\end{document}