\documentclass[../../main.tex]{subfiles}

\begin{document}

\section{Experimento 5: 35 períodos observados, 15 de dependencia} \label{sec:exp5}
En este experimento, tomamos 35 períodos observados, de los cuales 15 son de dependencia
con un máximo de 4 incrementos. Así, el cociente entre períodos de dependencia sobre
observados es de 0.43, y el de subidas sobre períodos de dependencia de 0.27, ambos muy
parecidos a los del \hyperref[sec:exp1]{wxperimento 1}. Tomamos para los NiNi el mismo
valor de media de efectos fijos de tratados y controles (\(\mu_{{EF}_{NiNi}} = 10\)).

En otras palabras, lo que ocurre es que se reduce la cantidad total de datos temporales
pero se mantienen las proporciones previas. La idea de este escenario - y los siguientes -
es evaluar el desempeño de los distintos modelos en casos en donde la información
disponible sobre los períodos previos al tratamiento es más limitada.

Los resultados obtenidos, resumidos en la Tabla \ref{tab:results_exp5}, muestran solamente
un leve deterioro de todos los modelos en comparación con \hyperref[tab:results_exp1]{los
del experimento 1}. Consideramos que esto resulta bastante alentador, ya que sugiere que
los modelos son robustos frente a una reducción en la cantidad de datos temporales
pre-tratamiento.

\begin{table}[H]
    \centering
    \renewcommand{\arraystretch}{1.2}
    \begin{tabular}{|c|c|c|c|}
        \hline
         & \textbf{Puntaje} \(F_1\) & \textbf{Precisión} & \textbf{Cobertura} \\ \hline\hline
        \textbf{LSTM}
            & $0.78815 \pm 0.02252$ & $0.69409 \pm 0.03919$ & $0.91444 \pm 0.01953$ \\ \hline
        \textbf{Convolucional}
            & $\mathbf{0.81989 \pm 0.02092}$ & $0.73371 \pm 0.03725$ & $\mathbf{0.93110 \pm 0.01520}$ \\ \hline
        \makecell{\textbf{LSTM +} \\ \textbf{Convolucional}}
            & $0.81942 \pm 0.01833$ & $\mathbf{0.73744 \pm 0.03291}$ & $0.92359 \pm 0.01573$ \\ \hline
        \textbf{PSM}
            & $0.61062 \pm 0.01661$ & $0.61155 \pm 0.01669$ & $0.60970 \pm 0.01656$ \\
        \hline
    \end{tabular}
    \caption{Promedio de las métricas \(F_1\), precisión y cobertura sobre la
    clase positiva (controles) en el conjunto de test en las 100 simulaciones del
    experimento 5.}
    \label{tab:results_exp5}
\end{table}

Los valores de los hiperparámetros, que se reflejan en la Tabla
\ref{tab:hyperparams_exp5}, siguen mostrando los mismos patrones que antes. La tasa de
aprendizaje más elegida fue 0.001, el dropout 0.3, los valores más elegidos de tamaños de
lote lo fueron en menos de la mitad de las simulaciones, y las neuronas ocultas del bloque
LSTM se reduce cuando se lo combina con el procesamiento convolucional.

\begin{table}[H]
    \centering
    \renewcommand{\arraystretch}{1.2}
    \begin{tabular}{|c|c|c|c|c|}
        \hline
            & \makecell{\textbf{Tamaño}\\\textbf{de lote}}
            & \makecell{\textbf{Neuronas en}\\\textbf{capas ocultas}}
            & \makecell{\textbf{Tasa de}\\\textbf{aprendizaje}}
            & \textbf{Dropout} \\ \hline\hline
        \textbf{LSTM}
            & 64 (39\%) & 128 (62\%) & 0.001 (99\%) & 0.3 (51\%) \\ \hline
        \textbf{Convolucional}
            & 64 (36\%) & -          & 0.001 (98\%) & 0.3 (77\%) \\ \hline
        \makecell{\textbf{LSTM +}\\\textbf{Convolucional}}
            & 32 (41\%) & 32 (41\%)  & 0.001 (95\%) & 0.3 (78\%) \\
        \hline
    \end{tabular}
    \caption{Valores de hiperparámetros seleccionados con mayor frecuencia en las 100
    simulaciones en cada arquitectura en el experimento 5. El porcentaje entre paréntesis
    indica el porcentaje de veces que fue elegido ese valor en las 100 simulaciones.}
    \label{tab:hyperparams_exp5}
\end{table}

\end{document}