\documentclass[../../main.tex]{subfiles}

\begin{document}

\section{Experimento 5: 25 períodos observados, 10 de dependencia}
Proporción entre períodos de dependencia y observados: 0.4.

\begin{table}[H]
    \centering
    \renewcommand{\arraystretch}{1.2}
    \begin{tabular}{|c|c|c|c|}
        \hline
         & \textbf{Puntaje} \(F_1\) & \textbf{Precisión} & \textbf{Sensibilidad} \\ \hline\hline
        \textbf{LSTM}
            & 0.571055 & 0.446358 & 0.796076 \\ \hline
        \textbf{Convolucional}
            & \textbf{0.626289} & \textbf{0.507315} & \textbf{0.821558} \\ \hline
        \textbf{LSTM + Convolucional}
            & 0.618374 & 0.503425 & 0.806757 \\ \hline
        \textbf{PSM}
            & 0.364804 & 0.365307 & 0.364306 \\
        \hline
    \end{tabular}
    \caption{Promedio de las métricas \(F_1\), precisión y sensibilidad sobre la
    clase positiva (controles) en el conjunto de test en las 100 simulaciones del
    Experimento 5.}
    \label{tab:results_exp5}
\end{table}

\begin{table}[H]
    \centering
    \renewcommand{\arraystretch}{1.2}
    \begin{tabular}{|c|c|c|c|c|}
        \hline
            & \makecell{\textbf{Tamaño}\\\textbf{de lote}}
            & \makecell{\textbf{Neuronas en}\\\textbf{capas ocultas}}
            & \makecell{\textbf{Tasa de}\\\textbf{aprendizaje}}
            & \textbf{Dropout} \\ \hline\hline
        \textbf{LSTM}
            & 128 (38\%) & 128 (77\%) & 0.001 (98\%) & 0.5 (43\%) \\ \hline
        \textbf{Convolucional}
            & 64 (35\%) & -           & 0.001 (93\%) & 0.3 (78\%) \\ \hline
        \makecell{\textbf{LSTM +}\\\textbf{Convolucional}}
            & 64 (39\%) & 128 (38\%)  & 0.001 (95\%) & 0.3 (86\%) \\
        \hline
    \end{tabular}
    \caption{Valores de hiperparámetros seleccionados con mayor frecuencia en las 100
    simulaciones en cada arquitectura. El porcentaje entre paréntesis indica el porcentaje
    de veces que fue elegido ese valor en las 100 simulaciones.}
    \label{tab:hyperparams_exp5}
\end{table}

\end{document}