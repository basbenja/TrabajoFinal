\documentclass[../../main.tex]{subfiles}

\begin{document}

\section{Experimento 7: 15 períodos observados, 8 de dependencia} \label{sec:exp7}
En este último experimento, tomamos 15 períodos de observación, de los cuales 8 son de
dependencia con una cantidad máxima de 2 incrementos. De esta manera, la proporción de
períodos de dependencia y observados es de 0.53, un valor mayor al de los experimentos
anteriores, y de aumentos sobre períodos de dependencia de 0.25, menor al de los otros
escenarios. De nuevo, el valor de \(\mu_{{EF}_{NiNi}}\) es el mismo que para tratados
y controles (10).

Los resultados, que se reflejan en la Tabla \ref{tab:results_exp7}, muestran una leve
mejoría con respecto a los del \hyperref[sec:exp6]{experimento 6}. Esto puede deberse en
gran parte a que, si bien hay una menor cantidad de datos, un poco más de la mitad de los
períodos observados son de dependencia. Sin embargo, el caso de la precisión sigue siendo
preocupante.

\begin{table}[H]
    \centering
    \renewcommand{\arraystretch}{1.2}
    \label{tab:results_exp7}
    \begin{tabular}{|c|c|c|c|}
        \hline
         & \textbf{Puntaje} \(F_1\) & \textbf{Precisión} & \textbf{Cobertura} \\ \hline\hline
        \textbf{LSTM}
            & $0.61563 \pm 0.01930$ & $0.49560 \pm 0.02659$ & $0.81554 \pm 0.03051$ \\ \hline
        \textbf{Convolucional}
            & $\mathbf{0.67466 \pm 0.02309}$ & $\mathbf{0.54877 \pm 0.03250}$ & $\mathbf{0.87960 \pm 0.03372}$ \\ \hline
        \makecell{\textbf{LSTM +} \\ \textbf{Convolucional}}
            & $0.66751 \pm 0.02285$ & $0.54698 \pm 0.03536$ & $0.86140 \pm 0.03752$ \\ \hline
        \textbf{PSM}
            & $0.37886 \pm 0.01287$ & $0.37936 \pm 0.01294$ & $0.37837 \pm 0.01282$ \\
        \hline
    \end{tabular}
    \caption{Promedio de las métricas \(F_1\), precisión y cobertura sobre la
    clase positiva (controles) en el conjunto de test en las 100 simulaciones del
    experimento 7.}
\end{table}

Los hiperparámetros más seleccionados como los óptimos se encuentran en la Tabla
\ref{tab:hyperparams_exp7}, donde se ve que los valores más elegidos resultaron ser los
mismos para todas las arquitecturas. Además, con el número de neuronas de las capas
ocultas de la arquitectura LSTM + Convolucional ocurrió lo mismo que en el experimento
anterior, el valor más seleccionado fue el más alto en el espacio de búsqueda.

\begin{table}[H]
    \centering
    \renewcommand{\arraystretch}{1.2}
    \label{tab:hyperparams_exp7}
    \begin{tabular}{|c|c|c|c|c|}
        \hline
            & \makecell{\textbf{Tamaño}\\\textbf{de lote}}
            & \makecell{\textbf{Neuronas en}\\\textbf{capas ocultas}}
            & \makecell{\textbf{Tasa de}\\\textbf{aprendizaje}}
            & \textbf{Dropout} \\ \hline\hline
        \textbf{LSTM}
            & 64 (39\%) & 128 (80\%) & 0.001 (99\%) & 0.3 (49\%) \\ \hline
        \textbf{Convolucional}
            & 64 (39\%) & -          & 0.001 (99\%) & 0.3 (79\%) \\ \hline
        \makecell{\textbf{LSTM +}\\\textbf{Convolucional}}
            & 64 (40\%) & 128 (38\%) & 0.001 (98\%) & 0.3 (81\%) \\
        \hline
    \end{tabular}
    \caption{Valores de hiperparámetros seleccionados con mayor frecuencia en las 100
    simulaciones en cada arquitectura. El porcentaje entre paréntesis indica el porcentaje
    de veces que fue elegido ese valor en las 100 simulaciones.}
\end{table}

\end{document}