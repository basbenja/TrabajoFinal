\documentclass[../../main.tex]{subfiles}

\begin{document}

\section{Experimento 4: 35 períodos observados, 15 de dependencia}
En este Experimento, tomamos 35 períodos observados, de los cuales 15 son de dependencia
con un máximo de 4 subidas. Así, el ratio de períodos de dependencia sobre observados es
de 0.43, y el de subidas sobre períodos de dependencia de 0.27, ambos muy parecidos a los
del Experimento 1. La idea de este escenario y los siguientes es evaluar el desempeño de
los distintos modelos en casos en donde la información disponible es sobre una cantidad
más limitada de períodos pre-tratamiento.

Los resultados obtenidos, resumidos en la Tabla \ref{tab:results_exp4}, muestran solamente
un muy leve deterioro de todos los modelos en comparación los del Experimento 1, lo cual
es bastante alentador.

\begin{table}[H]
    \centering
    \renewcommand{\arraystretch}{1.2}
    \begin{tabular}{|c|c|c|c|}
        \hline
         & \textbf{Puntaje} \(F_1\) & \textbf{Precisión} & \textbf{Sensibilidad} \\ \hline\hline
        \textbf{LSTM}
            & 0.788148 & 0.69409 & 0.914435 \\ \hline
        \textbf{Convolucional}
            & \textbf{0.819889} & 0.733707 & \textbf{0.9311} \\ \hline
        \textbf{LSTM + Convolucional}
            & 0.819418 & \textbf{0.737445} & 0.92359 \\ \hline
        \textbf{PSM}
            & 0.610622 & 0.611555 & 0.609697 \\
        \hline
    \end{tabular}
    \caption{Promedio de las métricas \(F_1\), precisión y sensibilidad sobre la
    clase positiva (controles) en el conjunto de test en las 100 simulaciones del
    Experimento 4.}
    \label{tab:results_exp4}
\end{table}

\begin{table}[H]
    \centering
    \renewcommand{\arraystretch}{1.2}
    \begin{tabular}{|c|c|c|c|c|}
        \hline
            & \makecell{\textbf{Tamaño}\\\textbf{de lote}}
            & \makecell{\textbf{Neuronas en}\\\textbf{capas ocultas}}
            & \makecell{\textbf{Tasa de}\\\textbf{aprendizaje}}
            & \textbf{Dropout} \\ \hline\hline
        \textbf{LSTM}
            & 64 (39\%) & 128 (62\%) & 0.001 (99\%) & 0.3 (51\%) \\ \hline
        \textbf{Convolucional}
            & 64 (36\%) & -          & 0.001 (98\%) & 0.3 (77\%) \\ \hline
        \makecell{\textbf{LSTM +}\\\textbf{Convolucional}}
            & 32 (41\%) & 32 (41\%)  & 0.001 (95\%) & 0.3 (78\%) \\
        \hline
    \end{tabular}
    \caption{Valores de hiperparámetros seleccionados con mayor frecuencia en las 100
    simulaciones en cada arquitectura. El porcentaje entre paréntesis indica el porcentaje
    de veces que fue elegido ese valor en las 100 simulaciones.}
    \label{tab:hyperparams_exp4}
\end{table}

\end{document}