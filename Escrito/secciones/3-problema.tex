\documentclass[../main.tex]{subfiles}
% \graphicspath{{\subfix{../images/}}}

\begin{document}
Como explicamos anteriormente, en cualquier evaluación de impacto, para poder calcular el
efecto de un programa, se debe obtener una estimación apropiada del contrafactual: ¿qué
hubiera pasado con la variable de resultado de los beneficiarios del programa si el mismo
no hubiera existido? Vimos también que para lograrlo, se construye un grupo de control,
formado por individuos que no han sido tratados pero que idealmente, son estadísticamente
similares a los que sí lo fueron.

En esto, aparece el problema del sesgo de autoselección, que consiste en no tomar en
cuenta las diferencias preexistentes entre tratados y no tratados, que pueden haber
afectado tanto a la decisión de participar en el programa como a los resultados
potenciales posteriores.

En evaluaciones sobre programas con asignación aleatoria, el grupo de control se obtiene
directamente tomando todos los individuos que no fueron tratados y el problema de
autoselección se soluciona naturalmente. Sin embargo, en las evaluaciones
cuasi-experimentales, hechas sobre tratamientos en donde la asignación al tratamiento no
fue aleatoria y se desconocen los motivos que han llevado a los individuos a inscribirse o
recibir el tratamiento, construir un grupo de control adecuado constituye un gran desafío.
Este es uno de los temas centrales de este trabajo.

Una técnica estadística para abordar este problema es el PSM, discutida anteriormente y
que es la que tomamos como punto de partida aquí. Este método empareja individuos tratados
con no tratados basándose en su probabilidad estimada de participación, calculada a partir
de un conjunto de variables que se consideran influyentes en el programa bajo cuestión.

Estas variables son seleccionadas por los evaluadores y pueden representar tanto
características observadas en un instante determinado como también compartamientos a lo
largo del tiempo. Por ejemplo, si consideramos un programa de financiamento a empresas, se
puede tomar como variable el número de empleados y los ingresos mensuales de solamente el
mes anterior al inicio del programa, o se podría tomar esta cantidad observada durante
diez meses previos al comienzo, siempre y cuando el investigador considere que es
relevante hacerlo. De esta forma, es posible incluir series de tiempo en el cálculo del
puntaje de propensión, tomando como variables varios períodos de la misma característica.
Esto permite capturar no solo un estado puntual de las unidades sino también cómo ha sido
su evolución, lo cual puede contribuir a un mejor emparejamiento y consecuentemente, a una
estimación más precisa del efecto del tratamiento.

Otra cuestión a tener en cuenta a la hora de identificar los individuos de control es la
forma de implementación del programa con respecto al tiempo. Hay unos en los que el
ingreso al tratamiento por parte de los individuos se realiza en un único instante de
tiempo, pero en otros la entrada al programa ocurre de manera secuencial. En este último
caso, se habla de una \textbf{cohorte} o una \textbf{camada} para referirse al conjunto de
individuos que ingresaron al programa en el mismo momento.

Habiendo presentado estas cuestiones, nuestro trabajo se enfoca en programas con entrada
secuencial y en los que supondremos que los individuos tratados resultaron ser tratados (o
decidieron inscribirse al programa) por la dinámica temporal observada en una
característica en períodos anteriores a la asignación al programa. Bajo estas
circunstancias, el PSM presenta ciertas limitaciones:
\begin{itemize}
    \item Por un lado, cuando existen múltiples cohortes, la forma en la que se aplica la
    técnica es por cohorte. Es decir, lo que se hace es estimar la probabilidad de ingreso
    al programa por camada, y se buscan controles separadamente para cada una de estas
    camadas. Dicho de otra manera, el proceso de emparejamiento se realiza tantas veces
    como cohortes haya.
    \item Por otro lado, cuando las variables tenidas en cuenta en realidad representan
    una misma característica pero en distintos períodos de tiempo, la regresión logística,
    que es la forma en la que se calcula el puntaje de propensión, no es capaz de capturar
    esta relación temporal entre ellas, sino que las ve como si fueran independientes.
\end{itemize}

Con estos problemas en mente, el foco de nuestro trabajo está en explorar una alternativa
para la identificación de grupos de control bajo las hipótesis mencionadas previamente.
Esta se basa en la utilización de diferentes tipos de redes neuronales, concretamente:
redes convolucionales, redes LSTM, y la combinación de ellas. Como explicamos en el
capítulo anterior, estas incorporan naturalmente la relación temporal de los datos,
permitiendo capturar patrones a lo largo del tiempo. En nuestro escenario, en donde
suponemos que la dinámica previa a la intervención es un factor clave para entender la
participación en el programa, consideramos que estas redes pueden resultar especialmente
útiles.

Para evaluar la estrategia presentada, desarrollamos diferentes conjuntos de datos
sintéticos de tipo panel\footnotemark, ya que en los casos reales algunos comportamientos
no pueden verificarse directamente. Aquí, algunos de los parámetros a variar son la
cantidad de períodos observados (previos al inicio del programa), la cantidad de períodos
de dependencia temporal, y la dinámica temporal observada. Teniendo esto en cuenta, los
objetivos de nuestro trabajo se pueden resumir en los siguientes puntos: \footnotetext{Los
datos de tipo panel son aquellos en los que a cada unidad considerada le corresponde una
serie de tiempo.}
\begin{itemize}
    \item Evaluar la capacidad de las redes neuronales de reconocer individuos de control
    ante diferentes dinámicas temporales de la variable observada.
    \item Obtener resultados que permitan comparar el desempeño de las redes con
    el del PSM en diferentes escenarios.
    \item Superar las limitaciones observadas del PSM utilizando las redes neuronales.
    Esto es: incorporar de manera más efectiva la dependencia temporal de las covariables
    y poder reconocer los controles por cohorte en un solo proceso, sin la necesidad de
    tener que repetir la búsqueda por cohorte.
\end{itemize}

Nuestras hipótesis son las siguientes:
\begin{itemize}
    \item Cuando no hay una dependencia temporal marcada, es decir no hay un
    comportamiento determinado en la evolución de la variable observada, el PSM y las
    redes tienen aproximadamente el mismo desempeño.
    \item Por el contario, cuando existe una alta dependencia temporal, o sea cuando en
    los datos forzamos un determinado comportamiento temporal, las redes funcionan mejor
    que el PSM.
    \item Las redes LSTM tienen un mejor desmpeño que las convolucionales, y la combinación
    de las dos es mejor que ambas.
    \item Ante menor cantidad de períodos de observación, las redes cometen cada
    vez más errores.
\end{itemize}

A continuación, presentamos el marco sobre el cual llevamos a cabo los experimentos,
detallando el alcance de nuestro trabajo, la forma en que construimos las diferentes
simulaciones, las arquitecturas de redes, las métricas para evaluar los modelos,
y las herramientas que nos ayudaron en todo este proceso.

\end{document}