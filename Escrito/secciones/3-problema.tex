\documentclass[../main.tex]{subfiles}
% \graphicspath{{\subfix{../images/}}}

\begin{document}
Como explicamos anteriormente, en cualquier evaluación de impacto, para poder calcular el
efecto de un programa, se debe obtener una estimación apropiada del contrafactual: ¿qué
hubiera pasado con la variable de objetivo de los beneficiarios del programa si el mismo
no hubiera existido? Vimos también que para lograrlo, se construye un grupo de control,
formado por individuos que no han sido tratados pero que idealmente, son estadísticamente
similares a los que sí lo fueron.

En esto, aparece el problema del sesgo de autoselección, que consiste en no tomar en
cuenta las diferencias preexistentes entre tratados y no tratados, que pueden haber
afectado tanto a la decisión de participar en el programa como a los resultados
potenciales posteriores.

En evaluaciones sobre programas con asignación aleatoria, este grupo se obtiene
directamente tomando todos los individuos que no fueron tratados y el problema de
autoselección se soluciona naturalmente. Sin embargo, en las evaluaciones
cuasi-experimentales, hechas sobre tratamientos en donde la asignación al tratamiento no
fue aleatoria y se desconocen los motivos que han llevado a los individuos a inscribirse o
recibir el tratamiento, construir un grupo de control adecuado constituye un gran desafío.
Este es uno de los focos centrales de este trabajo.

Una técnica estadística muy potente para abordar este problema es el PSM. Esta técnica
empareja individuos tratados con no tratados basándose en su probabilidad estimada de
participación, calculada a partir de un conjunto de variables que se consideran
influyentes en el programa bajo cuestión.

Estas variables pueden representar tanto características observadas en un instante
determinado como también compartamientos a lo largo del tiempo. Por ejemplo, si
consideramos un tratamiento a empresas, se puede tomar como variable el número de
empleados solamente en el mes anterior al inicio del programa, o se podría tomar esta
cantidad observada durante diez meses previos al comienzo. De esta forma, es posible
incluir series de tiempo en el cálculo del puntaje de propensión, tomando como variables
varios períodos de la misma característica. Esto permite capturar no solo un estado
puntual de las unidades sino también como ha sido su evolución, lo cual puede contribuir
a un mejor emparejamiento y a una estimación más precisa. En este trabajo, nos enfocamos
particularmente en esta última elección de variables.

Otra cuestión a tener en cuenta a la hora de identificar los individuos de control es la
forma de implementación del programa con respecto al tiempo. Hay unos en los que el
ingreso al tratamiento por parte de los individuos se realiza en un único instante de
tiempo, pero en otros la entrada al programa ocurre de manera secuencial. En este último
caso, se habla de \textbf{cohorte} para referirse al conjunto de individuos que ingresaron
al programa en el mismo momento. Nosotros consideraremos programas con este tipo
de entradas.

% De hecho, cuando el ingreso
% al tratamiento por parte de los individuos se realiza en un único instante de tiempo, el
% grupo de control obtenido mediante su uso conforma el mejor estimador del contrafactual.
% No obstante, cuando la entrada al programa ocurre de manera secuencial, los modelos
% logísticos no son capaces de gestionar esa probabilidad variable en el tiempo.

% Objetivos


% Hipótesis

\begin{comment}
 Sin embargo, cuando la entrada al programa ocurre de
manera secuencial, los modelos logísticos no son capaces de gestionar esa probabilidad
variable en el tiempo. Lo que se hace entonces es estimar la probabilidad por camada de
ingreso al programa, es decir se agrupan individuos que hayan ingresado al tratamiento en
temporalidades similares, y se buscan controles separadamente para cada una de estas
camadas.

En este trabajo, exploramos una alternativa distinta para este último caso. Proponemos la
utilización de redes neuronales, específicamente del tipo LSTM \textbf{y
convolucionales????}, capaces de incorporar la dimensión temporal de los datos, para la
detección automática de grupos de control, en el caso particular en que el programa en
cuestión se implementa \textbf{secuencialmente} y existe una \textbf{alta dependencia
temporal} entre una variable observada de los individuos y el momento de ingreso al
tratamiento. El principal objetivo es evaluar el potencial y la efectividad de estas redes
para capturar dependencias dinámicas y características temporales inherentes a los datos
observados, para de esta forma mejorar el proceso de inferencia causal en la evaluación de
impacto.
\end{comment}

\end{document}