\documentclass[../main.tex]{subfiles}
% \graphicspath{{\subfix{../images/}}}

\begin{document}
% El problema que abordamos en este proyecto se enmarca dentro de lo que se conoce como \textbf{Clasificación de Series de Tiempo} (CST).

Como explicamos anteriormente, el principal problema en la Evaluación de Impacto de
tratamientos sin asignación aleatoria es el de la identificación de un grupo de control o
comparación adecuado, que cumpla las características descritas previamente.

Cuando el ingreso al tratamiento por parte de los individuos se realiza en un único
instante de tiempo, el grupo de control obtenido mediante la técnica de PSM conforma el
mejor estimador del contrafactual. Sin embargo, cuando la entrada al programa ocurre de
manera secuencial, los modelos logísticos no son capaces de gestionar esa probabilidad
variable en el tiempo. Lo que se hace entonces es estimar la probabilidad por camada de
ingreso al programa, es decir se agrupan individuos que hayan ingresado al tratamiento en
temporalidades similares, y se buscan controles separadamente para cada una de estas
camadas.

En este trabajo, exploramos una alternativa distinta para este último caso. Proponemos la
utilización de redes neuronales, específicamente del tipo LSTM \textbf{y
convolucionales????}, capaces de incorporar la dimensión temporal de los datos, para la
detección automática de grupos de control, en el caso particular en que el programa en
cuestión se implementa \textbf{secuencialmente} y existe una \textbf{alta dependencia
temporal} entre una variable observada de los individuos y el momento de ingreso al
tratamiento. El principal objetivo es evaluar el potencial y la efectividad de estas redes
para capturar dependencias dinámicas y características temporales inherentes a los datos
observados, para de esta forma mejorar el proceso de inferencia causal en la evaluación de
impacto.

% \textbf{Nuestra principal hipótesis es que ... ¿PONERLO?}.


\end{document}