\documentclass[../main.tex]{subfiles}

\begin{document}

Cuando se invierten recursos en un proyecto, suele haber un propósito claro por detrás y,
con el paso del tiempo, surge la necesidad de cuestionarse si lo invertido tuvo algún
efecto real sobre el objetivo que se estaba tratando de lograr. Por ejemplo, si uno decide
dedicarle más horas fuera de clase al estudio de una materia, es normal que nos surja la
pregunta de si realmente esas horas sirvieron. En el caso de una empresa, quizás se
espera que contratando más empleados, se aumenten la productividad y las ganancias. Las
políticas públicas tampoco son una excepción: tomemos el caso de un programa de subsidios
para la compra de alimentos en un determinado sector de la población; aquí resulta aún más
importante analizar si, luego de un tiempo, esto ayudó a mejorar los niveles de salud de
dicho sector.

En este contexto, cobra relevancia la metodología conocida como \textbf{evaluación de
impacto}. La evaluación de impacto es una técnica que permite estimar cuantitativamente
los efectos - positivos o negativos - que ha tenido la implementación de un programa,
proyecto o tratamiento sobre determinadas variables de interés - las variables objetivo -
de la población receptora del mismo. La forma en la que esto se logra es a través de un
\textbf{análisis contrafactual}: ¿qué hubiera pasado con las variables de la población
receptora - los ``tratados'' - si el programa no hubiera existido?

La situación anterior, que recibe comúnmente el nombre de ``el contrafactual'', es sin
dudas un escenario hipotético, no observable en la realidad; y es por esto que es
necesario aproximarlo de alguna forma. Para ello, una técnica consiste en construir un
\textbf{grupo de control} o \textbf{comparación}. Este debería estar formado por unidades
que no han sido beneficiarias del programa pero que, idealmente, antes del inicio del
mismo, eran muy similares a quienes sí lo fueron. Con el grupo de control identificado, se
puede estimar el \textbf{efecto promedio del programa en los tratados} (\(ATT\))
comparando los resultados de ambos grupos.

Dicho esto, el principal desafío en una evaluación de impacto es construir un grupo de
control \textit{válido}. Esto es, uno que permita asegurar que la diferencia observada en
los resultados se debe pura y exclusivamente al tratamiento en cuestión, y no a otros
factores. Si la asignación al programa es aleatoria, este grupo es todo el conjunto de los
no tratados (dentro de la población elegible). Sin embargo, en políticas donde la
asignación no tiene reglas claras, y se desconocen cuáles son las verdaderas razones que
han llevado a las unidades a participar o no del programa, tomar a los no tratados como
grupo de comparación puede producir estimaciones sesgadas, resultantes de no haber
considerado las diferencias preexistentes entre ambos grupos. Este problema es conocido
como \textbf{sesgo de autoselección}.

Existen diferentes métodos que buscan mitigar este sesgo, siendo una bastante utilizada
el \textbf{pareamiento por puntaje de propensión} (PSM) \cite{psm1983}. El PSM consiste en
estimar el puntaje de propensión - también llamado probabilidad de participación - tanto
para tratados y no tratados, y construir un grupo de control emparejando a cada tratado
con uno o varios individuos no tratados que tengan puntajes similares. Para el cálculo de
esta probabilidad, los evaluadores deben seleccionar ciertas variables que consideren
determinantes en la decisión de participar.

Ahora bien, cuando se cree que esta decisión puede estar influida por comportamientos
observados a lo largo del tiempo en períodos previos a la implementación del programa, los
métodos estadísticos utilizados para calcular el puntaje de propensión no son los más
efectivos para capturar posibles patrones y dinámicas, y hacer un emparejamiento adecuado.

Es en estos escenarios donde consideramos que los modelos de \textbf{aprendizaje
automático} pueden contribuir a una mejor selección de unidades pertenecientes al grupo de
control. El aprendizaje automático es un campo de la inteligencia artificial cuyo enfoque
está en desarrollar técnicas que permitan a las computadoras aprender automáticamente a
partir de los datos, sin la necesidad de tener que proveerlas de reglas. Esta estrategia
facilita la identificación de patrones, incluso cuando pueden parecer muy difíciles de
identificar a simple vista.

Un modelo particular que ha tenido un gran desarrollo en los últimos años es el conocido
como \textbf{redes neuronales artificiales}, cuya inspiración estuvo en el intento de
modelar el aprendizaje en el cerebro humano. Dentro de estas, existen algunos tipos
específicos adecuados para el procesamiento de datos temporales, como las \textbf{redes
neuronales recurrentes} y las \textbf{redes neuronales convolucionales}.

En este trabajo, proponemos la utilización de estos dos tipos de redes neuronales para la
identificación de grupos de control válidos en contextos donde la asignación al programa
es no aleatoria, potencialmente dependiente de resultados pasados, y cuya implementación
se desarrolla en múltiples períodos de tiempo. Para la evaluación de nuestro enfoque,
generamos datos sintéticos diseñados para imitar situaciones reales, y comparamos los
diferentes resultados con los obtenidos con el PSM. De esta manera, buscamos aportar una
herramienta alternativa para mejorar la calidad de las evaluaciones de impacto en este
tipo de programas.


% Poner esto despues de la primera oracion del abstract?
% Su aplicación es particularmente importante en el marco de políticas públicas ya que
% permite aumentar la efectividad de las mismas, invertir mejor los recursos y rendir
% cuentas a la ciudadanía.


\section{Estructura del trabajo}
En el Capítulo 2, se presentan los conceptos fundamentales necesarios para comprender el
desarrollo de este trabajo. En particular, explicamos en profundidad la metodología de la
evaluación de impacto, los problemas que aparecen y algunas de las técnicas utilizadas.
Luego, nos enfocamos en el aprendizaje automático, centrándonos en las redes neuronales, y
más particularmente en las de tipo LSTM y convoluciones.

Posteriormente, en el Capítulo 3, se describen algunas de las limitaciones que presenta el
PSM en los escenarios estudiados en este trabajo y argumentamos por qué consideramos que
el enfoque propuesto debería de superarlas.

En el Capítulo 4, se detalla el marco experimental utilizado. Se mencionan todas las
decisiones tomadas en las diferentes etapas, desde la generación de datos sintéticos, la
búsqueda de hiperparámetros, las métricas de importancia y la forma de comparar el
desempeño de las redes neuronales con el del PSM.

En el Capítulo 5, se muestran los resultados obtenidos en los diferentes experimentos
llevados a cabo, haciendo especial énfasis en el puntaje \(F_1\), la precisión y la
recuperación, y en los valores de los hiperparámetros hallados en las diferentes
simulaciones.

Finalmente, en el Capítulo 6, se presentan las conclusiones derivadas a partir de los
resultados, se propone una manera de aplicar nuestra alternativa en casos reales, y se
introducen algunos aspectos de nuestro desarrollo que podrían mejorarse en trabajos
futuros.

\end{document}