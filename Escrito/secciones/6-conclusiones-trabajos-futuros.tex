\documentclass[../main.tex]{subfiles}

\begin{document}

En este trabajo se exploró la utilización de redes neuronales para la selección de grupos
de control en evaluaciones de impacto sobre programas cuya asignación de beneficiarios no
ha sido aleatoria, dependió potencialmente de resultados pasados y en los que hubo
múltiples cohortes. Se evaluaron diferentes arquitecturas de redes neuronales y se comparó
su desempeño con el del PSM, un método bastante utilizado para esta tarea. A continuación,
desarrollamos las principales conclusiones obtenidas a partir de los resultados,
proponemos una forma de aplicar los métodos explorados en la realidad, y mencionamos
algunos aspectos de nuestro enfoque que podrían investigarse en trabajos futuros.

\section{Conclusiones generales}
% Desempeño de las redes VS Desempeño del PSM
Para empezar, en todos los escenarios se vio que con las redes neuronales se obtuvieron
mejores resultados que con el PSM. Utilizamos tests de hipótesis que nos permitieron
concluir que las diferencias observadas en las métricas fueron estadísticamente
significativas.

% Comparar las diferentes arquitecturas
Con respecto a la comparación entre las diferentes arquitecturas, contrario a nuestro
pensamiento inicial, con la que mejores resultados obtuvimos en la gran mayoría de los
escenarios fue la Convolucional. Para nuestra sorpresa, la combinación de LSTM +
Convolucional no trajo mejoras sobre la Convolucional, aunque sí sobre la LSTM. Esto
sugiere que en general, los operaciones aplicadas por las redes convolucionales
permitieron una mejor captura de los patrones propuestos en los diferentes escenarios, y
que la incorporación de un bloque LSTM no tiene un efecto significativo (al menos con la
arquitectura específica y el espacio de búsqueda de hiperparámetros propuestos).

% Hiperparametros mas utilizados en las 100 simulaciones
Como presentamos en las diferentes tablas de hiperparámetros, con respecto a los de
aprendizaje, la tendencia general que observamos a lo largo de los diferentes escenarios
fue que la tasa de aprendizaje más seleccionada fue de 0.001, el dropout de 0.3, y que los
valores más elegidos de tamaños de lote lo fueron en menos de la mitad de las
simulaciones. Algo a notar es que el valor de 0.3 de dropout fue el menor dentro del
espacio de búsqueda, lo cual puede indicar que con valores menores, se obtendrían aún
mejores resultados.

Por otro lado, si prestamos atención al número de neuronas en las capas LSTM, lo que vimos
en la mayoría de los casos fue que el valor más seleccionado en la arquitectura LSTM fue
de 128, pero que en la LSTM + Convolucional, este valor se reducía. Esto refuerza la
hipótesis que el procesamiento hecho por las convoluciones es más efectivo que el hecho
por las neuronas LSTM. Sin embargo, como en el caso de la arquitectura LSTM el valor más
seleccionado (128) era el más alto en el espacio de búsqueda, se podría evaluar el uso de
valores más altos.

% Hablar de la relación resultados - períodos - tendencia - proporciones
Considerando las características de los diferentes escenarios, podemos concluir que las
redes demostraron un rendimiento prometedor en aquellos donde había una alta cantidad de
períodos observados y las tendencias eran largas y con ``ruido'' (experimentos
\hyperref[sec:exp1]{1} y \hyperref[sec:exp5]{5}), o cuando eran cortas y con un
comportamiento monótono (\hyperref[sec:exp3]{experimento 3}). En escenarios donde no hay
un patrón claro en los comportamientos pasados, sino que es una característica más
``escondida'' la que distingue a tratados y controles de NiNi
(\hyperref[sec:exp4]{experimento 4}), ninguno de los modelos mostró resultados aceptables.
Además, prestando atención particularmente a los valores obtenidos en los experimentos
\hyperref[sec:exp6]{6} y \hyperref[sec:exp7]{7}, donde la cantidad de períodos observados
era baja (25 y 15 respectivamente), apuntan que ante una mayor proporción de períodos de
dependencia sobre observados, el desempeño de las redes mejora.

\section{Aplicación en casos reales}
En nuestros experimentos, nosotros sabemos quiénes son los individuos que deberían ser
identificados como controles y quiénes como NiNi. Sin embargo, en una situación real, a la
hora de llevar a cabo una evaluación de impacto, solamente se conocen quiénes han sido los
tratados y quiénes los no tratados; y dentro de este último conjunto, están mezclados NiNi
e individuos que potencialmente podrían servir como controles. Entonces, nos surgió la
pregunta de cómo podríamos hacer para aplicar la metodología presentada aquí en un entorno
real. A continuación, detallamos nuestra propuesta.

Para empezar, se deben satisfacer las condiciones que estamos asumiendo sobre los
programas utilizadas este trabajo. Es decir, la asignación debe haber sido no aleatoria,
la participación en el programa dependió potencialmente de alguna tendencia vista en una
variable en períodos pasados, y hay una o más cohortes.

Una vez que se tiene cierta seguridad que los supuestos anteriores se cumplen, la manera
propuesta de seleccionar individuos de control con nuestra metodología es la siguiente:
\begin{enumerate}[itemsep=0.1cm, label=\textbf{\arabic*.}]
    \item Separar individuos tratados y no tratados, y para todos tomar los \(n\) períodos
    que haya disponibles antes del tratamiento\footnote{Como mencionamos anteriormente,
    según los resultados observados, mientras mayor sea este \(n\), mayores son las
    probabilidades de identificar buenos controles.}.
    \item Tomar \textbf{todos} los individuos tratados con etiqueta 1 y algunos de los no
    tratados con etiqueta 0, y repetir estos últimos tantas veces como cohortes haya.
    Luego, entrenar el modelo con todos estos\footnote{Lo ideal aquí sería entrenar el
    modelo tomando los valores de hiperparámetros que resultaron seleccionados con mayor
    frecuencia, como mostramos recién, según la arquitectura que se esté usando.}.
    \item Una vez que el modelo haya sido entrenado, hacer inferencia sobre todos los
    individuos no tratados, incluso con los que el modelo ha sido entrenado. Para llevar a
    cabo esta inferencia, habría que repetir todos estas unidades con los diferentes
    inicios de programa.
    \item Los individuos para los cuales el modelo haya inferido un 1 serán los controles
    y para los que haya inferido un 0 serán los NiNi. Puede ocurrir que el mismo individuo
    sirva como control para diferentes cohortes.
\end{enumerate}

Ahora bien, un detalle es que en el paso 2, lo que muy probablemente ocurra es que algunos
de los no tratados que ponemos en el conjunto de entrenamiento - a los cuales les
colocamos la etiqueta 0 - puedan servir como controles. Es por esto que en el paso 3
los incluimos en el conjunto sobre el cual se realiza la inferencia.

Otra opción es, en lugar de entrenar la red con todos los tratados, se puede separar
este grupo en dos, y usar una parte para entrenar la red, y la otra como \textit{proxy}
para verificar la correcta identificación de 1s.

Además, una vez hecha la inferencia, se puede realizar un chequeo para verificar que los
controles identificados sean similares a los tratados. Por ejemplo, a través de pruebas de
hipótesis de diferencia de medias entre los dos grupos en cada uno de los \(n\) períodos
que se hayan tomado. Si las diferencias resultan significativas en la mayoría de los
períodos, entonces el grupo de controles selecciones probablemente no sea válido.

\section{Trabajos futuros}
Una limitación de nuestro trabajo es que, al trabajar con datos sintéticos, sabemos
quiénes deberían ser identificados como controles y quiénes como NiNi. Sin embargo, como
mencionamos anteriormente, lo que ocurre en la realidad es que solamente tenemos
conocimiento sobre quiénes fueron tratados y quiénes no. Por lo tanto, siguiendo la
propuesta de aplicación, al entrenar el modelo utilizando algunos no tratados con etiqueta
0, es muy probable que se excluyan posibles controles. En esta línea, se podrían explorar
alternativas para evitar el uso de no tratados con etiqueta 0 durante el entrenamiento, y
en su lugar generar artificialmente individuos que sean claramente distintos de los
tratados.

Otra mejora que consideramos consiste en el uso de arquitecturas de redes neuronales más
recientes como los Transformers, que han demostrado tener un desempeño notable en tareas
para las que anteriormente se utilizaban redes recurrentes, como el procesamiento de
lenguaje natural.

Asimismo, sería interesante investigar la utilización de técnicas de aprendizaje no
supervisado, como el clustering, para evaluar si en los escenarios propuestos aquí,
resulta posible separar al grupo de los controles de los NiNi.

Por último, incorporar un mayor número de variables, tanto temporales como estáticas,
podría complejizar los escenarios simulados y permitir una evaluación más robusta del
desempeño de las arquitecturas propuestas.

\end{document}