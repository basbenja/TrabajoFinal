\documentclass[../main.tex]{subfiles}

\begin{document}

En este trabajo se exploró la utilización de redes neuronales para la selección de grupos
de control en evaluaciones de impacto sobre programas cuya asignación de beneficiarios es
no aleatoria, depende de resultados observados de la variable objetivo en períodos previos
al inicio del tratamiento, y en el que hay múltiples cohortes. Se evaluaron diferentes
arquitecturas de redes neuronales y se comparó su desempeño con el del PSM, un método
ampliamente utilizado para esta tarea. A continuación, desarrollamos las principales
conclusiones obtenidas a partir de los resultados, proponemos una forma de aplicar los
métodos explorados en casos reales, y mencionamos algunos aspectos de nuestro enfoque que
podrían investigarse en trabajos futuros.

\section{Conclusiones generales}
% Desempeño de las redes VS Desempeño del PSM
En primer lugar, el rendimiento de las redes neuronales en las métricas evaluadas superó
consistentemente al del PSM en todos los escenarios considerados. Este resultado fue
respaldado por pruebas de hipótesis que permitieron concluir que las diferencias
observadas fueron estadísticamente significativas, lo que refuerza la validez del enfoque
propuesto.

% Comparar las diferentes arquitecturas
Con respecto a la comparación entre las diferentes arquitecturas, contrario a nuestras
expectativas iniciales, con la que mejores resultados obtuvimos en la gran mayoría de los
escenarios fue la Convolucional. Para nuestra sorpresa, la combinación de LSTM +
Convolucional no trajo mejoras sobre la Convolucional, aunque sí sobre la LSTM. Esto
sugiere que, en general, las operaciones aplicadas por las redes convolucionales
permitieron una mejor captura de los patrones propuestos en los diferentes escenarios, y
que la incorporación de un bloque LSTM no tuvo un efecto significativo (al menos con la
arquitectura específica y el espacio de búsqueda de hiperparámetros propuestos).

Sin embargo, es importante destacar una restricción de la arquitectura Convolucional:
solamente puede procesar entradas de tamaño fijo, es decir secuencias de la misma
longitud. En cambio, la LSTM, como se explicó en la \hyperref[sec:recurrentes]{Sección
2.5}, es capaz de manejar secuencias de longitud variable. Esta característica la
convierte en una opción más flexible para su uso en casos reales donde no se cuenta con la
información completa de todos los períodos de tiempo considerados para todos los
individuos.

% Hiperparametros mas utilizados en las 100 simulaciones
Como observamos en las diferentes tablas de hiperparámetros, la tendencia general fue que
la tasa de aprendizaje más seleccionada fue de 0.001, el dropout de 0.3, y que los valores
de tamaños de lote elegidos con mayor frecuencia lo fueron en menos de la mitad de las
simulaciones. Algo a notar es que el valor de dropout 0.3 era el menor dentro del espacio
de búsqueda establecido, lo cual puede indicar que con valores aún menores podrían mejorar
el desempeño.

En relación al número de neuronas en las capas LSTM, el valor más frecuente en la
arquitectura LSTM fue de 128 en la mayoría de los casos, pero en la LSTM + Convolucional,
este valor tendió a reducirse. Este hecho refuerza la hipótesis que el procesamiento
realizado por las convoluciones resulta más efectivo que el llevado a cabo por las
neuronas LSTM. Sin embargo, como en el caso de la LSTM el valor más seleccionado (128) era
el más alto en el espacio de búsqueda, se podría considerar el uso de valores más altos.

% Hablar de la relación resultados - períodos - tendencia - proporciones
Teniendo en cuenta las características de los diferentes escenarios, podemos concluir que
las redes demostraron un rendimiento prometedor en aquellos donde había una alta cantidad
de períodos observados y las tendencias eran largas y ruidosas (experimentos
\hyperref[sec:exp1]{1} y \hyperref[sec:exp5]{5}), o cuando eran cortas y con un
comportamiento monótono (\hyperref[sec:exp3]{Experimento 3}). En cambio, en el escenario
donde no había un patrón claro en los comportamientos pasados, sino que era una
característica menos observable la que distinguía a tratados y controles de NiNi
(\hyperref[sec:exp4]{Experimento 4}), ninguno de los modelos mostró resultados aceptables.

Por otro lado, los valores obtenidos en los experimentos \hyperref[sec:exp6]{6} y
\hyperref[sec:exp7]{7}, donde la cantidad de períodos observados era baja (25 y 15
respectivamente), sugieren que el desempeño de las redes mejora a medida que aumenta la
proporción entre períodos de dependencia y períodos observados. Esta tendencia también
puede observarse al comparar los experimentos \hyperref[sec:exp1]{1} y
\hyperref[sec:exp2]{2}, lo cual refuerza nuestra hipótesis.

\section{Aplicación en casos reales}
En nuestros experimentos, conocemos de antemano cuáles son las unidades que deberían ser
identificadas como controles y cuáles como NiNi. Sin embargo, en una situación real, a la
hora de llevar a cabo una evaluación de impacto, solamente se conoce quiénes han sido
tratados y quiénes no; y dentro de este último grupo, están mezclados NiNi e individuos
que potencialmente podrían servir como controles. Esto nos llevó a preguntarnos cómo
podría aplicarse la metodología presentada en este trabajo en un entorno real. A
continuación, detallamos nuestra propuesta.

Para empezar, deben cumplirse los supuestos considerados en este trabajo respecto a los
programas evaluados. Es decir, la asignación a la intervención es no aleatoria, depende de
resultados observados en la variable objetivo en períodos anteriores al inicio del
programa, y existe una o más cohortes.

Una vez verificado el cumplimiento de estos supuestos, la manera que proponemos de aplicar
nuestra metodología para seleccionar individuos de control es la siguiente:
\begin{enumerate}[itemsep=0.1cm, label=\textbf{\arabic*.}]
    \item Separar los individuos tratados y no tratados, y para todos tomar los \(n\)
    períodos que haya disponibles antes del tratamiento\footnote{Como mencionamos
    anteriormente, según los resultados observados, mientras mayor sea este \(n\), mayores
    son las probabilidades de identificar buenos controles.}.
    \item Etiquetar a \textbf{todos} los individuos tratados con 1 y a un subconjunto de
    los no tratados con 0, repitiendo a estos últimos tantas veces como cohortes haya con
    los inicios de programa correspondientes. Luego, entrenar el modelo con todos
    estos\footnote{Lo ideal aquí sería entrenar el modelo tomando los valores de
    hiperparámetros que resultaron seleccionados como los óptimos con mayor frecuencia,
    como mostramos recién, según la arquitectura que se esté usando.}.
    \item Una vez que el modelo haya sido entrenado, realizar inferencia sobre todos los
    individuos no tratados, incluidos aquellos usados en el entrenamiento. Para llevar a
    cabo esta inferencia, habría que repetir todos estas unidades con los diferentes
    inicios de programa.
    \item Los individuos para los cuales el modelo haya inferido un 1 serán los controles
    y para los que haya inferido un 0 serán los NiNi. Puede ocurrir que el mismo individuo
    sirva como control para diferentes cohortes — esto sucedería si un mismo individuo es
    identificado como 1 pero con diferentes inicios de programa —.
\end{enumerate}

Un detalle importante es que en el paso \textbf{2}, lo que muy probablemente ocurra es que
algunos de los no tratados que colocamos en el conjunto de entrenamiento — a los cuales
etiquetamos con 0 — puedan servir como controles. Es por esto que en el paso \textbf{3}
habría que incluirlos en el conjunto sobre el cual se realiza la inferencia.

Una alternativa sería, en lugar de entrenar la red con todos los tratados, dividir este
grupo en dos, y usar una parte para entrenar la red, y la otra como \textit{proxy} para
verificar la correcta identificación de 1s antes de hacer la inferencia final.

Además, una vez realizada la inferencia, se pueden llevar a cabo chequeos manuales para
verificar que los controles identificados sean similares a los tratados. Por ejemplo, se
podrían hacer pruebas de hipótesis de diferencia de medias entre ambos grupos en cada uno
de los \(n\) períodos que se hayan tomado. Si las diferencias resultan significativas en
la mayoría de los períodos, entonces el grupo de controles seleccionado probablemente no
sea adecuado.

\section{Trabajos futuros}
Una limitación de nuestro trabajo es que, al trabajar con datos sintéticos, sabemos de
antemano quiénes deberían ser identificados como controles y quiénes como NiNi. Sin
embargo, como mencionamos anteriormente, lo que ocurre en la realidad es que solamente se
tiene conocimiento sobre quiénes fueron tratados y quiénes no. Por lo tanto, siguiendo la
propuesta de aplicación, al entrenar el modelo utilizando algunos no tratados etiquetados
como 0, es muy probable que se excluyan posibles controles válidos. En esta línea, se
podrían explorar alternativas para evitar el uso de no tratados con etiqueta 0 durante el
entrenamiento, y en su lugar generar artificialmente individuos que sean claramente
distintos de los tratados.

Otra mejora consiste en el uso de arquitecturas de redes neuronales más recientes como los
Transformers, que han demostrado tener un desempeño notable en tareas para las que
anteriormente se utilizaban redes recurrentes, como el procesamiento de lenguaje natural.

Asimismo, sería interesante investigar la utilización de técnicas de aprendizaje no
supervisado, como el clustering, para evaluar si en los escenarios propuestos aquí,
resulta posible separar al grupo de los controles de los NiNi sin etiquetas previas.

Un aporte valioso sería comparar la metodología propuesta con otras técnicas de
identificación de grupos de control distintas del PSM, como los controles sintéticos
\cite{abadie2025syntheticcontrolsexperimentaldesign}, las diferencias en diferencias para
múltiples períodos de tiempo \cite{CALLAWAY2021200}, y la regresión de dos vías de Mundlak
\cite{wooldridge-two-way-mundlak-2021}.

Por último, incorporar un mayor número de variables sobre cada individuo — tanto
temporales como estáticas — permitiría simular escenarios más complejos y evaluar de forma
más robusta el desempeño de las arquitecturas propuestas. Además, consideramos que incluir
variables exógenas que también varían con el tiempo y conforman el contexto en el que se
encuentran los individuos, como la tasa de interés, el índice de inflación, el
tipo de cambio de monedas extranjeras, el valor de los bonos, entre otras, podría
contribuir a que los modelos hagan una mejor identificación de controles.

\end{document}