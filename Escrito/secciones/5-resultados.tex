\documentclass[../main.tex]{subfiles}

\begin{document}
Llevamos a cabo cuatro experimentos, diferenciados entre sí por la cantidad de períodos
observados de los diferentes individuos y el tipo de tendencia observada en estos,
ambos parámetros establecidos a la hora de generar los datos.

Para cada experimento, corrimos 100 simulaciones con las diferentes arquitecturas,
llevando a cabo una búsqueda de hiperparámetros para cada una.

Tomamos como métrica objetivo ... Para obtener el resultado final de un determinado
modelo, tomamos el promedio de ... a lo largo de las diferentes simulaciones. Y para
compararlo con los resultados obtenidos con el PSM por cohortes ...

\subsection{Experimento 1}

\subsection{Experimento 2}

\subsection{Experimento 3}

\subsection{Experimento 4}

\end{document}