\documentclass[../main.tex]{subfiles}

\begin{document}
En lo que sigue, detallamos las diferentes características presentes en cada uno de
los escenarios junto con los resultados obtenidos con las diferentes arquitecturas
de redes neuronales como con el método de PSM.

Recordemos que para cada experimento realizamos 100 simulaciones por modelo, tanto para
las diferentes arquitecturas de redes como para el PSM. En el caso de las redes neuronales,
cada simulación incluyó una búsqueda de hiperparámetros sobre 16 combinaciones distintas,
realizada automáticamente por Optuna. El resultado reportado en cada simulación corresponde
a la red entrenada con la mejor combinación de hiperparámetros.

En los resultados, mostramos el puntaje \(F_1\) promedio obtenido en las 100 simulaciones
para cada arquitectura y el PSM. Además, de cada hiperparámetro, mostramos el valor
que resultó seleccionado con mayor frecuencia en las simulaciones.

Como mencionamos previamente, denotamos con:
\begin{itemize}
    \item \(L\) a la cantidad de períodos observados antes del inicio del programa.
    \item \(n_{pd}\) a la cantidad de períodos de dependencia, con \(0 \le n_{pd} < L\).
    \item \(m\) a la cantidad de veces, dentro de los períodos de dependencia, en los que
    permitimos que el valor de la variable sea mayor al inmediato anterior.
    \item \(c\) a la cantidad de cohortes.
    \item \(\mu_{EF_T}\), \(\mu_{EF_C}\) y \(\mu_{EF_{NiNi}}\) a la media de los efectos
    fijos de los grupos de tratados, controles y NiNis respectivamente.
\end{itemize}

\section{Experimento 1}
En este primer escenario, tomamos 45 períodos observados previos al inicio del programa
para cada individuo con 20 períodos de dependencia, en donde permitimos un máximo de 6
``subidas''. Es importante notar la proporción períodos de dependencia sobre períodos
observados, que en este caso es de 0.44 aproximadamente. Todos los parámetros utilizados
se encuentran en la Tabla \ref{tab:params_exp1}. Al tener 45 períodos observados y 3
cohortes, los inicios de programa son 46, 47 y 48.

\begin{table}[H]
    \centering
    \begin{tabular}{|c|c|c|c|c|c|c|}
        \hline
        \(L\) & \(n_{pd}\) & \(m\) & \(c\) & \(\mu_{EF_T}\) & \(\mu_{EF_C}\) & \(\mu_{EF_{NiNi}}\) \\ \hline\hline
        45 & 20 & 6 & 3 & 10 & 10 & 10 \\
        \hline
    \end{tabular}
    \caption{Parámetros para la generación de datos utilizados en el Experimento 1.}
    \label{tab:params_exp1}
\end{table}

La Tabla \ref{tab:results_exp1} muestra los resultados obtenidos en las métricas \(F_1\),
precisión y sensibilidad por las diferentes redes y el PSM. Cada valor representa el
promedio de esa métrica en las 100 simulaciones en el conjunto de test.

\begin{table}[H]
    \centering
    \renewcommand{\arraystretch}{1.2}
    \begin{tabular}{|c|c|c|c|}
        \hline
         & Puntaje \(F_1\) & Precisión & Sensibilidad \\ \hline\hline
        \textbf{LSTM} & & & \\ \hline
        \textbf{Convolucional} & & & \\ \hline
        \textbf{LSTM + Convolucional} & & & \\ \hline
        \textbf{PSM} & & & \\
        \hline
    \end{tabular}
    \caption{H}
    \label{tab:results_exp1}
\end{table}

La Tabla \ref{tab:hyperparams_exp1} muestra, para cada arquitectura, el valor de cada
hiperparámetro que fue seleccionado con mayor frecuencia como el óptimo duarante la
búsqueda en las 100 simulaciones. Cada celda contiene dicho valor y entre paréntesis el
porcentaje de simulaciones en el que dicho valor resultó ser el mejor, de acuerdo a la
optimización realizada por Optuna mediante validación cruzada.

Por último, se muestran las matrices de confusión promedio obtenidas en las 100 simulaciones
para cada arquitectura.

\begin{table}[H]
    \centering
    \renewcommand{\arraystretch}{1.2}
    \begin{tabular}{|c|c|c|c|c|}
        \hline
         & \makecell{Tamaño\\de lote}
           & \makecell{Neuronas en\\capas ocultas}
           & \makecell{Tasa de\\aprendizaje}
           & Dropout \\ \hline\hline
        \textbf{LSTM}                 & 128 (44\%) & 128 (69\%) & 0.001 (98\%) & 0.3 (54\%) \\ \hline
        \textbf{Convolucional}        & 32 (46\%)  &  -         & 0.001 (90\%) & 0.3 (66\%) \\ \hline
        \textbf{LSTM + Convolucional} & 32 (37\%)  & 64 (37\%)  & 0.001 (86\%) & 0.3 (72\%) \\
        \hline
    \end{tabular}
    \caption{H}
    \label{tab:hyperparams_exp1}
\end{table}


\section{Experimento 2}

\section{Experimento 3}

\section{Experimento 4}

\section{Experimento 5}

\section{Experimento 6}

\end{document}