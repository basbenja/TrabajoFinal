\documentclass[a4paper,12pt,spanish]{book}
\usepackage{preamble}

\begin{document}

% Formato: https://www.famaf.unc.edu.ar/documents/5247/ANEXO_II_RHCD-2024-572-E-UNC-DECFAMAF.pdf
\begin{titlepage}
\begin{center}
    % Configuración del espaciado
    \setstretch{1.5}

    \noindent
    \begin{minipage}[t]{0.5\linewidth}
        \flushleft
        \includegraphics[width=6cm]{logos/unc.png}
    \end{minipage}%
    \begin{minipage}[t]{0.5\linewidth}
        \flushright
        \includegraphics[width=6cm]{logos/famaf.png}
    \end{minipage}

    \vspace{2cm}

    {\LARGE \textbf{Aprendizaje Automático para la Selección de Grupos de Control en
    Evaluación de Impacto}}
    \\[1cm]
    {\large por}
    \\[0.5cm]
    {\Large \textbf{Benjamín Bas Peralta}}

    \vspace{1cm}

    \noindent
    Presentado ante la \textbf{FACULTAD DE MATEMÁTICA, ASTRONOMÍA, FÍSICA Y COMPUTACIÓN}
    como parte de los requerimientos para la obtención del grado de Licenciado en Ciencias
    de la Computación de la

    \textbf{UNIVERSIDAD NACIONAL DE CÓRDOBA}

    \vspace{0.5cm}

    Agosto, 2025

    \vspace{0.5cm}

    Directores: Dr. Martín Domínguez \\ Mgter. David Giuliodori

    \vfill

    \begin{center}
        \includegraphics[width=3cm]{logos/creative-commons.png}\\
        \vspace{0.3cm}
        Este trabajo se distribuye bajo una Licencia Creative Commons \\
        \href{https://creativecommons.org/licenses/by-nc-sa/4.0/deed.es}
        {Atribución - No Comercial - Compartir Igual 4.0 Internacional.}
    \end{center}

\end{center}
\end{titlepage}
\cleardoublepage

% Resumen - Palabras clave - Abstract - Keywords
\subfile{secciones/0-abstract.tex}
\clearpage

% Agradecimientos
\thispagestyle{empty}
\chapter*{Agradecimientos}
A David y Martín, por acompañarme a lo largo de estos meses enseñándome y guiándome,
y por su disposición a responder todas las dudas que fueran surgiendo.

A mi familia, Andrés, María, Sofi y Feli, por su comprensión y apoyo incondicional a lo
largo de toda la carrera, festejando mis logros y siempre motivándome a seguir adelante y
dar lo mejor de mí; por ayudarme cuando los tiempos no alcanzaban y por darme la
posibilidad de estudiar sin otras preocupaciones.

A Juli y Toni, mis amigos y compañeros desde el cursillo de ingreso hasta el último día de
cursado. Hicieron que este recorrido sea mucho más lindo; gracias por cada trabajo,
explicación, consejo, charla y momento compartido. Espero que dure para toda la vida.

A mis amigos en general, por siempre darme un espacio en donde poder distraerme
cuando aparecía el estrés, festejar mis logros, apoyarme y muchas veces hacerme ver
las cosas de otra manera.

A todos los profesores y ayudantes que me crucé en estos años. Admiro profundamente su
vocación y su esfuerzo constante por formar grandes profesionales y científicos, ayudando
a los alumnos en absolutamente todo. Ojalá siempre me toquen docentes como los de FAMAF.

A cada persona que pasó por mi vida durante este tiempo, y que de forma directa
o indirecta contribuyó a que hoy yo esté acá.

Por último, a la universidad pública, gratuita y de calidad; que me dio la oportunidad de
formarme con una educación de un altísimo nivel y que abre las puertas a todas las
personas para que puedan alcanzar sus sueños. Es un recurso invaluable que nos ofrece
nuestro país y que hay que defenderlo siempre.
\clearpage

\thispagestyle{empty}
\chapter*{Reconocimientos}
Este trabajo utilizó recursos computacionales de UNC Supercómputo (CCAD) de la Universidad
Nacional de Córdoba \cite{ccad}, que forman parte del Sistema Nacional de Computación de
Alto Desempeño (SNCAD) de la República Argentina.


\tableofcontents
\clearpage

\chapter{Introducción}
\subfile{secciones/1-introduccion.tex}

\chapter{Marco Teórico}
En este capítulo, presentamos los conceptos teóricos necesarios para entender el contexto
del problema y cuál es la solución que planteamos.

Primeramente, explicamos la evaluación de impacto, que es el campo de aplicación sobre el
que vamos a trabajar. Describimos qué es una evaluación de impacto, en qué consiste,
cuáles son los problemas involucrados, y las distintas técnicas que se utilizan para
llevarla a cabo. Al final de esta sección, presentamos el método de \textit{Propensity
Score Matching}, que es el que tomaremos como parámetro para evaluar nuestra solución
propuesta.

Luego, abordamos el área de la Inteligencia Artificial, enfocándonos particularmente
en el campo de los algoritmos de Aprendizaje Automático. Explicamos cómo funcionan estos y
cuáles son los elementos presentes en ellos haciendo especial enfásis en los algoritmos de
Aprendizaje Supervisado, dentro de los cuales se enmarca nuestra propuesta.

A continuación, hablamos sobre un \textit{modelo} particular dentro del aprendizaje
automático que ha cobrado particular importancia en los últimos años: las redes neuronales
artificiales. Partimos explicando las de tipo \textit{feedforward} para luego introducir
otras más específicas como son las convolucionales y las recurrentes, de las cuales
haremos uso en nuestros experimentos.

\section{Evaluación de impacto}
\subfile{secciones/2-marco-teorico/1-eval-impacto}

\section{Inteligencia Artificial}
\subfile{secciones/2-marco-teorico/2-ia}

\section{Redes Neuronales Artificiales} \label{sec:rna}
\subfile{secciones/2-marco-teorico/3-rna}

\section{Redes Neuronales Convolucionales}
\subfile{secciones/2-marco-teorico/4-convolucionales}

\section{Redes Neuronales Recurrentes}
\subfile{secciones/2-marco-teorico/5-recurrentes}

% \section{Clasificación de Series de Tiempo}
% \subfile{secciones/2-marco-teorico/6-series-de-tiempo}

\chapter{Presentación del problema} \label{sec:problema}
\subfile{secciones/3-problema}

\chapter{Marco Experimental}
El alcance del trabajo desarrollado comprende las siguientes etapas, de las cuales
hablaremos en este capítulo:
\begin{enumerate}[itemsep=0.1cm, label=\textbf{\arabic*.}]
    \item Generación de datos sintéticos.
    \item Diseño de arquitecturas de redes neuronales.
    \item Búsqueda de hiperparámetros y entrenamiento de las redes propuestas.
    \item Evaluación del modelo con diferentes métricas.
    \item Comparación con los resultados obtenidos con el PSM.
\end{enumerate}

También nombraremos las diferentes herramientas que nos ayudaron a llevar a cabo
cada uno de estos pasos.

\section{Conjuntos de Datos} \label{sec:datos}
\subfile{secciones/4-marco-experimental/1-datos}

\section{Arquitecturas de Redes Neuronales}
\subfile{secciones/4-marco-experimental/2-arquitecturas}

\section{Herramientas}
\subfile{secciones/4-marco-experimental/3-herramientas}

\section{Métricas} \label{sec:metricas}
\subfile{secciones/4-marco-experimental/4-metricas}

\section{Comparación de técnicas} \label{sec:comparacion}
\subfile{secciones/4-marco-experimental/5-comparacion.tex}

\chapter{Resultados} \label{sec:resultados}
\subfile{secciones/5-resultados/intro.tex}
\subfile{secciones/5-resultados/exp1.tex}

\subfile{secciones/5-resultados/exp2.tex}
\subfile{secciones/5-resultados/exp3.tex}
\subfile{secciones/5-resultados/exp4.tex}
\subfile{secciones/5-resultados/exp5.tex}
\subfile{secciones/5-resultados/exp6.tex}

\chapter{Conclusiones y Trabajos Futuros}
\subfile{secciones/6-conclusiones-trabajos-futuros.tex}

\newpage
\footnotesize
\nocite{python-docs}
\nocite{jupyter-docs}
\nocite{nabu}
\nocite{numpy-docs}
\nocite{pandas-docs}
\nocite{pytorch-docs}
\nocite{mlflow-docs}
\nocite{optuna-docs}
\printbibliography[heading=bibintoc]

\newpage
% \thispagestyle{empty} Los abajo firmantes, miembros del Tribunal de Evaluación de tesis,
% damos Fe que el presente ejemplar impreso, se corresponde con el aprobado por éste
% Tribunal \vfill
\newpage
\vfill
\addtocounter{page}{-1}
\clearpage
\thispagestyle{empty}
\phantom{a}
\vfill
\newpage
\vfill
\addtocounter{page}{-1}
\end{document}
