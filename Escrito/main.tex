\documentclass[a4paper,12pt,spanish]{book}
\usepackage{preamble}

\title{
    % Logos side by side
    \begin{minipage}{0.4\textwidth}
        \centering
        \includegraphics[width=\linewidth]{logos/famaf.png}
    \end{minipage}
    \hfill
    \begin{minipage}{0.4\textwidth}
        \centering
        \includegraphics[width=\linewidth]{logos/unc.png}
    \end{minipage}
    
    \textbf{
        Aprendizaje Automático para \\
        la Selección de Grupos de Control en \\
        Evaluación de Impacto
    } 
    \thanks{
        {
        {\protect\includegraphics[height=7mm]{logos/creative-commons.png}} \fontsize{10}{11}\selectfont  Esta obra está bajo una \href{https://creativecommons.org/licenses/by-nc-sa/4.0/}{Licencia Creative Commons Atribución-NoComercial-CompartirIgual 4.0 Internacional.}
        }
    }
}

\author{
    Autor: Benjam\'in Bas Peralta.\\
    Directores: Dr. Martín A. Domínguez, Mgter. David Giuliodori.\\ 
    Facultad de Matemática, Astronomía, Física y Computación\\
    Universidad Nacional de Córdoba
}

\begin{document}

% Carátula
\maketitle
\cleardoublepage

% Resumen
\section*{Resumen}
\subfile{secciones/0-abstract.tex}
\bigskip
\textbf{Palabras Clave}: Análisis de Redes Sociales, Aprendizaje Automático, Detección de influenciadores, Detección de comunidades, Modelos de predicción, Twitter
\clearpage

% Agradecimientos
\thispagestyle{empty}
\section*{Agradecimientos}
-- TO DO --
\clearpage

% Indice
\tableofcontents
\clearpage

\chapter{Introducción}
\subfile{secciones/1-introduccion.tex}

\chapter{Marco Teórico}

\section{Evaluación de Impacto}
\subfile{secciones/2-marco-teorico/1-eval-impacto}

\section{Inteligencia Artificial}
\subfile{secciones/2-marco-teorico/2-ia}

\section{Redes Neuronales Convolucionales}
\subfile{secciones/2-marco-teorico/3-convolucionales}

\section{Redes Neuronales Recurrentes}
\subfile{secciones/2-marco-teorico/4-recurrentes}

\section{Clasificación de Series de Tiempo}
\subfile{secciones/2-marco-teorico/5-clasif-series-de-tiempo}


\chapter{Presentación del problema}
\subfile{secciones/3-problema}


\chapter{Marco Experimental}

\section{Conjuntos de Datos}
\subfile{secciones/4-marco-experimental/1-datos}

\section{Arquitecturas de Redes Neuronales}
\subfile{secciones/4-marco-experimental/2-arquitecturas}

\section{Herramientas}
\subfile{secciones/4-marco-experimental/3-herramientas}

% \subsection{Clasificación de Series de Tiempo?}

\chapter{Resultados}


\chapter{Conclusiones y Trabajos Futuros}

\newpage
\footnotesize
% \bibliographystyle{agsm}
% \bibliography{refs}
\printbibliography

\newpage
% \thispagestyle{empty}
% Los abajo firmantes, miembros del Tribunal de Evaluación de tesis, damos Fe que el presente ejemplar impreso, se corresponde con el aprobado por éste Tribunal
% \vfill
\newpage
\vfill
\addtocounter{page}{-1}
\clearpage
\thispagestyle{empty}
\phantom{a}
\vfill
\newpage
\vfill
\addtocounter{page}{-1}
\end{document}
