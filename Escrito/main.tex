\documentclass[a4paper,12pt,spanish]{book}
\usepackage{preamble}

\begin{document}

% Formato: https://www.famaf.unc.edu.ar/documents/5247/ANEXO_II_RHCD-2024-572-E-UNC-DECFAMAF.pdf
\begin{titlepage}
\begin{center}
    % Configuración del espaciado
    \setstretch{1.5}

    \noindent
    \begin{minipage}[t]{0.5\linewidth}
        \flushleft
        \includegraphics[width=6cm]{logos/unc.png}
    \end{minipage}%
    \begin{minipage}[t]{0.5\linewidth}
        \flushright
        \includegraphics[width=6cm]{logos/famaf.png}
    \end{minipage}

    \vspace{2cm}

    {\LARGE \textbf{Aprendizaje Automático para la Selección de Grupos de Control en
    Evaluación de Impacto}}
    \\[1cm]
    {\large por}
    \\[0.5cm]
    {\Large \textbf{Benjamín Bas Peralta}}

    \vspace{1cm}

    \noindent
    Presentado ante la \textbf{FACULTAD DE MATEMÁTICA, ASTRONOMÍA, FÍSICA Y COMPUTACIÓN}
    como parte de los requerimientos para la obtención del grado de Licenciado en Ciencias
    de la Computación de la

    \textbf{UNIVERSIDAD NACIONAL DE CÓRDOBA}

    \vspace{0.5cm}

    Agosto, 2025

    \vspace{0.5cm}

    Directores: Dr. Martín Domínguez \\ Mgter. David Giuliodori

    \vfill

    \begin{center}
        \includegraphics[width=3cm]{logos/creative-commons.png}\\
        \vspace{0.3cm}
        Este trabajo se distribuye bajo una Licencia Creative Commons \\
        \href{https://creativecommons.org/licenses/by-nc-sa/4.0/deed.es}
        {Atribución - No Comercial - Compartir Igual 4.0 Internacional.}
    \end{center}

\end{center}
\end{titlepage}

\cleardoublepage

% Resumen - Palabras clave - Abstract - Keywords
\subfile{secciones/0-abstract.tex}
\clearpage

% Agradecimientos
\thispagestyle{empty}
\chapter*{Agradecimientos}
\clearpage

\tableofcontents
\clearpage

\chapter{Introducción}
\subfile{secciones/1-introduccion.tex}

\chapter{Marco Teórico}
En este capítulo, presentamos los conceptos teóricos necesarios para entender el contexto
del problema y cuál es la solución que planteamos.

Primeramente, explicamos la Evaluación de Impacto, que es el campo de aplicación sobre el
que vamos a trabajar. Describimos qué es una Evaluación de Impacto, en qué consiste,
cuáles son los problemas involucrados, y las distintas técnicas que se utilizan para
llevarla a cabo. Al final de esta sección, presentamos el método de \textit{Propensity
Score Matching}, que es el que tomaremos como parámetro para evaluar nuestra solución
propuesta.

Luego, nos abordamos al campo de la Inteligencia Artificial, enfocándonos particularmente
en el campo de los algoritmos de Aprendizaje Automático. Explicamos cómo funcionan estos y
cuáles son los elementos presentes en ellos haciendo especial enfásis en los algoritmos de
Aprendizaje Supervisado, dentro de los cuales se enmarca nuestra propuesta.

A continuación, hablamos sobre un \textit{modelo} particular dentro del Aprendizaje
Automático que ha cobrado particular importancia en los últimos años: las Redes Neuronales
Artificiales. Partimos explicando las de tipo \textit{Feedforward} para Luego introducir
otras más específicas como son las Convolucionales y las Recurrentes, de las cuales
haremos uso en nuestros experimentos.

Por último, y para combinar el método de solución propuesto junto con el área de
aplicación, hablamos sobre las series de tiempo. Mencionamos qué son y cuáles son los
parámetros involucrados en ellas, y repasamos los últimos trabajos llevados a cabo que
utilizan Inteligencia Artificial para resolver el problema de clasificación de series de
tiempo.

\section{Evaluación de Impacto}
\subfile{secciones/2-marco-teorico/1-eval-impacto}

\section{Inteligencia Artificial}
\subfile{secciones/2-marco-teorico/2-ia}

\section{Redes Neuronales Artificiales}
\subfile{secciones/2-marco-teorico/3-rna}


\section{Redes Neuronales Convolucionales}
\subfile{secciones/2-marco-teorico/4-convolucionales}

\section{Redes Neuronales Recurrentes}
\subfile{secciones/2-marco-teorico/5-recurrentes}

% \section{Clasificación de Series de Tiempo}
% \subfile{secciones/2-marco-teorico/6-series-de-tiempo}


\chapter{Presentación del problema} \label{sec:problema}
\subfile{secciones/3-problema}


\chapter{Marco Experimental}
El alcance del trabajo desarrollado está dado por las siguientes etapas, de las
cuales hablaremos en este capítulo:
\begin{enumerate}
    \item Generación de datos sintéticos.
    \item Transformación de datos.
    \item Diseño de modelos.
    \item Búsqueda de hiperparámetros.
    \item Entrenamiento del modelo.
    \item Evaluación del modelo con diferentes métricas.
    \item Comparación con los resultados obtenidos con el PSM.
\end{enumerate}

También nombraremos las diferentes herramientas que nos ayudaron a llevar a cabo
cada uno de estos pasos.

\section{Conjuntos de Datos}
\subfile{secciones/4-marco-experimental/1-datos}

\section{Arquitecturas de Redes Neuronales}
\subfile{secciones/4-marco-experimental/2-arquitecturas}

\section{Herramientas}
\subfile{secciones/4-marco-experimental/3-herramientas}

\section{Métricas}
\subfile{secciones/4-marco-experimental/4-metricas}

\chapter{Resultados}


\chapter{Conclusiones y Trabajos Futuros}

\newpage
\footnotesize
\nocite{python-docs}
\nocite{jupyter-docs}
\nocite{nabu}
\nocite{numpy-docs}
\nocite{pandas-docs}
\nocite{pytorch-docs}
\nocite{mlflow-docs}
\nocite{optuna-docs}
\printbibliography[heading=bibintoc]

\newpage
% \thispagestyle{empty} Los abajo firmantes, miembros del Tribunal de Evaluación de tesis,
% damos Fe que el presente ejemplar impreso, se corresponde con el aprobado por éste
% Tribunal \vfill
\newpage
\vfill
\addtocounter{page}{-1}
\clearpage
\thispagestyle{empty}
\phantom{a}
\vfill
\newpage
\vfill
\addtocounter{page}{-1}
\end{document}
